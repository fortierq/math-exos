\documentclass[10pt,a4paper]{article}
\usepackage{base}

\title{Colle PCSI 25: espace préhilbertiens}

\newcounter{question}
\newcommand{\initQ}{\setcounter{question}{0}}
\newenvironment{question}{\addtocounter{question}{1}
	\noindent {\it {Question} \thequestion.\ }}
{\par}
\newcounter{exo}
\newcommand{\initE}{\setcounter{exo}{0}}
\newenvironment{exo}{\vspace{0.5cm}\setcounter{question}{0}\addtocounter{exo}{1} \noindent \textbf{Exercice \theexo}. \normalsize }{\par}

\begin{document}
	\maketitle
	
	\section*{Colle 1}

	\setcounter{exo}{0}
	PATTE Rémi (9): ne connaît rien du tout sur Gram-Schmidt. Connaît très mal les définitions du cours.\\
	MONNIN Guillaume (11): ne se souvient pas de la définition de << libre >>\\
	
	\begin{exo}
		Gram-Schmidt.
	\end{exo}

	\begin{exo}
		Montrer qu'une famille orthogonale est libre.
	\end{exo}	

	\section*{Colle 2}

	\setcounter{exo}{0}
	PETIT (11): écrit des choses du genre $\vert \vert x + y \vert \vert = \vert \vert x \vert \vert + \vert \vert y \vert \vert$. Se trompe dans l'inégalité de Cauchy-Schwartz.\\
	MONTHILLER Thibaud (13): assez bien mais manque de rigueur.\\
	
	\begin{exo}
		Théorème de Pythagore.
	\end{exo}
		
	\begin{exo}
		Soit $x_1$, ..., $x_n$ des réels dont la somme vaut 1.
		Montrer que $\sum x_i^2 \geq \frac{1}{n}$.\\
		Quelle est la valeur minimum possible de $\sum x_i^2$?		
	\end{exo}	
	
	\section*{Colle 3}
	\setcounter{exo}{0}
	Ruben (17): connaît très bien son cours. Très efficace dans les exos.\\
	Réjane Gradelet (16): connait très bien Cauchy Schwartz, y compris la démo.
	
	\begin{exo}
		Thm de Cauchy Schwartz.
	\end{exo}

	\begin{exo}
		Orthonormaliser ... avec Gram-Schmidt.
	\end{exo}	

\end{document}
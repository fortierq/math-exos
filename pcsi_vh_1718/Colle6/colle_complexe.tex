\documentclass[10pt,a4paper]{article}
\usepackage[utf8]{inputenc}
\usepackage{amsmath}
\usepackage{amsfonts}
\usepackage{amssymb}
\usepackage{graphicx}
\title{Colle PCSI 6: nombres complexes}

\newcounter{question}
\newcommand{\initQ}{\setcounter{question}{0}}
\newenvironment{question}{\addtocounter{question}{1}
	\noindent {\it {Question} \thequestion.\ }}
{\par}
\newcounter{exo}
\newcommand{\Z}{\mathbb{Z}}

\newcommand{\initE}{\setcounter{exo}{0}}
\newenvironment{exo}{\vspace{0.5cm}\setcounter{question}{0}\addtocounter{exo}{1} \noindent \textbf{Exercice \theexo}. \normalsize }{\par}

\begin{document}
	\maketitle

	\section*{Colle 1}
	LADEIRA Ruben (11): ne sait pas déterminer une similitude géométriquement. \\
	BONNOT Alex (12): écrit plusieurs fois $e^{z}e^{z'} = e^{z z'}$...\\
	Cléo BASTIEN (11): ne sait pas bien traduire des conditions géométriques avec les complexes.\\
	
	\begin{exo} 
		Quelles sont les similitudes directes?
	\end{exo}
	
	\begin{exo}
		Quelle est la transformation donnée par :
		$f (z) = 3 - 2iz$
	\end{exo}
	\begin{exo}
		Solutions de $e^z = i$?
	\end{exo}

	\section*{Colle 2}
	\setcounter{exo}{0}
	NESPOULOUS Bastien (15): bien sauf une erreur de calcul\\
	BINET Mathilde (11): trop hésitante\\
	
	\begin{exo}
		Racines n-ième d’un complexe?
	\end{exo}

	\begin{exo}
		Déterminer pour tout $z \in C$ l’image de z par la rotation de centre $2 + 3i$ et d’angle de mesure $-\frac{\pi}{2}$.
	\end{exo}
	
	\begin{exo}
		résoudre $e^z$ + 2$e^{-z} = i$. (poser $Z = e^z$)
	\end{exo}
	\section*{Colle 3}
	\setcounter{exo}{0}
	PATTE Rémi (13): erreurs de calculs\\
	BOUGHAMMOURA Ahmed (16): très bien\\
	
	\begin{exo}
		COmment savoir si 3 nombres complexes sont alignés? orthogonaux?
	\end{exo}
	
	\begin{exo}
		Résoudre $z^2 - z = i - 1$
	\end{exo}
	
	\begin{exo}
		Quelle est la transformation donnée par :
		$f (z) = (1 + i) \times z + 1$
	\end{exo}
\end{document}
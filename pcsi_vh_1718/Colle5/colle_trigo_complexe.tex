\documentclass[10pt,a4paper]{article}
\usepackage[utf8]{inputenc}
\usepackage{amsmath}
\usepackage{amsfonts}
\usepackage{amssymb}
\usepackage{graphicx}
\title{Colle PCSI 3: Étude de fonctions et trigonométrie}

\newcounter{question}
\newcommand{\initQ}{\setcounter{question}{0}}
\newenvironment{question}{\addtocounter{question}{1}
	\noindent {\it {Question} \thequestion.\ }}
{\par}
\newcounter{exo}
\newcommand{\Z}{\mathbb{Z}}

\newcommand{\initE}{\setcounter{exo}{0}}
\newenvironment{exo}{\vspace{0.5cm}\setcounter{question}{0}\addtocounter{exo}{1} \noindent \textbf{Exercice \theexo}. \normalsize }{\par}

\begin{document}
	\maketitle

	\section*{Colle 1}
	ZGOUR Hajar (16): très bien\\
	Erwyn (15):\\
	
	\begin{exo} 
		\begin{itemize}
			\item Compatibilité du module avec les opérations (§ II.2)
			\item Enoncer l'inégalité triangulaire et cas d'égalité. Dessin. 
		\end{itemize}
	\end{exo}
	
	\begin{exo}
		Calculer $\cos(\frac{\pi}{8})$.
	\end{exo}
	
	\begin{exo}
		Résoudre l'équation:
		$$\cos(x) - \sqrt{3} \sin(x) = 1$$
	\end{exo}

	\section*{Colle 2}
	\setcounter{exo}{0}
	Thibaut TRONCIN (13): ne se souvient pas bien de la forme polaire/exponentielle, oublie les modulos\\
	Antonin Verjus (12): erreur dans les valeurs particulieres de cos et sin, ne connaît pas arccos.\\
	
	\begin{exo}
		\begin{itemize}
			\item Compatibilité de la conjugaison avec les opérations (§ I.4)
			\item Tout ce que tu sais sur l'argument d'un nombre complexe? Définitions, propriétés...
		\end{itemize}
	\end{exo}

	\begin{exo}
		Calculer $(1 + i\sqrt{3})^9$
	\end{exo}
	
	\section*{Colle 3}
	\setcounter{exo}{0}
	Arthur Thepenier (16): très bien\\
	Loris Vene (9): dit $(a+b)^2 = a^2 + b^2$, $\bar{1}$ = -1, me demande est-ce que $e^{2i\pi} = 1$...\\
	
	\begin{exo}
		\begin{itemize}
			\item Argument d'un produit et corollaire (inverse et quotient seulement) (§ II.3)
			\item Tout ce que tu sais sur l'exponentielle d'un nombre complexe? Définitions, propriétés...
		\end{itemize}
	\end{exo}
	
	\begin{exo}
		Résoudre l'équation:
		$$\cos(2x) = \cos^2(x)$$
	\end{exo}
	
	\begin{exo}
		résoudre $e^z$ + 2$e^{-z} = i$. (poser $Z = e^z$)
	\end{exo}
\end{document}
\documentclass[10pt,a4paper]{article}
\usepackage[utf8]{inputenc}
\usepackage{amsmath}
\usepackage{amsfonts}
\usepackage{amssymb}
\usepackage{graphicx}
\title{Colle PCSI 8: Sommes, fonctions trigonométriques et hyperboliques.}

\newcounter{question}
\newcommand{\initQ}{\setcounter{question}{0}}
\newenvironment{question}{\addtocounter{question}{1}
	\noindent {\it {Question} \thequestion.\ }}
{\par}
\newcounter{exo}
\newcommand{\Z}{\mathbb{Z}}

\newcommand{\initE}{\setcounter{exo}{0}}
\newenvironment{exo}{\vspace{0.5cm}\setcounter{question}{0}\addtocounter{exo}{1} \noindent \textbf{Exercice \theexo}. \normalsize }{\par}

\begin{document}
	\maketitle
	
	
	\section*{Colle 1}
	\setcounter{exo}{0}
	MFOUKA Tomessa (14): écrit $y^2 = x^2 \implies y = x$ sans plus de justifications. Sinon bien\\
	GAFFET Axelle (11): ne se souvient pas de la formule de la dérivée de la composée\\
	
	\begin{exo}
		Dessin de $\sin$ et $\arcsin$. Dérivabilité et dérivée de la fonction $\arcsin$. 
	\end{exo}

	\begin{exo}
		Minimum de $x \longmapsto x \ln(x)$ sur $\mathbb{R}^{+*}$?
	\end{exo}

	\begin{exo}
		Calculer $(1 + i\sqrt{3})^9$
	\end{exo}


	\section*{Colle 2}
	\setcounter{exo}{0}
	LEROY Adrien (16): Très bien sauf pour trouver l'inverse d'une fonction.\\
	GOUX Alexandre (13): connaît la définition d'une fct surj mais ne sais pas bien la manipuler en pratique\\
	
	\begin{exo}
	    Définition d'une fonction inj, surj, bij. Dérivée de la réciproque?
	\end{exo}
		
	\begin{exo}
		Montrer que $\arctan(x) + \arctan(\frac{1}{x}) = \frac{\pi}{2}$, $\forall x > 0$.
	\end{exo}
	
	\begin{exo}
		Montrer que $f: x \longmapsto \frac{x}{1-x^2}$ est bijective de $]-1, 1[$ dans $\mathbb{R}$ et exprimer sa bijection réciproque.
	\end{exo}	
			
	\section*{Colle 3}
	\setcounter{exo}{0}
	MONTEIL Anicia (8): ne maitrise pas bien la notion de dérivée ni celle de majorant/minorant. Fait des erreurs très basiques genre dérivée de ln(1+x), 1+x $<$ 1 $\Leftrightarrow$ x $<$ 1...\\
	GODEAU Victor (15): bien sauf pour trouver l'inverse d'une fonction\\
	
	\begin{exo}
		Définition minorant, majorant, max, min. Mq $sin x \leq x$ 
	\end{exo}

	\begin{exo}
		Minimum de $x \longmapsto x + \frac{1}{x}$ sur $\mathbb{R}^{+*}$?
	\end{exo}

	\begin{exo}
		Mq $\ln(1+x) \leq x$ puis: $$(1 + \frac{1}{n})^n \leq e$$
	\end{exo}
	
\end{document}
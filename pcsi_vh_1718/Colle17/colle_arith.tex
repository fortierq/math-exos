\documentclass[10pt,a4paper]{article}
\usepackage[utf8]{inputenc}
\usepackage{base}
\title{Colle PCSI 17: arithmétique}

\newcounter{question}
\newcommand{\initQ}{\setcounter{question}{0}}
\newenvironment{question}{\addtocounter{question}{1}
	\noindent {\it {Question} \thequestion.\ }}
{\par}
\newcounter{exo}

\newcommand{\initE}{\setcounter{exo}{0}}
\newenvironment{exo}{\vspace{0.5cm}\setcounter{question}{0}\addtocounter{exo}{1} \noindent \textbf{Exercice \theexo}. \normalsize }{\par}

\begin{document}
	\maketitle
\section*{Colle 1}
\setcounter{exo}{0}
	TRONCIN Thibaud (16): bien mais ne pense pas assez à simplement utiliser les définitions \\
	
	
	\begin{exo}
		Déterminer les nb premiers $\leq 50$ avec la méthode d'Eratosthène.
	\end{exo}
	
	\begin{exo}
		Soient $f_n$ les termes de la suite de Fibonacci.
		\begin{enumerate}
			\item Mq $f_{n+1} f_{n-1} - f_n^2 = (-1)^n$
			\item Mq $f_{n+1}$ et $f_n$ sont premiers entre eux
		\end{enumerate}
	\end{exo}

	\section*{Colle 2}
	
	GUYOT Marion (13): $a \vert b$ et $a \vert c$ $\implies$ $a \vert PGCD(b, c)$ mal connu. Du mal sur l'exo.\\
	
	\begin{exo}
		Énoncer le thm de décomposition facteurs premiers.
	\end{exo}
			
	\begin{exo}
		soient $a$ et $b$ premiers entre eux et $c\in  \Z$.\\
		Mq PGCD$(a, bc)$ = PGCD$(a, c)$
	\end{exo}	

	\begin{exo}
		mq $n \geq 2$ et $2^n - 1$ premier $\implies$ $n$ premier.
	\end{exo}	


	\section*{Colle 3}
	\setcounter{exo}{0}
	Arthur Thépenier (16): bien mais essaie des choses parfois trop compliquées. Écrit $d \vert a + b \Longleftrightarrow d \vert a$ et $d \vert b$.\\
	VERJUS Antonin (13): écrit $d \vert a + b \Longleftrightarrow d \vert a$ et $d \vert b$. Des difficultés sur l'exo.
	
	\begin{exo}
		Résoudre $37x+23y = 1$
	\end{exo}
	
	\begin{exo}
		Soient $a$, $b$ deux entiers premiers entre eux. Mq $a+b$ et $ab$ sont premiers entre eux.
	\end{exo}
	
\end{document}
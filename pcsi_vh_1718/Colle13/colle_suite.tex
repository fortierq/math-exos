\documentclass[10pt,a4paper]{article}
\usepackage[utf8]{inputenc}
\usepackage{base}
\title{Colle PCSI 13: suites}

\newcounter{question}
\newcommand{\initQ}{\setcounter{question}{0}}
\newenvironment{question}{\addtocounter{question}{1}
	\noindent {\it {Question} \thequestion.\ }}
{\par}
\newcounter{exo}

\newcommand{\initE}{\setcounter{exo}{0}}
\newenvironment{exo}{\vspace{0.5cm}\setcounter{question}{0}\addtocounter{exo}{1} \noindent \textbf{Exercice \theexo}. \normalsize }{\par}

\begin{document}
	\maketitle

	\section*{Colle 1}
	ZGOUR Hajar (14): bien\\
	
	\begin{exo}
		Terme général de $u_{n+1} = 3 u_n + 2$, $u_0 = 1$?
	\end{exo}	
	
	\begin{exo}
		mq $\sin(n)$ n'a pas de limite
	\end{exo}
	
	\section*{Colle 2}
	\setcounter{exo}{0}
	TRONCIN Thibaud (11): propose une solution très manifestement fausse. Manque de rigueur et de précision. \\
	DERRAR Youri (8): très grosses erreurs de calculs. Ne connaît pas bien le cours.\\
	
	\begin{exo}
		Terme général de $u_{n+1} = 4 u_n - 1$, $u_0 = 0$?
	\end{exo}

	\begin{exo}
		limite de $u_0 = 3$, $u_{n+1} = \frac{1}{u_n} + \frac{un}{2}$?
	\end{exo}
	
	\begin{exo}
		On définit $u_0 \in [0, \frac{\pi}{2}]$, $u_{n+1} = \sin(u_n)$. Est-ce que $u_n$ a une limite? laquelle?
	\end{exo}
	
	\section*{Colle 3}
	\setcounter{exo}{0}
	Arthur Thepenier (15): Très bien.\\
	DERET Simon (8): Ne connaît pas le cours.\\
	
	\begin{exo}
		Terme général de $u_{n+2} = u_{n+1} + u_{n-1}$, $u_0 = 0$, $u_1 = 1$?
	\end{exo}	

	\begin{exo}
		limite de $u_{n+1} = 1 + u_n^2$?
	\end{exo}
	
	\begin{exo}
		trouver $u$ tq $u_0 =0$, $u_1 = 0$ et $u_{n+2} = 5u_{n+1} - 6u_n + 2n^2$: chercher solution particulière $v_n = n^2 + an + b$ puis poser $w_n = u_n - v_n$.
	\end{exo}		
\end{document}
\documentclass[10pt,a4paper]{article}
\usepackage{base}

\title{Colle PCSI 23: EV}

\newcounter{question}
\newcommand{\initQ}{\setcounter{question}{0}}
\newenvironment{question}{\addtocounter{question}{1}
	\noindent {\it {Question} \thequestion.\ }}
{\par}
\newcounter{exo}
\newcommand{\initE}{\setcounter{exo}{0}}
\newenvironment{exo}{\vspace{0.5cm}\setcounter{question}{0}\addtocounter{exo}{1} \noindent \textbf{Exercice \theexo}. \normalsize }{\par}

\begin{document}
	\maketitle
	
	\section*{Colle 1}
	\setcounter{exo}{0}
	TRONCIN Thibaud (13): rechigne à faire les démonstrations. Problèmes de quantificateurs. Bien sinon.\\
	KHALIL (15): bien mais pourrait être mieux rédigé.
	
	\begin{exo}
		Est-ce que l'ensemble des suites croissantes/monotones/convergentes/bornée est un EV?
	\end{exo}	
	
	\begin{exo}
		Est-ce que les $x \longmapsto \vert x - n \vert$ forment une famille libre, pour $n \in \s{1, ..., 100}$?
	\end{exo}	
	
	\section*{Colle 2}
	\setcounter{exo}{0}
	Arthur Thépenier (17): pense à utiliser des racines de polynômes (même si ce n'était pas la méthode que je voulais lui faire trouver). Très bien.\\
	GUYOT Marion (15): A la bonne idée de procéder par récurrence. Petit problème de logique dans le raisonnement par l'absurde. Biens inon. \\
	
	\begin{exo}
		Mq $(1)$, $(n)$, $(n^2)$ sont indépendantes. Et pour $(1), (n), ..., (n^k)$?
	\end{exo}

	\section*{Colle 3}
	\setcounter{exo}{0}
	Antonin V (13): Assez bien mais lent (en particulier pour trouver des contre-exemples).\\
	
	\begin{exo}
		Est-ce que l'ensemble des fonctions impaires/monotones/qui s'annulent/qui s'annulent en 0 est un EV?
	\end{exo}	
\end{document}pmlo,knjbhv-g(cf'xdsq)
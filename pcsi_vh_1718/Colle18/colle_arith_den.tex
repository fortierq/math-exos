\documentclass[10pt,a4paper]{article}
\usepackage[utf8]{inputenc}
\usepackage{base}
\title{Colle PCSI 18: arithmétique et dénombrement}

\newcounter{question}
\newcommand{\initQ}{\setcounter{question}{0}}
\newenvironment{question}{\addtocounter{question}{1}
	\noindent {\it {Question} \thequestion.\ }}
{\par}
\newcounter{exo}

\newcommand{\initE}{\setcounter{exo}{0}}
\newenvironment{exo}{\vspace{0.5cm}\setcounter{question}{0}\addtocounter{exo}{1} \noindent \textbf{Exercice \theexo}. \normalsize }{\par}

\begin{document}
	\maketitle
\section*{Colle 1}
\setcounter{exo}{0}
	GODEAUX Victor (14): assez bien\\
	El Abbassi Myriam (16): très rapide. Quelques confusions sur la quantification des variables.\\
	
	\begin{exo}
		Déterminer les nb premiers $\leq 50$ avec la méthode d'Eratosthène.
	\end{exo}
	
	\begin{exo}
		Combien y a t-il de sous-ensembles de taille pair d'un ensemble à $n$ éléments? Et de taille un multiple de 3?
	\end{exo}

	\section*{Colle 2}
	Monteil Alicia (10): ne connait pas du tout le thm de décomposition en facteur premier. lente.\\
	FERNANDEZ Henri (10): ne connaît pas la méthode de résolution d'une équation sur des entiers\\
	
	\begin{exo}
		Énoncer le thm de décomposition facteurs premiers.
	\end{exo}
			
	\begin{exo}
		soient $a$ et $b$ premiers entre eux et $c\in  \Z$.\\
		Mq PGCD$(a, bc)$ = PGCD$(a, c)$
	\end{exo}	

	\begin{exo}
		Soit $E = \lbrace 1, ..., n \rbrace$. Calculer $\sum_{X \subset E} \vert X \vert$, $\sum_{X, Y \subset E} \vert X \cap Y \vert$
	\end{exo}	


	\section*{Colle 3}
	\setcounter{exo}{0}
	GOUX Alexandre (15): ne se souvient pas de l'algo d'Euclide étendu. Écrit $n 2^n = 2^{n+1}$. Mais aussi de très bonnes idées.\\
	DERRAR Youri (11): lent et connaissances imprécises.\\
	\begin{exo}
		Résoudre $37x+23y = 1$
	\end{exo}
	
	\begin{exo}
		Soient $a$, $b$ deux entiers premiers entre eux. Mq $a+b$ et $ab$ sont premiers entre eux.
	\end{exo}
	
\end{document}
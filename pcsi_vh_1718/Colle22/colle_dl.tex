\documentclass[10pt,a4paper]{article}
\usepackage[utf8]{inputenc}
\usepackage{base}
\title{Colle PCSI 22: DL}

\newcounter{question}
\newcommand{\initQ}{\setcounter{question}{0}}
\newenvironment{question}{\addtocounter{question}{1}
	\noindent {\it {Question} \thequestion.\ }}
{\par}
\newcounter{exo}

\newcommand{\initE}{\setcounter{exo}{0}}
\newenvironment{exo}{\vspace{0.5cm}\setcounter{question}{0}\addtocounter{exo}{1} \noindent \textbf{Exercice \theexo}. \normalsize }{\par}

\begin{document}
	\maketitle
\section*{Colle 1}
\setcounter{exo}{0}
	Myriam EL Abassi (17): excellent\\
	Loris (16): très bien.\\
	\begin{exo}
		Question de cours
	\end{exo}
	
	\begin{exo}
		Dérivée $n$ème de $\frac{1}{1 - x}$? $\frac{1}{1 + x}$?
 	\end{exo}

	\begin{exo}
		$DL_0$ ordre 3 de $\cos(x) e^x$
	\end{exo}

	\begin{exo}
		calcul de limite
	\end{exo}
	
	\section*{Colle 2}
	Henri FERNANDEZ (15): n'utilise pas un DL à un ordre assez grand. Erreurs de calculs simples.  \\
	DIEULOT Agathe (11): ne sait pas utiliser le DL de $(1+x)^\alpha$\\
	
	\begin{exo}
		Question de cours
	\end{exo}
	
	\begin{exo}
		$DL_0$ ordre 4 de $(\ln(1+x))^2$ 
	\end{exo}

	\section*{Colle 3}
	\setcounter{exo}{0}
	Youri DERRAR (7): ne connaît pas le DL de $\ln(1-x)$ et n'arrive pas à le retrouver.\\
	JACQUEMARD Steven (15): Bien.\\
	
	\begin{exo}
		Question de cours
	\end{exo}
			
	\begin{exo}
		$DL_0$ ordre 4 de $(\ln(\cos(x)))$ 
	\end{exo}
	
\end{document}
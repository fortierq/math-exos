\documentclass[10pt,a4paper]{article}
\usepackage{base}

\title{Colle PCSI 30: probabilités}

\newcounter{question}
\newcommand{\initQ}{\setcounter{question}{0}}
\newenvironment{question}{\addtocounter{question}{1}
	\noindent {\it {Question} \thequestion.\ }}
{\par}
\newcounter{exo}
\newcommand{\initE}{\setcounter{exo}{0}}
\newenvironment{exo}{\vspace{0.5cm}\setcounter{question}{0}\addtocounter{exo}{1} \noindent \textbf{Exercice \theexo}. \normalsize }{\par}

\begin{document}
	\maketitle
	
	\section*{Colle 1}
	GODEAU Victor (14): erreurs dans la formule de Markov. bien sinon.\\
	NESPOULOS (17): bien, bonnes manipulations de sommes\\
	
	\setcounter{exo}{0}
	
	\begin{exo}
		formules de Markov et Bienaymé-Tchebichev
	\end{exo}
	\begin{exo}
		Soit $X$ une variable aléatoire à valeur dans $\s{0, ..., N}$. Montrer que $E(X) = \sum_{k=0}^{N-1} P(X > k)$.\\
		En déduire la loi de  $min(X, Y)$ si $X$ et $Y$ sont deux lois uniformes.
	\end{exo}

	\section*{Colle 2}
	MONTEIL Alicia (12): a du mal à dénombrer toutes les possibilités.\\
	LEROY (15): a mal appliqué la formule du binome de newton. Bien sinon.\\
	
	\setcounter{exo}{0}

	\begin{exo}
		Thm de transfert.
	\end{exo}

	\begin{exo}
		On lance 2 dés. Est-ce que << un des dés tombe sur 1 >> et << la somme des 2 dés vaut 7 >> sont indépendant? Et si on remplace 1 par 6? 
	\end{exo}	
		
	\begin{exo}
		Est-ce qu'un évènement peut etre indépendant de lui-même?
	\end{exo}	
	
	\section*{Colle 3}
	\setcounter{exo}{0}
	GOUX Alexandre (15): bien\\
	MOTTET (14): ne se souvient pas de la formule de Markov. Assez bien sinon.\\
	
	\begin{exo}
		Définition, espérance et variance d'une loi binomiale.
	\end{exo}

	\begin{exo}
		Soit X et Y deux variables aléatoires indépendantes suivant des lois de
		Bernoulli de paramètres p et q . Déterminer la loi de la variable
		Z = max(X, Y ) .
	\end{exo}
	
\end{document}
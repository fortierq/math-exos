\documentclass[10pt,a4paper]{article}
\usepackage[utf8]{inputenc}
\usepackage{base}
\title{Colle PCSI 11: bornes, partie entiere, inégalités}

\newcounter{question}
\newcommand{\initQ}{\setcounter{question}{0}}
\newenvironment{question}{\addtocounter{question}{1}
	\noindent {\it {Question} \thequestion.\ }}
{\par}
\newcounter{exo}

\newcommand{\initE}{\setcounter{exo}{0}}
\newenvironment{exo}{\vspace{0.5cm}\setcounter{question}{0}\addtocounter{exo}{1} \noindent \textbf{Exercice \theexo}. \normalsize }{\par}

\begin{document}
	\maketitle

	\section*{Colle 1}
	MFAUKA Tomessa (13): écrit réel $\leq$ ensemble\\
	VENNE (14): bien\\
	
	\begin{exo}
		mq Inf $\floor{x} + \floor{\frac{1}{x}}$ = 1.
	\end{exo}
	
	\begin{exo}
		Borne inf et sup de $E = \lbrace \frac{1}{2^n} + \frac{(-1)^n}{n} \rbrace$?
	\end{exo}
	\section*{Colle 2}
	\setcounter{exo}{0}
	LEROY Adrien (14): bien\\
	VERJUS Antonin (12): trop lent, hésitant\\
	
	\begin{exo}
		Convertir 0,20172017...
	\end{exo}

	\begin{exo}
		Soit $f : \mathbb{R} \longmapsto \mathbb{R}$ continue tq $f(x + y) = f(x) + f(y)$.  mq $f(x) = C x$.
	\end{exo}	
	
	
	\section*{Colle 3}
	\setcounter{exo}{0}
	MONTEL Alicia (10): écrit $(-1)^n n^2 = n^{n+2}$...\\
	ROMAND Erwyn (14): Bien\\
	
	\begin{exo}
		Sup $\s{\vert x - y \vert, ~~x, y \in A}$?
	\end{exo}

	\begin{exo}
		Borne inf et sup de $E = \lbrace \frac{1}{n^2} + {(-1)^n} \rbrace$?
	\end{exo}		
\end{document}
\documentclass[10pt,a4paper]{article}
\usepackage[utf8]{inputenc}
\usepackage{amsmath}
\usepackage{amsfonts}
\usepackage{amssymb}
\usepackage{graphicx}
\title{Colle PCSI 15: suites.}

\newcounter{question}
\newcommand{\initQ}{\setcounter{question}{0}}
\newenvironment{question}{\addtocounter{question}{1}
	\noindent {\it {Question} \thequestion.\ }}
{\par}
\newcounter{exo}
\newcommand{\Z}{\mathbb{Z}}

\newcommand{\initE}{\setcounter{exo}{0}}
\newenvironment{exo}{\vspace{0.5cm}\setcounter{question}{0}\addtocounter{exo}{1} \noindent \textbf{Exercice \theexo}. \normalsize }{\par}

\begin{document}
	\maketitle
	
	
	\section*{Colle 1}
	\setcounter{exo}{0}

	\begin{exo}
		Borne inf et sup de $E = \lbrace \frac{1}{2^n} + \frac{(-1)^n}{n} \rbrace$?
	\end{exo}

	\begin{exo}
		Limite de $u_0 \in \mathbb{R}^+$, $u_{n+1} = \frac{1}{2} (u_n +\frac{a}{u_n})$?
	\end{exo}
	
	\section*{Colle 2}
	\setcounter{exo}{0}
	
	\begin{exo}
		Suite récurrente d'ordre 2 avec second membre.
	\end{exo}		
	
	\section*{Colle 3}
	\setcounter{exo}{0}
		
	\begin{exo}
		Propriétés caractéristiques de la partie entière.
	\end{exo}
	
	\begin{exo}
		Terme général de $u_{n+1} = 3 u_n + 2$, $u_0 = 1$?
	\end{exo}	

	\begin{exo}
		On définit $u_0 \in [0, \frac{\pi}{2}]$, $u_{n+1} = \sin(u_n)$. Est-ce que $u_n$ a une limite? laquelle?
	\end{exo}	
	
\end{document}
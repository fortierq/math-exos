\documentclass[10pt,a4paper]{article}
\usepackage{base}

\title{Colle PCSI 27: applications linéaires}

\newcounter{question}
\newcommand{\initQ}{\setcounter{question}{0}}
\newenvironment{question}{\addtocounter{question}{1}
	\noindent {\it {Question} \thequestion.\ }}
{\par}
\newcounter{exo}
\newcommand{\initE}{\setcounter{exo}{0}}
\newenvironment{exo}{\vspace{0.5cm}\setcounter{question}{0}\addtocounter{exo}{1} \noindent \textbf{Exercice \theexo}. \normalsize }{\par}

\begin{document}
	\maketitle
	
	\section*{Colle 1}

	\setcounter{exo}{0}
	KHALIL (17): très bien.\\
	TRONCIN (11): ne connaît pas la méthode pour trouver une base de l'image d'un endomorphisme\\
	
	\begin{exo}
		Définition d'une matrice de changement de base et formule de changement de base.
	\end{exo}

	\begin{exo}
		Déterminer base du ker et im d'une application linéaire.
	\end{exo}	

	\begin{exo}
		Mq $Ker f \cap Im f = \lbrace 0 \rbrace$ $\Longleftrightarrow$ $Ker f^2 = Ker f$
	\end{exo}
	
	\section*{Colle 2}

	\setcounter{exo}{0}
	VERJUS (13): assez bien mais manque de confiance en soi\\
	Arthur (16): Bien\\
	
	\begin{exo}
		Définition et propriétés du noyau.
	\end{exo}
		
	\begin{exo}
		Changer de base la matrice d'une application linéaire.	
	\end{exo}	
	
	\begin{exo}
		Soient $f$, $g$ deux endo qui commutent. Mq $Ker(g)$ est stable par $f$.
	\end{exo}
	
	\section*{Colle 3}
	\setcounter{exo}{0}
	GUYOT (15): petite erreur dans le pivot de gauss. Bien sinon.\\
	
	\begin{exo}
		Thm du rang.
	\end{exo}

	\begin{exo}
		Déterminer base du ker et im d'une application linéaire.
	\end{exo}	

	\begin{exo}
		Mq $E = Ker f + Im f$ $\Longleftrightarrow$ $Im f^2 = Im f$
	\end{exo}
\end{document}
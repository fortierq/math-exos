\documentclass[10pt,a4paper]{article}
\usepackage[utf8]{inputenc}
\usepackage{base}
\title{Colle PCSI 19: continuité}

\newcounter{question}
\newcommand{\initQ}{\setcounter{question}{0}}
\newenvironment{question}{\addtocounter{question}{1}
	\noindent {\it {Question} \thequestion.\ }}
{\par}
\newcounter{exo}

\newcommand{\initE}{\setcounter{exo}{0}}
\newenvironment{exo}{\vspace{0.5cm}\setcounter{question}{0}\addtocounter{exo}{1} \noindent \textbf{Exercice \theexo}. \normalsize }{\par}

\begin{document}
	\maketitle
\section*{Colle 1}
\setcounter{exo}{0}
	PETIT Laurine (11): ne connaît pas bien les définitions de limite et continuité (mélange les variables et quantificateurs)\\
		
	\begin{exo}
		Montrer que si $f : \R \longrightarrow \R$ continue a une limite 1 en $\infty$ et -1 en $-\infty$ alors $f$ s'annule.
 	\end{exo}

	\section*{Colle 2}
	PATTE Rémi (15): trouve tout seul le 1er exo. Manque de rigueur sur les quantificateurs. \\
	Arthur THEPENIER (16): bien\\
	
	\begin{exo}
		$f : [0, 1] \rightarrow [0, 1]$ continue $\Longrightarrow$ $f$ a un point fixe
	\end{exo}

	\section*{Colle 3}
	\setcounter{exo}{0}
	Ruben (11): inverse croissant et décroissant. Pense que la fonction doit être injective/surjective pour appliquer le TVI.\\
	TRONCIN Thibaud (13): inverse croissant et décroissant. Pense qu'une fonction de R dans R doit prendre toutes les valeurs réelles (être surjective).\\
	
	\begin{exo}
		Equivalent de $\tan(x)$ en 0?
	\end{exo}
	
	\begin{exo}
		$f$ continue décroissante sur $\R$ $\implies$ $f$ a un point fixe
	\end{exo}
	
\end{document}
\documentclass[10pt,a4paper]{article}
\usepackage[utf8]{inputenc}
\usepackage{base}
\title{Colle PCSI 13: suites}

\newcounter{question}
\newcommand{\initQ}{\setcounter{question}{0}}
\newenvironment{question}{\addtocounter{question}{1}
	\noindent {\it {Question} \thequestion.\ }}
{\par}
\newcounter{exo}

\newcommand{\initE}{\setcounter{exo}{0}}
\newenvironment{exo}{\vspace{0.5cm}\setcounter{question}{0}\addtocounter{exo}{1} \noindent \textbf{Exercice \theexo}. \normalsize }{\par}

\begin{document}
	\maketitle

	\section*{Colle 1}
	MONNIN Guillaume (12): moyen\\
	EL Abbassi Myriam (12): ne connaît pas bien la méthode\\
	
	\begin{exo}
		Limite d'une suite $u_{n+1} = f(u_n)$.
	\end{exo}	
	
	\section*{Colle 2}
	\setcounter{exo}{0}
	Hugo MOTTET (17): très bien\\
	FRANCOIS Léonard (11): ne connaît pas bien la méthode\\
	
	\begin{exo}
		Limite d'une suite $u_{n+1} = f(u_n)$.
	\end{exo}	
	
	\section*{Colle 3}
	\setcounter{exo}{0}
	MONTHILLER Thibaud (15): bien\\
	FERNANDEZ Henri (12): ne connaît pas bien la méthode\\
	
	\begin{exo}
		Limite d'une suite $u_{n+1} = f(u_n)$.
	\end{exo}	
\end{document}
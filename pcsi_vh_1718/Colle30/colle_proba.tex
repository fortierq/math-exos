\documentclass[10pt,a4paper]{article}
\usepackage{base}

\title{Colle PCSI 30: probabilités}

\newcounter{question}
\newcommand{\initQ}{\setcounter{question}{0}}
\newenvironment{question}{\addtocounter{question}{1}
	\noindent {\it {Question} \thequestion.\ }}
{\par}
\newcounter{exo}
\newcommand{\initE}{\setcounter{exo}{0}}
\newenvironment{exo}{\vspace{0.5cm}\setcounter{question}{0}\addtocounter{exo}{1} \noindent \textbf{Exercice \theexo}. \normalsize }{\par}

\begin{document}
	\maketitle
	
	\section*{Colle 1}
	MONNIN Guillaume (8): ne connaît pas les définitions/théorèmes du cours, propose des solutions un peu au pif à base de $n!$, $\binom{n}{k}$...\\
	TRONCIN (12): ne connait pas bien les définitions du cours. bien pour l'exo.\\
	
	\setcounter{exo}{0}
	
	\begin{exo}
		Formule probas totales
	\end{exo}
	\begin{exo}
		Proba que 2 élèves aient leur anniversaire le même jour?
	\end{exo}

	\section*{Colle 2}

	\setcounter{exo}{0}
	MONTHILLIER Thibaud (note: 13): propose des solutions un peu au pif à base de $n!$, $\binom{n}{k}$... assez bien sinon.\\
	Arthur (18): très bien.\\
	
	\begin{exo}
		Définition et propriété des probas conditionnelles.
	\end{exo}
		
	\begin{exo}
		Est-ce qu'un évènement peut etre indépendant de lui-même?
	\end{exo}	
	
	\begin{exo}
		On lance 2 dés. Est-ce que << un des dés tombe sur 1 >> et << la somme des 2 dés vaut 7 >> sont indépendant? Et si on remplace 1 par 6? 
	\end{exo}	
	
	\section*{Colle 3}
	\setcounter{exo}{0}
	Réjane(14): bien.\\
	
	\begin{exo}
		Formule de Bayes
	\end{exo}

	\begin{exo}
		Proba d'obtenir 2 as au poker?	
	\end{exo}
	
\end{document}
\documentclass[10pt,a4paper]{article}
\usepackage[utf8]{inputenc}
\usepackage{amsmath}
\usepackage{amsfonts}
\usepackage{amssymb}
\usepackage{graphicx}
\usepackage[french]{babel}
\title{Colle MP 12: Espaces préhilbertiens et séries entières}

\newcounter{question}
\newcommand{\initQ}{\setcounter{question}{0}}
\newenvironment{question}{\addtocounter{question}{1}
	\noindent {\it {Question} \thequestion.\ }}
{\par}
\newcounter{exo}
\newcommand{\Z}{\mathbb{Z}}

\newcommand{\initE}{\setcounter{exo}{0}}
\newenvironment{exo}{\vspace{0.5cm}\setcounter{question}{0}\addtocounter{exo}{1} \noindent \textbf{Exercice \theexo}. \normalsize }{\par}

\begin{document}
	\maketitle
	
	\section*{Colle 1}
	MAJOET Pierre (8/20): ne se souvient pas du tout du thm spectral (ni l'énoncé, ni la preuve). Un peu perdu sur l'exercice.\\
	Abdel (14/20): ne se souvient plus de la preuve de cours, mais bien pour l'exo. A de l'assurance.
	
	\begin{exo}
		Théorème spectral pour les endomorphisme symétriques
	\end{exo}
%
%	\begin{exo}
%		Soit $D_n$ le nombre de partitions de $\lbrace 1, 2, ..., n \rbrace$.
%		\begin{itemize}
%			\item Calculer $D_1$, $D_2$, $D_3$.
%			\item Mq $$D_{n+1} = \sum_{k=0}^{n} \binom{n}{k} D_k$$
%			\item Soit $S_n = \frac{1}{e}$
%		\end{itemize}
%	\end{exo}

	\begin{exo}
		Montrer que si $\alpha \in \mathbb{R}$, $f(t) =\cos(\alpha \arcsin(t))$ est développable en série entière sur un voisinage de 0 et trouver son rayon de convergence.\\
		Aide: montrer que $f$ vérifie l'ED $(1 - t^2) y'' - t y' + \alpha^2 y = 0$.
	\end{exo}
			
	\begin{exo}
		$O_n(\mathbb{R})$ est-il compact? connexe par arc?
	\end{exo}
	
	\section*{Colle 2}
	\setcounter{exo}{0}
	BRUGGER Martin (13/20): Ecrit $r \leq min(R_a, R_b)$ au lieu de $r \geq min(R_a, R_b)$.\\
	NADAL Julien (11/20): Très lent.\\
	
	\begin{exo}
		Somme et produit de Cauchy
	\end{exo}

	\begin{exo}
		Soit $a_0 = 1$ et $a_{n+1} = \sum_{k=0}^{n} a_k a_{n-k}$.
		\begin{itemize}
			\item Calculer $S(x) = \sum_{k=0}^{\infty} a_k x^k$.
			\item En déduire $a_n$, $\forall n$.
		\end{itemize}
	\end{exo}
	
	\begin{exo}
		\begin{itemize}
			\item Soit $A \in O_n(\mathbb{R})$. Mq:
			$$\vert \sum a_{i,j} \vert \leq n$$ 
			Indice: utiliser vecteur avec que des 1.
			\item Que connait tu comme produit scalaire sur les matrices? Quels sont ses propriétés?
			\item Soit $M \in O_n(\mathbb{R})$. Mq:
			$$\sum \vert m_{i,j} \vert \leq n \sqrt{n}$$ 
			Indice: utiliser $(A, B) \longmapsto tr({}^t A B)$.
		\end{itemize}
	\end{exo}
	
	\section*{Colle 3}
	\setcounter{exo}{0}
	Xavier (14/20): Bien\\
	Schwarz Thibaut (12/20): pas très rigoureux. \\
	
	\begin{exo}
		Lemme d'Abel.
	\end{exo}
	
	\begin{exo}
		Soit $u$ un endomorphisme symétrique positif. Montrer que $u$ a une unique racine carré.\\
		Aide pour l'unicité: montrer que si $u = v^2$ alors $v$ stabilise les espaces propres de $u$.
	\end{exo}

	\begin{exo}
		Calculer $\sum_{n=0}^{\infty} \frac{n}{3^n}$.
	\end{exo}

\end{document}
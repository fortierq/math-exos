\documentclass[10pt,a4paper]{article}
\usepackage[utf8]{inputenc}
\usepackage{amsmath}
\usepackage{amsfonts}
\usepackage{amssymb}
\usepackage{graphicx}
\usepackage[french]{babel}
\title{Colle MP 7: EVN + Séries}

\newcounter{question}
\newcommand{\initQ}{\setcounter{question}{0}}
\newenvironment{question}{\addtocounter{question}{1}
	\noindent {\it {Question} \thequestion.\ }}
{\par}
\newcounter{exo}
\newcommand{\Z}{\mathbb{Z}}

\newcommand{\initE}{\setcounter{exo}{0}}
\newenvironment{exo}{\vspace{0.5cm}\setcounter{question}{0}\addtocounter{exo}{1} \noindent \textbf{Exercice \theexo}. \normalsize }{\par}

\begin{document}
	\maketitle
	
	\section*{Colle 1}
	Océane TOPENOT (14): Démo de cours bien. Exo correct.\\
	CHOPARD Etienne (11): Démo de cours assez bien. Très hésitant sur les exos, ne connaît pas bien le cours.\\
	
	\begin{exo}
		f continue, A connexe par arc $\implies$ $f(A)$ connexe par arc
	\end{exo}

	\begin{exo}
		Equivalent de $\ln(n!)$?
	\end{exo}
	
	\begin{exo}
		Soit $E$ un ensemble fini de $\mathbb{R}^2$. Mq $\mathbb{R}^2 - E$ est connexe par arc.
	\end{exo}

	\section*{Colle 2}
	\setcounter{exo}{0}
	Héloise (12): Démo de cours bien. qques petites erreurs de raisonnement\\
	TEMIZYUREK Muhammed (13): Démo de cours assez bien, ainsi que l'exo.\\
	
	\begin{exo}
		fct linéaire continue en dim finie?
	\end{exo}

	\begin{exo}
		Convergence/Equivalent de $\sum \frac{1}{n\log(n)}$?
	\end{exo}
	
	\begin{exo}
		$GL_n(\mathbb{R})$ est-il connexe par arc?
		Est-ce qu'il existe une norme sur $M_n(\mathbb{C})$?\\
		Mq $GL_n(\mathbb{C})$ ouvert dense connexe par arc dans $M_n(\mathbb{C})$. 
	\end{exo}

%	\begin{exo}
%		Adhérence et intérieur des matrices diagonalisables de $\mathcal{M}_n(\mathcal{C})$.
%	\end{exo}
	
	\section*{Colle 3}
	\setcounter{exo}{0}
	Lily (11): Démo de cours assez bien mais ne connaît pas la démo pour trouver un équivalent de la série harmonique et n'est pas à l'aise avec les techniques standards.\\
	Armand (14): petit bug dans la démo de cours mais bien sinon (jolis dessins).\\
	
	\begin{exo}
		Série Riemann.
	\end{exo}

	\begin{exo}
		Montrer que l’union de deux connexes par arcs non disjoints est connexe par arcs.
	\end{exo}
	\begin{exo}
		Mq $\sum_{k=1}^{n} \vert \sin(k) \vert$ diverge (aide: montrer que $\sin(k)^2 = \frac{1 - cos(2k)}{2}$, puis borner $\sum cos(2k)$).
	\end{exo}

\end{document}
\documentclass[10pt,a4paper]{article}
\usepackage[utf8]{inputenc}
\usepackage{amsmath}
\usepackage{amsfonts}
\usepackage{amssymb}
\usepackage{graphicx}
\usepackage[french]{babel}

\title{Colle MP 17: Variables aléatoires + séries entières}

\newcounter{question}
\newcommand{\initQ}{\setcounter{question}{0}}
\newenvironment{question}{\addtocounter{question}{1}
	\noindent {\it {Question} \thequestion.\ }}
{\par}
\newcounter{exo}
\newcommand{\Z}{\mathbb{Z}}

\newcommand{\initE}{\setcounter{exo}{0}}
\newenvironment{exo}{\vspace{0.5cm}\setcounter{question}{0}\addtocounter{exo}{1} \noindent \textbf{Exercice \theexo}. \normalsize }{\par}

\begin{document}
	\maketitle
	
	\section*{Colle 1}
	Mélanie (14): écrit une inégalité entre complexes. Bien sinon.\\
	CECE Gaetan (14): ne se souvient pas bien du critère des séries alternées ni du DSE de $\ln(1+x)$. Bien sinon.

	\begin{exo}
		Cours
	\end{exo}

	\begin{exo}
		Quels sont les $z$ pour lesquels $\sum \frac{z^n}{n}$ converge?
	\end{exo}
	
	\begin{exo}
		\begin{enumerate}
			\item Mq $E(X) = \sum_{k=1}^\infty P(X \geq k)$
			\item Si $X$, $Y$ sont uniformes sur $\lbrace 1,..., n \rbrace$, quelle est l'espérance de $min(X, Y)$ et $max(X, Y)$?
		\end{enumerate}
	\end{exo}	
		
	\section*{Colle 2}
	\setcounter{exo}{0}
	Damien (10): pense que $\mathcal{P}$ et $\mathbb{P}$ désignent la même chose. Se trompe dans le DSE et le rayon de convergence de $\ln$. Écrit $\ln(-2)$.\\
	Léa (16): très bien.
	
	\begin{exo}
		Cours: Si X est une v.a.d., alors pour tout $A \in \mathcal{P}(X(\Omega))$, $X^{-1}(A) \in \mathcal{T}$
	\end{exo}
	\begin{exo}
		DSE et rayon de CV de $\ln(1 + x - 2x^2)$?
	\end{exo}
	
	\begin{exo}
		Une variable aléatoire X suit une loi binomiale de taille n et de paramètre p .
		Quelle est la loi suivie par la variable Y = n - X?
	\end{exo}

	\begin{exo}
		Soit X une variable aléatoire suivant une loi géométrique de paramètre p. Calculer $E(\frac{1}{X})$.
	\end{exo}
	
	\section*{Colle 3}
	\setcounter{exo}{0}
	CORNUEZ Charlotte (15): quelques erreurs de calcul. TB sinon.\\
	POISOT Lucas (13): correct mais assez lent dans les exercices

	\begin{exo}
		Cours
	\end{exo}

	\begin{exo}
		Calculer $\sum_{n=0}^{\infty} \frac{n}{3^n}$ (en utilisant une série entière dérivée).
	\end{exo}

	\begin{exo}
		Mq la somme de 2 variables de Poisson indépendantes est une variable de Poisson.
	\end{exo}
	
	\begin{exo}
		Soit X et Y deux variables aléatoires indépendantes suivant des lois de
		Bernoulli de paramètres p et q . Déterminer la loi de la variable
		Z = max(X, Y ) .
	\end{exo}
	
\end{document}
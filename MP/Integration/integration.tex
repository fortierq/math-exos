\documentclass[10pt,a4paper]{article}
\usepackage{base}
\title{Colle MP: Intégration et fonctions vectorielles}

\newcounter{question}
\newcommand{\initQ}{\setcounter{question}{0}}
\newenvironment{question}{\addtocounter{question}{1}
	\noindent {\it {Question} \thequestion.\ }}
{\par}
\newcounter{exo}

\newcommand{\initE}{\setcounter{exo}{0}}
\newenvironment{exo}{\vspace{0.5cm}\setcounter{question}{0}\addtocounter{exo}{1} \noindent \textbf{Exercice \theexo}. \normalsize }{\par}

\begin{document}
	\maketitle
	
	\section*{Colle 1}
	Mathilde (14): assez bien\\
	Eva (11): utilise une comparaison série-intégrale alors que la fonction n'est pas monotone. Pas mal d'imprécisions.

	\begin{exo}
		Étudier la convergence de $\int_{0}^{\infty} {\sin(x)}$, $\int_{0}^{\infty} \frac{\c{\sin(x)}}{x}$, $\int_{0}^{\infty} \frac{\sin(x)}{x}$.
	\end{exo}

%	\begin{exo}
%		Trouver la limite de:
%		$$\sum_{k=1}^{n} \sin(\frac{k}{n}) \sin(\frac{k}{n^2}) ~~(= \int t \sin(t))$$
%		Aide: $\sin(\frac{k}{n^2}) \approx \frac{k}{n^2}$.
%	\end{exo}
	
	\section*{Colle 2}
	\setcounter{exo}{0}
	Lilou (12): erreurs dans la formule de changement de variable. Assez lente.\\
	Théo (13): assez lent\\

	\begin{exo}
		Cours
	\end{exo}

	\begin{exo}
		$$\int_{0}^{\frac{\pi}{2}} \ln(\sin(x))dx?$$
	\end{exo}

	\begin{exo}
		Déterminer l'intégrabilité de certaines fonctions.
	\end{exo}

%	\begin{exo} (erreur dans la méthode des rectangles)
%		Soit $f$ $C^1$ sur $[a, b]$. Soit $a_k = a + k\frac{b-a}{n}$.\\ Mq il existe une constante $M$ tq:
%		$$\vert \int_{a}^{b} f - \frac{b-a}{n} \sum_{k=0}^{n-1} f(a_k) \vert \leq M \frac{(b-a)^2}{2n}$$
%	\end{exo}

	\section*{Colle 3}
	\setcounter{exo}{0}
	Gabriel (15): erreur dans la définition de la limite. Bonne prise d'initiative sinon.\\
	Mylène (14): bien.
	\begin{exo}
		Cours
	\end{exo}

	\begin{exo}
		Calculer, par récurrence, $\int_0^1 \ln(x)^n dx$
	\end{exo}

	\begin{exo}
		Soit $f$ continue intégrable sur $\R$. Est-ce que $f \longrightarrow 0$? Et si $f$ décroissante? Et $xf(x)$?
	\end{exo}

\end{document}
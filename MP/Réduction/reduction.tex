\documentclass[10pt,a4paper]{article}
\usepackage[utf8]{inputenc}
\usepackage{amsmath}
\usepackage{amsfonts}
\usepackage{amssymb}
\usepackage{graphicx}
\title{Colle MP 3: réduction}

\newcounter{question}
\newcommand{\initQ}{\setcounter{question}{0}}
\newenvironment{question}{\addtocounter{question}{1}
	\noindent {\it {Question} \thequestion.\ }}
{\par}
\newcounter{exo}
\newcommand{\Z}{\mathbb{Z}}

\newcommand{\initE}{\setcounter{exo}{0}}
\newenvironment{exo}{\vspace{0.5cm}\setcounter{question}{0}\addtocounter{exo}{1} \noindent \textbf{Exercice \theexo}. \normalsize }{\par}

\begin{document}
	\maketitle
	
	\section*{Colle 1}
	CAIREY-REMONNAY Solène (18): cours parfait. Exercice aussi.\\
	PIERRE Alexandre (16): cours très bien. Exercice aussi.\\
	
	\begin{exo}
		Tout endo en dim finie admet un poly annulateur. Contre-exemple en dim infinie?
	\end{exo}

	\begin{exo}
		Soient $u, v$ deux endo d'un EV $E$.
		\begin{enumerate}
			\item Si $\lambda \neq 0$ est v.p de $u \circ v$ mq $\lambda$ est aussi v.p de $v \circ u$.
			\item Soient $E =\mathbb{R}[X]$ et $u, v$ deux endo définis par $u(P) = P'$ et $v(P)$ = primitive de $P$ s'annulant en 0.
			Déterminer $Ker(u \circ v)$ et $Ker(v \circ u)$.
			\item Mq (1) reste vrai si $\lambda = 0$ en dimension finie.
		\end{enumerate}
	\end{exo}

	\section*{Colle 2}
	\setcounter{exo}{0}
	MARTIN Manon (11): ne se souvient pas bien de la démo. Confusion entre endomorphisme/vecteur/réel/polynome.\\
	Colin CHAISE (15): bien pour la démo de cours. Erreur dans le thm du rang.\\
	
	\begin{exo}
		Théorème des noyaux.\\
		A t-on besoin qu'ils soient premiers entre eux.
	\end{exo}

	\begin{exo}
		Soit $E$ $\mathbb{K}$-EV de dimension finie, $u \in \mathcal{L}(E)$ et $P$ annulateur de $u$. Supposons que $P = QR$ avec $Q$, $R$ premiers entre eux. Mq $Im Q(u) = Ker R(u)$.
	\end{exo}
%	\begin{exo}
%		Trouver $u_n$ et $v_n$ tq:
%		$$u_{n+1} = u_n - v_n$$
%		$$v_{n+1} = 2u_n + 4 v_n$$
%		$$u_0 = 2$$
%		$$v_0 = 1$$
%	\end{exo}

	\section*{Colle 3}
	\setcounter{exo}{0}
	JACQUOT Juliette (10): ne se souvient pas bien de la démo, ni des méthodes pour trouver des valeurs propres.\\
	GAUBERT Baptiste (10): ne se souvient pas bien de la démo. Confusion entre endomorphisme/vecteur/réel/polynome.
	
	\begin{exo}
		Spectre et racines d'un polynôme annulateur, du polynôme minimal
	\end{exo}
	
	\begin{exo}
		Valeurs propres de $\begin{pmatrix}
		0 & ... & 0 & 1 \\ 
		&  &  & ... \\ 
		0 &  & 0 & 1 \\ 
		1 & ... &  & 1
		\end{pmatrix} $ ?
	\end{exo}
	
	\begin{exo}
		Soit $E$ $\mathbb{K}$-EV de dim $n$ et $p$, $q$ deux projecteurs de $E$ qui commutent.
		Mq:
		\begin{enumerate}
			\item $p \circ q$ est un projecteur.
			\item $Im(p \circ q) = Im(p) \cap Im(q)$
			\item $Ker(p \circ q) = Ker(p) + Ker(q)$
		\end{enumerate}
	\end{exo}
	 
\end{document}
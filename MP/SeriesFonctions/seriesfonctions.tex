\documentclass[10pt,a4paper]{article}
\usepackage[utf8]{inputenc}
\usepackage{amsmath}
\usepackage{amsfonts}
\usepackage{lmodern,textcomp}
\usepackage{amssymb}
\usepackage{graphicx}
\usepackage[french]{babel}
\title{Colle MP: Séries et suites de fonctions}

\newcounter{question}
\newcommand{\initQ}{\setcounter{question}{0}}
\newenvironment{question}{\addtocounter{question}{1}
	\noindent {\it {Question} \thequestion.\ }}
{\par}
\newcounter{exo}
\newcommand{\Z}{\mathbb{Z}}

\newcommand{\initE}{\setcounter{exo}{0}}
\newenvironment{exo}{\vspace{0.5cm}\setcounter{question}{0}\addtocounter{exo}{1} \noindent \textbf{Exercice \theexo}. \normalsize }{\par}

\begin{document}
	\maketitle
	
	\section*{Colle 1}
	Colin (14): bien\\
	Solène (17): Très bien\\
	
	\begin{exo}
		Comparaison modes convergences.
	\end{exo}

	\begin{exo}
		Convergence simple et uniforme de $f_n(x) = \sqrt{x^2 + \frac{1}{n}}$?\\
		Calculer $\lim \int_{-1}^1 \sqrt{x^2 + \frac{1}{n}} dx$.
	\end{exo}

	\begin{exo}
		Soit $f:[a,b] \longrightarrow \mathbb{R}$ une fonction continue. On suppose que, pour tout $k\geq0$, on a $\int_a^b f(t)t^kdt=0$.\\
		Démontrer que $\int_a^bf(t)^2dt=0$.\\
		En déduire que f est la fonction nulle.		
	\end{exo}
	
	\section*{Colle 2}
	\setcounter{exo}{0}
	\noindent PIERRE Alexandre (14): petites confusions de quantificateurs, dans la démonstration \\
	Manon (12): des difficultés pour trouver un contre-exemple (pourtant assez simple)\\
	
	\begin{exo}
		Conservation continuité. Contre exemple si la CV n'est pas uniforme?
	\end{exo}
	
	\begin{exo}
		Montrer que $$\int_{0}^{1/2}\frac{1}{1-x} dx = \sum_{k=1}^{\infty} \frac{1}{k 2^k}$$
		En déduire $\sum_{k=1}^{\infty} \frac{1}{k 2^k}$.
	\end{exo}
	
	\begin{exo}
		Soit $f_n(x) = n^2 x (1 - nx)$ si $x \in [0,1/n]$ et $f_n(x)=0$ sinon.\\
		Étudier la limite simple de la suite $(f_n)$.\\
		Calculer $\int_0^1f_n(t)dt$. Y-a-t-il convergence uniforme sur [0,1]?
		Étudier la convergence uniforme sur $[a,1]$ pour $a \in ]0,1]$.
	\end{exo}
	
	\section*{Colle 3}
	\setcounter{exo}{0}
	GAUBERT Baptiste (13): erreurs de calcul\\
	Juliette (13): erreurs de calcul\\
	
	\begin{exo}
		Théorème : limite d'une suite d'intégrales. Contre-exemple si la convergence n'est pas uniforme?
	\end{exo}
	
	\begin{exo}
		Convergence simple et uniforme sur $\mathbb{R}$ de $fn(x) = \frac{nx}{1+n^2 x^2}$
	\end{exo}

\end{document}
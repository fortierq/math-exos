\documentclass[10pt,a4paper]{article}
\usepackage{base}
\title{Colle MP 12: endomorphismes symétriques}

\newcounter{question}
\newcommand{\initQ}{\setcounter{question}{0}}
\newenvironment{question}{\addtocounter{question}{1}
	\noindent {\it {Question} \thequestion.\ }}
{\par}
\newcounter{exo}

\newcommand{\initE}{\setcounter{exo}{0}}
\newenvironment{exo}{\vspace{0.5cm}\setcounter{question}{0}\addtocounter{exo}{1} \noindent \textbf{Exercice \theexo}. \normalsize }{\par}

\begin{document}
	\maketitle
	
	\section*{Colle 1}
	Rauch Julien (12): manque de clarté, beaucoup de confusions. oubli de $tr(AB) = tr(BA)$\\
	Arnaud (9): erreur dans la définition d'une matrice associée à un endomorphisme. a complètement oublié comment l'étude pratique d'endomorphisme orthogonaux en dimension 3.\\
	Stepan (14): oubli de $tr(AB) = tr(BA)$\\
	
	\begin{exo}
		équivalence matrice symétrique et endo symétrique
	\end{exo}

	\begin{exo}
		Soit u un endomorphisme symétrique d'un espace euclidien E vérifiant, pour tout $x\in E$, $⟨u(x),x⟩=0$. Mq $u = 0$.		
	\end{exo}

%	\begin{exo}
%		Soit $E = C ([−1, 1] , R)$ muni du produit scalaire $\int_{-1}^{1} fg$.\\
%		Quel est l'orthogonal des fonctions nulles sur $[-1, 0]$? Sont-ils supplémentaires?
%	\end{exo}
		
	\section*{Colle 2}
	\setcounter{exo}{0}
	ARRIGONI Valentin (14): Assez bien.\\
	Tom (16): Bien\\
	
	\begin{exo}
		Stabilité pour les endo orthogonaux.
	\end{exo}

	\begin{exo}
		Soit $A \in M_n(\mathbb{R})$. Démontrer que la matrice $^tAA$ est diagonalisable et que ses valeurs propres sont des réels positifs.
	\end{exo}
	
	\begin{exo}
		(57) Mq $S_n$ et $A_n$ sont supplémentaire orthogonaux pour le prod canonique. Distance d'une matrice ... à $S_3$?
	\end{exo}
		
	\begin{exo}
		(39) Soit $f$ endo d'un espace euclidien tq $(f(x) \vert x) = 0$, $\forall x$. Mq $Ker f = Im f ^\perp$.
	\end{exo}
	
\section*{Colle 3}
	\setcounter{exo}{0}
	Achille (15): oubli d'une petite partie de la démo\\
	Lily (16): Bien.\\
	
	\begin{exo}
		1ers lemme du thm spectral
	\end{exo}

	\begin{exo}
		Caractériser la nature géométrique d'une matrice orthogonale.
	\end{exo}
	
	\begin{exo}
		Mq $O_n(\mathbb{R})$ est un compact non connexe par arc. Quelles sont ses composantes connexes par arcs?
	\end{exo}
	
	\begin{exo}
		Soit $A \in O_n(\mathbb{R})$. Mq:
		$$\vert \sum a_{i,j} \vert \leq n$$ 
		Indice: utiliser vecteur avec que des 1.
	\end{exo}
	
	\begin{exo}
		Soit $M \in O_n(\mathbb{R})$. Mq:
		$$\sum \vert m_{i,j} \vert \leq n \sqrt{n}$$ 
		Indice: utiliser $(A, B) \longmapsto tr({}^t A B)$.
	\end{exo}

\end{document}
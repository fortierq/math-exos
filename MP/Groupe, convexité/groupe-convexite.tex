\documentclass[10pt,a4paper]{article}
\usepackage[latin1]{inputenc}
\usepackage{amsmath}
\usepackage{amsfonts}
\usepackage{amssymb}
\usepackage{graphicx}
\title{Colle MP : Structures, convexit�}

\newcounter{question}
\newcommand{\initQ}{\setcounter{question}{0}}
\newenvironment{question}{\addtocounter{question}{1}
	\noindent {\it {Question} \thequestion.\ }}
{\par}
\newcounter{exo}

\newcommand{\initE}{\setcounter{exo}{0}}
\newenvironment{exo}{\vspace{0.5cm}\setcounter{question}{0}\addtocounter{exo}{1} \noindent \textbf{Exercice \theexo}. \normalsize }{\par}

\begin{document}
	\maketitle
	
	\section*{Colle 1}
	JEUFFRAULT K�vin (14): bien pour le cours. Assez bien pour l'exo.\\
	CHOPARD Etienne (15): bien.\\
	_entropy
	\begin{exo}
			Sous-groupes de Z
	\end{exo}

	\begin{exo}
		Soit $x_1$, ..., $x_n$ tq $\sum x_i = K$. Valeur minimum de $\sum x_i^2$?
	\end{exo}
	
	\begin{exo}
		Donner une preuve ou un contre-exemple:\\
		\begin{itemize}
			\item La r�union de deux parties convexes est convexe.
			\item L'intersection de deux parties convexes est convexe.
			\item Une fonction convexe sur un intervalle ouvert est continue.
			\item Une fonction convexe sur un intervalle ouvert est d�rivable.
		\end{itemize}
	\end{exo}
	
	\section*{Colle 2}
	\setcounter{exo}{0}
	Nathan DURY (12): ne sait pas bien interpr�ter la convexit� sur un dessin. TB pour l'exo.\\
	R�da (12): un peu lent\\
	
	\begin{exo}
			D�finition et diverses in�galit�s de convexit�.\\
			Propri�t� des cordes
	\end{exo}
	
	\begin{exo}
		D�montrer que, pour tout $x \in ]0, \frac{\pi}{2}]$, $\frac{2}{\pi} \leq \frac{\sin(x)}{x} \leq 1$.
	\end{exo}
	
	\begin{exo}
		Soit $G$ groupe tq $x^2 = e$ $\forall x \in G$. Mq $G$ est ab�lien.
	\end{exo}

	\begin{exo}
		Soit G un groupe d'ordre pair. Montrer qu'il existe un �l�ment $x \in G$, $x \neq e$ tel que $x^2 = e$.
	\end{exo}
	
	\section*{Colle 3}
	\setcounter{exo}{0}
	AZIZI Marouane (12): bien pour le cours mais manque important de rigueur/clart�.\\
	Julien (13): un peu lent.\\
	
	\begin{exo}
		Id�aux de K[X] 
	\end{exo}

	\begin{exo}
		Exo option info: mq $x \longmapsto x \log_2(x)$ est convexe puis en d�duire par r�currence que la somme des profondeurs des $n$ feuilles d'un arbre binaire est au moins $n \log_2(n)$.
	\end{exo}
	\begin{exo}
		Soit $G$ un groupe ab�lien et $x, y \in G$ d'ordres $p, q$. \\
		\begin{question}
			Si $p$ et $q$ sont premiers entre eux, montrer que l'ordre de $xy$ est le ppcm de $p$ et $q$.
		\end{question}
		\begin{question}
			Est-ce toujours vrai si $p$ et $q$ ne sont pas premiers entre eux?
		\end{question}
	\end{exo}
		
	 
\end{document}
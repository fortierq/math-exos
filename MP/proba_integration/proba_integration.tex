\documentclass[10pt,a4paper]{article}
\usepackage[utf8]{inputenc}
\usepackage{amsmath}
\usepackage{amsfonts}
\usepackage{amssymb}
\usepackage{graphicx}
\usepackage[french]{babel}
\title{Colle MP 17: Variables aléatoires et intégration}

\newcounter{question}
\newcommand{\initQ}{\setcounter{question}{0}}
\newenvironment{question}{\addtocounter{question}{1}
	\noindent {\it {Question} \thequestion.\ }}
{\par}
\newcounter{exo}
\newcommand{\Z}{\mathbb{Z}}

\newcommand{\initE}{\setcounter{exo}{0}}
\newenvironment{exo}{\vspace{0.5cm}\setcounter{question}{0}\addtocounter{exo}{1} \noindent \textbf{Exercice \theexo}. \normalsize }{\par}

\begin{document}
	\maketitle
	
	\section*{Colle 1}	
	GUYOT Jeanne (13/20): assez bien\\
	BEGEY Marion (13/20): reste bloquée sur des choses "faciles"\\
	
	\begin{exo}
		dérivabilité sous le signe intégrale\\
		intégration terme à terme d’une série de fonctions
	\end{exo}

	\begin{exo}
		Fonctions génératrices des lois usuelles?
	\end{exo}
	
	\begin{exo}
		Soit $f(x) = \int_{0}^{x} e^{-t^2}$ et $g(x) = \int_{0}^{1} \frac{e^{-x^2 (1+t^2)}}{1+t^2}$.\\
		Mq $g$ est dérivable et $g'(x) = -2f'(x) f(x)$.\\
		En déduire $\lim_\infty f(x)$.
	\end{exo}
		
	\section*{Colle 2}
	\setcounter{exo}{0}
	MAULET Louis (14/20): majore $f_n$ au lieu de $f$ par $g$. petites erreurs de calcul. Bien sinon.\\
	ZOUGGARI Raphaël (12/20): majore $f_n$ au lieu de $f$ par $g$. se trompe dans l'aire d'un triangle.\\
	
	\begin{exo}
		intégration terme à terme d’une série de fonctions\\
		théorème de convergence dominée
	\end{exo}

%	\begin{exo}
%		Une famille veut faire des enfants jusqu'à avoir un garçon. Combien auront-ils de
%	\end{exo}

	\begin{exo}
		Somme de $n$ variables géométrique indépendantes: espérance, variance, fonction génératrice? 
	\end{exo}	
	
	\begin{exo}
		$\lim_\infty \int_{0}^{n} (1 - \frac{x^2}{n^2})^{n^2}$ = $\frac{\sqrt{\pi}}{2}$?
	\end{exo}
		
	\section*{Colle 3}
	\setcounter{exo}{0}
	PONS Ariane (12/20): assez bien\\
	Mehdy (16/20): Bien.\\
	
	\begin{exo}
		théorème de convergence dominée\\
		continuité sous le signe intégrale
	\end{exo}

	\begin{exo}
		Limite puis équivalent de $\int_{1}^{\infty} e^{-x^n}$? (poser $t = x^n$)
	\end{exo}

	\begin{exo}
		$\lim \int_{0}^{\infty} e^{-t} \sin^n(t) dt$?
	\end{exo}

\end{document}
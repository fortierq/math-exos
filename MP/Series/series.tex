\documentclass[10pt,a4paper]{article}
\usepackage[utf8]{inputenc}
\usepackage{amsmath}
\usepackage{amsfonts}
\usepackage{amssymb}
\usepackage{graphicx}
\usepackage[french]{babel}
\title{Colle MP 8: Séries}

\newcounter{question}
\newcommand{\initQ}{\setcounter{question}{0}}
\newenvironment{question}{\addtocounter{question}{1}
	\noindent {\it {Question} \thequestion.\ }}
{\par}
\newcounter{exo}
\newcommand{\Z}{\mathbb{Z}}

\newcommand{\initE}{\setcounter{exo}{0}}
\newenvironment{exo}{\vspace{0.5cm}\setcounter{question}{0}\addtocounter{exo}{1} \noindent \textbf{Exercice \theexo}. \normalsize }{\par}

\begin{document}
	\maketitle
	
	\section*{Colle 1}
	MAUBLANC Paul (14): assez bien\\
	Lise (15): bien\\
	
	\begin{exo}
		Critère de D'Alembert.
	\end{exo}
	
	\begin{exo}
		Convergence puis équivalent de $\sum_{2}^{\infty} \frac{1}{k\ln(k)}$? 
	\end{exo}

	\begin{exo}
		Calcul de $\sum_{2}^{\infty} \ln(1 - \frac{1}{n^2})$? 
	\end{exo}
%	\begin{exo}
%		Convergence de $\sum \frac{1}{{p_n}^{p_n}}$, où $p_n$ = nb de chiffres dans l'écriture décimale de $n$?
%	\end{exo}
	
	\section*{Colle 2}
	\setcounter{exo}{0}
	Rémi (12): Ne connaît pas bien la preuve. Ok pour l'exo.\\
	Eloïse (15): Bien. \\
 	
	\begin{exo}
		Critère spécial des séries alternées.\\
		Montrer que chaque condition est nécessaire. (considérer $u_n = \frac{1}{n}$ si $n$ pair, $\frac{1}{n^2}$ sinon)
	\end{exo}

	\begin{exo}
		Convergence puis calcul de $\sum_{2}^{\infty} \frac{1}{k(k+1)}$? 
	\end{exo}
	
	\begin{exo}
		Convergence puis calcul de $\sum_{k=1}^{\infty} \frac{k}{2^k}$?
	\end{exo}
%	\begin{exo}
%		Adhérence et intérieur des matrices diagonalisables de $\mathcal{M}_n(\mathcal{C})$.
%	\end{exo}
	
	\section*{Colle 3}
	\setcounter{exo}{0}
	ANDRIEU Grégoire (8): ne connaît pas du tout la preuve. Dit que $A^k = (a_{i,j}^k)$.\\
	Julie (14): Assez bien.\\
	
	\begin{exo}
		Exponentielle de matrice.
	\end{exo}
	
	\begin{exo}
		Equivalent de $\ln(n!)$ quand $n \longrightarrow \infty$?
	\end{exo}

	\begin{exo}
		\begin{itemize}
			\item Calculer $\sum_{k=0}^{n} sin(k)$.
			\item En déduire que $\sum_{k=0}^{n} \vert \sin(k) \vert \rightarrow \infty$ et que $\sum \frac{\sin^2(k)}{k}$ converge.
			\item (Abel) Mq $\sum \frac{sin(k \theta)}{k}$ converge.
		\end{itemize}			
	\end{exo}

\end{document}
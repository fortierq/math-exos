\documentclass[10pt,a4paper]{article}
\usepackage[utf8]{inputenc}
\usepackage{amsmath}
\usepackage{amsfonts}
\usepackage{amssymb}
\usepackage{graphicx}
\title{Colle MP 3: réduction}

\newcounter{question}
\newcommand{\initQ}{\setcounter{question}{0}}
\newenvironment{question}{\addtocounter{question}{1}
	\noindent {\it {Question} \thequestion.\ }}
{\par}
\newcounter{exo}
\newcommand{\Z}{\mathbb{Z}}

\newcommand{\initE}{\setcounter{exo}{0}}
\newenvironment{exo}{\vspace{0.5cm}\setcounter{question}{0}\addtocounter{exo}{1} \noindent \textbf{Exercice \theexo}. \normalsize }{\par}

\begin{document}
	\maketitle
	
	\section*{Colle 1}
	MOUROT Jeanne (8): très faible. Ne connaît pas la définition de valeur propre, confond scalaire et vecteur.\\
	CORDIER Pauline (12): confusion sur le polynome caractéristique ($\det(\chi_u)$...). Un peu lente.\\
	
	\begin{exo}
		Question de cours.
	\end{exo}
	
	\begin{exo}
		Trouver les suites vérifiant:
		$$u_{n+1} = u_n - v_n$$
		$$v_{n+1} = 2u_n + 4v_n$$
	\end{exo}

	\begin{exo}
		Soient $u, v$ deux endo d'un EV $E$.
		\begin{enumerate}
			\item Si $\lambda \neq 0$ est v.p de $u \circ v$ mq $\lambda$ est aussi v.p de $v \circ u$.
			\item Mq (1) reste vrai si $\lambda = 0$ en dimension finie.
			\item Soient $E =\mathbb{R}[X]$ et $u, v$ deux endo définis par $u(P) = P'$ et $v(P)$ = primitive de $P$ s'annulant en 0.
			Déterminer $Ker(u \circ v)$ et $Ker(v \circ u)$.
		\end{enumerate}
	\end{exo}
	
	
	\section*{Colle 2}
	\setcounter{exo}{0}
	Ysaline (13): assez bien. \\
	GUICHON Joannes (14): démonstration incomplète. Bien sur les exos.\\
	
	\begin{exo}
		Question de cours.
	\end{exo}

	\begin{exo} (268)
		Soit $A \in GL_n(\mathbb{C})$ et $p \in N^*$. Mq $A$ diagonalisable $\Longleftrightarrow$ $A^p$ diagonalisable.
	\end{exo}
	
	\begin{exo}
		\begin{enumerate}
			\item $$J = \begin{pmatrix}
			0 & 1 & 0  & ... \\ 
			0 & 0 & 1 & ...\\ 
			... & ... & ...  & 1\\
			1 & 0 & ...  & 0\\
			\end{pmatrix}$$ Calculer $J^2, ..., J^n$ puis montrer que $J$ est diagonalisable, quelle sont ses v.p?
			\item En déduire la valeur de:
			$$\Delta_n = \begin{vmatrix}
			a_1 & a_2 & ... & a_n \\ 
			a_n & a_1 & ... & a_{n-1} \\ 
			... & ... & ... & ...\\
			a_2 & a_3 & ... & a_1\\
			\end{vmatrix}$$
		\end{enumerate}
	\end{exo}

	\section*{Colle 3}
	\setcounter{exo}{0}
	Laura (12): lente pour diagonaliser une matrice.\\
	Layla (17): Très bien.\\
	Théophane (13): assez bien mais un peu brouillon.\\
	
	\begin{exo}
		Question de cours.
	\end{exo}

	\begin{exo}
		Trouver le polynome minimal de:
		$A = \begin{pmatrix}
		-1 & 1 & 1 \\ 
		1 & -1 & 1 \\ 
		1 & 1 & -1
		\end{pmatrix}$
	\end{exo}
		
	\begin{exo}
		Trouver toutes les suites $u_n$ telles que:
		$$\forall n, u_{n+3} = 2 u_{n+2} + u_{n+1} - 2u_n$$
	\end{exo}
	 
\end{document}
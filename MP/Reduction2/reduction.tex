\documentclass[10pt,a4paper]{article}
\usepackage[utf8]{inputenc}
\usepackage{amsmath}
\usepackage{amsfonts}
\usepackage{amssymb}
\usepackage{graphicx}
\title{Colle MP 4: réduction 2}

\newcounter{question}
\newcommand{\initQ}{\setcounter{question}{0}}
\newenvironment{question}{\addtocounter{question}{1}
	\noindent {\it {Question} \thequestion.\ }}
{\par}
\newcounter{exo}
\newcommand{\Z}{\mathbb{Z}}

\newcommand{\initE}{\setcounter{exo}{0}}
\newenvironment{exo}{\vspace{0.5cm}\setcounter{question}{0}\addtocounter{exo}{1} \noindent \textbf{Exercice \theexo}. \normalsize }{\par}

\begin{document}
	\maketitle
	
	\section*{Colle 1}
	JAUAD Reda (note: 10): preuve de cours mal expliqué et peu rigoureuse\\
	Borgeois Slava (note: 14): Très bien sur le cours mais exo décevant.\\
	GUILLEMAUD Tom (note: 13): petits oublis dans la preuve\\
	Abdel (note: 16): bien mais confond vecteur et endomorphisme dans les espace de fonctions\\
	
	\begin{exo}
		Démo cours
	\end{exo}

	\begin{exo}
		Soient $u, v$ deux endo d'un EV $E$.
		\begin{enumerate}
			\item Si $\lambda \neq 0$ est v.p de $u \circ v$ mq $\lambda$ est aussi v.p de $v \circ u$.
			\item Mq (1) reste vrai si $\lambda = 0$ en dimension finie.
			\item Soient $E =\mathbb{R}[X]$ et $u, v$ deux endo définis par $u(P) = P'$ et $v(P)$ = primitive de $P$ s'annulant en 0.
			Déterminer $Ker(u \circ v)$ et $Ker(v \circ u)$.
		\end{enumerate}
	\end{exo}
	


	\begin{exo}
		Montrer $\chi_{AB} = \chi_{BA}$ pour $A \in GL$ puis $A$ quelconque.\\
		Indice: mq pour $p$ assez grand, $A + \frac{1}{p} I_n$ $\in$ $GL_n$.  
	\end{exo}
	
	\section*{Colle 2}
	\setcounter{exo}{0}
	Nimo Champion (note: 10): preuve de cours mal expliqué et peu rigoureuse\\
	SCHMIDT Arthur (note: 14): bien sur le cours, exo correct.\\
	DAUDEY Clément (note: 14): Bien sauf erreur de calcul\\
	LEMEDEF Stepan (note: 14): assez bien sauf quand il dit que $\det(A+B) = \det(A) + \det(B)$.\\
	
	\begin{exo}
		Démo cours
	\end{exo}
	
	\begin{exo}
		Diagonaliser 
		$$\begin{pmatrix}
		1 & -1 \\ 
		2 & 4
		\end{pmatrix}$$ 
		puis résoudre: 
		$$x' = x - y$$
		$$y' = 2x + 4y$$
	\end{exo}
	
	\begin{exo}
		Soit $E$ $\mathbb{K}$-EV de dim $n$ et $p$, $q$ deux projecteurs de $E$ qui commutent.
		Mq:
		\begin{enumerate}
			\item $p \circ q$ est un projecteur.
			\item $Im(p \circ q) = Im(p) \cap Im(q)$
			\item $Ker(p \circ q) = Ker(p) + Ker(q)$
		\end{enumerate}
	\end{exo}

	\section*{Colle 3}
	\setcounter{exo}{0}
	Prost Vincent (note: 13): preuve cours à peu près ok mais peu rigoureuse\\
	Elsa Maillot (note: 13): bien sur le cours mais bof sur les exos.\\
	Bonnetain Baptiste (13): petits manques de rigueurs (ensemble = entier...)\\
	AMRANE Paul (15): bien, mais ne se souvient pas bien du thm de décomposition en projecteurs\\
	NADAL Julien (12): confond Sp et Rac. Pas bonne colle pour un 5/2\\
	
	\begin{exo}
		Démo cours
	\end{exo}
	
	\begin{exo}
		Polynome minimal d'un projecteur?
	\end{exo}

	\begin{exo}
		\begin{enumerate}
			\item $$J = \begin{pmatrix}
			0 & 1 & 0  & ... \\ 
			0 & 0 & 1 & ...\\ 
			... & ... & ...  & 1\\
			1 & 0 & ...  & 0\\
			\end{pmatrix}$$ Calculer $J^2, ..., J^n$ puis montrer que $J$ est diagonalisable, quelle sont ses v.p?
			\item En déduire la valeur de:
			$$\Delta_n = \begin{vmatrix}
			a_1 & a_2 & ... & a_n \\ 
			a_n & a_1 & ... & a_{n-1} \\ 
			... & ... & ... & ...\\
			a_2 & a_3 & ... & a_1\\
			\end{vmatrix}$$
		\end{enumerate}
	\end{exo}

	\begin{exo}
		Soit $M \in M_2(\mathbb{R})$ tq $M^2 + M = U_2$. Quelles sont les vp possibles de $M$?
	\end{exo}
		 
\end{document}
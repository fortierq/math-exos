\documentclass[10pt,a4paper]{article}
\usepackage[utf8]{inputenc}
\usepackage{amsmath}
\usepackage{amsfonts}
\usepackage{amssymb}
\usepackage{graphicx}
\usepackage{lmodern,textcomp}
\title{Colle PCSI 8: Sommes et primitives}

\newcounter{question}
\newcommand{\initQ}{\setcounter{question}{0}}
\newenvironment{question}{\addtocounter{question}{1}
	\noindent {\it {Question} \thequestion.\ }}
{\par}
\newcounter{exo}
\newcommand{\Z}{\mathbb{Z}}

\newcommand{\initE}{\setcounter{exo}{0}}
\newenvironment{exo}{\vspace{0.5cm}\setcounter{question}{0}\addtocounter{exo}{1} \noindent \textbf{Exercice \theexo}. \normalsize }{\par}

\begin{document}
	\maketitle

	\section*{Colle 1}
	MONNIN Guillaume (9): ne se souvient pas de la formule de Bernouilli. ne sait pas trouver une primitive de $\sin^4$\\
	BAYARD (8): ne connaît pas la formule. écrit qu'une primitve de $\sin^4$ est $\frac{\cos^3(x)}{4}$. Ne connaît pas la somme des termes d'une suite géométrique. écrit somme des produits = produit somme
	\begin{exo}
		Formule de Bernouilli.
	\end{exo}
	
	\begin{exo}
		Primitive de $\sin^4$?
	\end{exo}

	\begin{exo}
		$\sum_{k=1}^{n} \frac{1}{k(k+1)}$?
	\end{exo}
	\begin{exo}
		Calculer $\sum_{k=1}^{n} cos(k x)$, $\sum \binom{n}{k} \cos(kx)$, $\sum_{k=1}^{n} k cos(k x)$.
	\end{exo}	
	
	\section*{Colle 2}
	\setcounter{exo}{0}
	MONTHILLER Thibaud (12): pas mal de petites erreurs, assez lent.\\
	BISE Clara (12): manque d'initiatives\\
	
	\begin{exo}
		méthode pour trouver primitive de 1/($ax^2 + bx + c$)?
	\end{exo}
	
	\begin{exo}
		Calculer: $\sum x^{k}$ puis $\sum k 2^{k-1}$. 
	\end{exo}

	\begin{exo}
		Calculer: 
		$$\sum k k!$$ 
	\end{exo}	
	\section*{Colle 3}
	\setcounter{exo}{0}
	Hugo MOTTET (15): oubli du 1er terme dans la formule du nibome\\
	BARBAROUX Elisa (13): correct\\
	
	\begin{exo}
		Formule du binome de Newton?
	\end{exo}
	\begin{exo}
		Calculer ($\sum (k+1)^3 - k^3$ puis) $\sum k^2$.
	\end{exo}

	\begin{exo}
		Primitive de $\frac{1}{3x^2+6x+3}$?
	\end{exo}
	
	\begin{exo}
		Calculer $\sum_{k\text{ pair}} \binom{n}{k}$.
	\end{exo}
\end{document}
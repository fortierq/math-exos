\documentclass[10pt,a4paper]{article}
\usepackage[utf8]{inputenc}
\usepackage{base}
\title{Colle PCSI 20: continuité}

\newcounter{question}
\newcommand{\initQ}{\setcounter{question}{0}}
\newenvironment{question}{\addtocounter{question}{1}
	\noindent {\it {Question} \thequestion.\ }}
{\par}
\newcounter{exo}

\newcommand{\initE}{\setcounter{exo}{0}}
\newenvironment{exo}{\vspace{0.5cm}\setcounter{question}{0}\addtocounter{exo}{1} \noindent \textbf{Exercice \theexo}. \normalsize }{\par}

\begin{document}
	\maketitle
\section*{Colle 1}
\setcounter{exo}{0}
	PRIORESCHI (note: 14): bien pour la question cours.\\
	BINET Mathilde (note: 12): ne sait pas bien utiliser les définitions\\
		
	\begin{exo}
		Montrer que si $f : \R \longrightarrow \R$ continue a une limite 1 en $\infty$ et -1 en $-\infty$ alors $f$ s'annule.
 	\end{exo}

	\begin{exo}
		$f$ continue décroissante sur $\R$ $\implies$ $f$ a un point fixe
	\end{exo}
	
	\section*{Colle 2}
	DERET Simon (note: 14): rapide sur le premier exo, mais manque de précision.\\
	BASTIEN Cléo (note: 13): manque de précisions.\\
	
	\begin{exo}
		$f : [0, 1] \rightarrow [0, 1]$ continue $\Longrightarrow$ $f$ a un point fixe
	\end{exo}

	\section*{Colle 3}
	\setcounter{exo}{0}
	François Léonard (13): bonnes idées mais reste très vague, ne définit pas ses variables.\\
	BONNOT Alex (13): manque d'assurance mais assez bien.\\
	
	\begin{exo}
		Si $f \longrightarrow \infty$ en $-/+ \infty$ alors la dérivée de $f$ s'annule.
	\end{exo}
	
\end{document}
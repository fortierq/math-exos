\documentclass[10pt,a4paper]{article}
\usepackage[utf8]{inputenc}
\usepackage{base}
\title{Colle PCSI 16: matrices}

\newcounter{question}
\newcommand{\initQ}{\setcounter{question}{0}}
\newenvironment{question}{\addtocounter{question}{1}
	\noindent {\it {Question} \thequestion.\ }}
{\par}
\newcounter{exo}

\newcommand{\initE}{\setcounter{exo}{0}}
\newenvironment{exo}{\vspace{0.5cm}\setcounter{question}{0}\addtocounter{exo}{1} \noindent \textbf{Exercice \theexo}. \normalsize }{\par}

\begin{document}
	\maketitle
\section*{Colle 1}
\setcounter{exo}{0}
	Youri (10): Bien pour l'application du pivot de Gauss. ne connaît pas la définition d'une transposée.\\
	Hugo M (16): petits soucis sur les indices d'une matrice. Bien sinon
	
	\begin{exo}
		Formules que tu connais avec la transposée?
	\end{exo}
	
	\begin{exo}
		\begin{itemize}
			\item Montrer que le produit et la somme de deux matrices nilpotentes qui commutent sont nilpotentes. Contre exemple si les matrices ne commutent pas?
			\item Mq si $A$ et $B$ commutent alors $A^n - B^n = (A - B)(A^{n-1} + A^{n-2}B + ... + B^{n-1})$.
			\item Soit $M \in \mathcal{M}_{n, n}$ nilpotente: $M^p = 0$, $p \in \mathbb{N}^*$. Montrer que $I_n$ - $M$ est inversible et déterminer son inverse.
		\end{itemize}
	\end{exo}

	\section*{Colle 2}
	Myriam (16): Bien pour l'application du pivot de Gauss.\\
	Adrien LEROY (16): Bien\\
	\begin{exo}
		Formule que tu connais avec ${}^{-1}$?
	\end{exo}
		
	\begin{exo}
		$\begin{pmatrix}
		\cos(\theta) & -\sin(\theta) \\ 
		\sin(\theta) & \cos(\theta)
		\end{pmatrix}^n ?$
	\end{exo}	

	\begin{exo}
		Calculer $A^{10}$, où:
		$$A = \begin{pmatrix}
		-1 & 0 \\ 
		3 & 2 
		\end{pmatrix}$$
	\end{exo}

	\section*{Colle 3}
	\setcounter{exo}{0}
	FERNANDEZ (10): beaucoup d'erreurs de calculs dans l'application du pivot de Gauss.\\
	Bastien (14): petits soucis de logique pour le raisonnement par analyse/synthese. assez bien sinon.
	\begin{exo}
		Mq toute matrice s'écrit comme mat sym. + mat antisym.
	\end{exo}

	\begin{exo}
		Calculer $A^n$, où:
		$$A = \begin{pmatrix}
		1 & 1 & 1 \\ 
		0 & 1 & 1 \\ 
		0 & 0 & 1
		\end{pmatrix}$$
	\end{exo}

\end{document}
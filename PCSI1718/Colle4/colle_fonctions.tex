\documentclass[10pt,a4paper]{article}
\usepackage[utf8]{inputenc}
\usepackage{amsmath}
\usepackage{amsfonts}
\usepackage{amssymb}
\usepackage{graphicx}
\title{Colle PCSI 8: Sommes, fonctions trigonométriques et hyperboliques.}

\newcounter{question}
\newcommand{\initQ}{\setcounter{question}{0}}
\newenvironment{question}{\addtocounter{question}{1}
	\noindent {\it {Question} \thequestion.\ }}
{\par}
\newcounter{exo}
\newcommand{\Z}{\mathbb{Z}}

\newcommand{\initE}{\setcounter{exo}{0}}
\newenvironment{exo}{\vspace{0.5cm}\setcounter{question}{0}\addtocounter{exo}{1} \noindent \textbf{Exercice \theexo}. \normalsize }{\par}

\begin{document}
	\maketitle
	
	
	\section*{Colle 1}
	\setcounter{exo}{0}
	PETIT Laurine (12): erreur dans les formules de trigo qu'elle arrive à corriger\\
	KHALIL Anas (13): assez bien\\
			
	\begin{exo}
		Formules de trigo?
	\end{exo}

	\begin{exo}
		Minimum de $x \longmapsto x \ln(x)$ sur $\mathbb{R}^{+*}$?
	\end{exo}

	\begin{exo}
		Résoudre l'équation:
		$$\cos(x) - \sqrt{3} \sin(x) = 1$$
	\end{exo}

	Déterminer la forme algébrique du nombre complexe
	
	\section*{Colle 2}
	\setcounter{exo}{0}
	ENZO (12): ne se souvient pas comment prouver une inégalité simple par étude de fonction\\
	GUYOT Marion (14): bien\\
	
	\begin{exo}
	    Dessin de $\sin$ et $\arcsin$. Dérivabilité et dérivée de la fonction $\arcsin$. 
	\end{exo}
	\begin{exo}
		Mq $\ln(1+x) \leq x$ puis: $$(1 + \frac{1}{n})^n \leq e$$
	\end{exo}
		
	\begin{exo}
		Montrer que $f: x \longmapsto \frac{x}{1-x^2}$ est bijective de $]-1, 1[$ dans $\mathbb{R}$ et exprimer sa bijection réciproque.
	\end{exo}	
			
	\section*{Colle 3}
	\setcounter{exo}{0}
	PRIORESCHI (12): multiples erreurs de calcul avec les modules ($\vert -z \vert = - \vert z \vert$...)\\
	JACQUEMARD Steven (12):\\
	
	\begin{exo}
		Définition minorant, majorant, max, min. Mq $sin x \leq x$ 
	\end{exo}

	\begin{exo}
		Minimum de $x \longmapsto x + \frac{1}{x}$ sur $\mathbb{R}^{+*}$?
	\end{exo}
	
	\begin{exo}
		Montrer que $\arctan(x) + \arctan(\frac{1}{x}) = \frac{\pi}{2}$, $\forall x > 0$.
	\end{exo}
	
\end{document}
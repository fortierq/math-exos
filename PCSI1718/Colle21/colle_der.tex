\documentclass[10pt,a4paper]{article}
\usepackage[utf8]{inputenc}
\usepackage{base}
\title{Colle PCSI 21: dérivabilité}

\newcounter{question}
\newcommand{\initQ}{\setcounter{question}{0}}
\newenvironment{question}{\addtocounter{question}{1}
	\noindent {\it {Question} \thequestion.\ }}
{\par}
\newcounter{exo}

\newcommand{\initE}{\setcounter{exo}{0}}
\newenvironment{exo}{\vspace{0.5cm}\setcounter{question}{0}\addtocounter{exo}{1} \noindent \textbf{Exercice \theexo}. \normalsize }{\par}

\begin{document}
	\maketitle
\section*{Colle 1}
\setcounter{exo}{0}
	KHALIL (14): se trompe dans des dérivées simples ($(u^n)'$)\\
	Guillaume (10): très lent, connaissances fragiles. Ne connaît pas le TVI.\\
	
	\begin{exo}
		Dérivée $n$ème de $\frac{1}{1 - x}$? $\frac{1}{1 + x}$?
 	\end{exo}

	\begin{exo}
		$f$ continue décroissante sur $\R$ $\implies$ $f$ a un point fixe
	\end{exo}
	
	\section*{Colle 2}
	Marion (14): se trompe dans la dérivée de $e^u$.\\
	Thibaud (14): oublie l'hypothèse que l'extremum ne doit pas être une borne pour le thm d'extremum local.

	\begin{exo}
		Donner une expression simple pour la dérivée $n$ème de $\cos(t) e^t$?
	\end{exo}

	\section*{Colle 3}
	\setcounter{exo}{0}
	Antonin VERJUS (13): oublie l'égalité des accroissements finis.\\
	Réjane (11): ne connais pas bien le thm d'égalité des accroissements finis, ni le théorème des extremums locaux.\\
	
	\begin{exo}
		Montrer que si $f$ dérivable sur $[a, b]$, $f'(a) < 0$ et $f'(b) > 0$ alors $f'$ s'annule.
	\end{exo}
		
	\begin{exo}
		Si $f \longrightarrow \infty$ en $-/+ \infty$ alors la dérivée de $f$ s'annule.
	\end{exo}
	
\end{document}
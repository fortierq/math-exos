\documentclass[10pt,a4paper]{article}
\usepackage[utf8]{inputenc}
\usepackage{amsmath}
\usepackage{amsfonts}
\usepackage{amssymb}
\usepackage{graphicx}
\title{Colle PCSI 13: matrices et développements limités.}

\newcounter{question}
\newcommand{\initQ}{\setcounter{question}{0}}
\newenvironment{question}{\addtocounter{question}{1}
	\noindent {\it {Question} \thequestion.\ }}
{\par}
\newcounter{exo}
\newcommand{\Z}{\mathbb{Z}}

\newcommand{\initE}{\setcounter{exo}{0}}
\newenvironment{exo}{\vspace{0.5cm}\setcounter{question}{0}\addtocounter{exo}{1} \noindent \textbf{Exercice \theexo}. \normalsize }{\par}

\begin{document}
	\maketitle
	
	
	\section*{Colle 1}
	\setcounter{exo}{0}
	BONNOT (cours: 6, exo: 7, note: 13/20): erreur dans la formule du binôme
	LAABI Amine (cours: 8, exo: 8, note: 16/20): Très bien. \\
	
	\begin{exo}
		Développement limité de arctan en 0 à l'ordre 6. 
	\end{exo}
	
	\begin{exo}
		Formule du binôme de Newton pour les matrices?
	\end{exo}

	\begin{exo}
		DL en 0 à l'ordre 3 de $x \longmapsto \cos(\sin(x))$ ($= 1 - \frac{x^2}{2} + \frac{5x^4}{24} + x^4 \epsilon(x)$).
	\end{exo}	
	
	\begin{exo}
		$\begin{pmatrix}
			\cos(\theta) & -\sin(\theta) \\ 
			\sin(\theta) & \cos(\theta)
		\end{pmatrix}^n ?$
	\end{exo}
	
	\section*{Colle 2}
	\setcounter{exo}{0}
	NACHIN Olivier (cours: 8, exo:9, note: 17/20): Très bien.\\
	
	\begin{exo}
		 Troncature d'un développement limité ;
	\end{exo}

	\begin{exo}
		Formule de Bernoulli pour les matrices?
	\end{exo}

	\begin{exo}
		$DL_3(0)$ de $\exp(\sqrt{1+x})$ ($= e(1 + \frac{x}{2} + \frac{x^3}{48} + o(x^3)$)
	\end{exo}

	\begin{exo}
		\begin{itemize}
			\item Montrer que la somme et le produit de deux matrices nilpotentes qui commutent sont nilpotentes.
			\item Soit $M \in \mathcal{M}_{n, n}$ nilpotente: $M^p = 0$, $p \in \mathbb{N}^*$. Montrer que $I_n$ - $M$ est inversible et déterminer son inverse. \\
			Indice: formule de Bernouilli pour les matrices qui commutent? 
		\end{itemize}
	\end{exo}
				
	\section*{Colle 3}
	\setcounter{exo}{0}
	BELLONCLE (cours: 7, exo: 6, note: 13/20): petite erreur dans le DL\\
	PANIER Estelle (cours: 4, exo: 4, note: 8/20): dit que $A^n$ est obtenue en mettant à la puissance $n$ chaque coefficient. Ne connaît pas $I_3$. Ne connaît pas la formule du binôme.\\
	
	\begin{exo}
		Inversibilité d'une matrice carrée de taille 2.
	\end{exo}

	\begin{exo}
		Quelle est la transposée de $AB$?
	\end{exo}

	\begin{exo}
		$DL_3(0)$ de $\sin(\exp(x) - 1)$?
	\end{exo}
	
	\begin{exo}
		Calculer $A^n$, où:
		$$A = \begin{pmatrix}
			1 & 1 & 1 \\ 
			0 & 1 & 1 \\ 
			0 & 0 & 1
		\end{pmatrix}$$
	\end{exo}
	
\end{document}
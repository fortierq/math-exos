\documentclass[10pt,a4paper]{article}
\usepackage[utf8]{inputenc}
\usepackage{base}
\title{Colle PCSI 15: matrices}

\newcounter{question}
\newcommand{\initQ}{\setcounter{question}{0}}
\newenvironment{question}{\addtocounter{question}{1}
	\noindent {\it {Question} \thequestion.\ }}
{\par}
\newcounter{exo}

\newcommand{\initE}{\setcounter{exo}{0}}
\newenvironment{exo}{\vspace{0.5cm}\setcounter{question}{0}\addtocounter{exo}{1} \noindent \textbf{Exercice \theexo}. \normalsize }{\par}

\begin{document}
	\maketitle
\section*{Colle 1}
\setcounter{exo}{0}
RIONDET Baptiste (16): bien pour $AB$ nilpotente. Petite erreur dans la formule du binôme de Newton.\\
ROMAND Erwyn (17): ne se souvient pas bien de Bernouilli pour les rééls, mais fait l'analogie matrice/réel.\\

\begin{exo}
	\begin{itemize}
		\item Montrer que le produit et la somme de deux matrices nilpotentes qui commutent sont nilpotentes. Contre exemple si les matrices ne commutent pas?
		\item Mq si $A$ et $B$ commutent alors $A^n - B^n = (A - B)(A^{n-1} + A^{n-2}B + ... + B^{n-1})$.
		\item Soit $M \in \mathcal{M}_{n, n}$ nilpotente: $M^p = 0$, $p \in \mathbb{N}^*$. Montrer que $I_n$ - $M$ est inversible et déterminer son inverse.
	\end{itemize}
\end{exo}

	\section*{Colle 2}
	PERRET Emeline (15): oublie l'initialisation de la récurrence. TB sinon.\\
	VENNE Loris (16): petite erreur de signe dans l'inverse d'une matrice $2\times2$. TB sinon.\\
	
	\begin{exo}
		$\begin{pmatrix}
		\cos(\theta) & -\sin(\theta) \\ 
		\sin(\theta) & \cos(\theta)
		\end{pmatrix}^n ?$
	\end{exo}	

	\begin{exo}
		Calculer $A^{10}$, où:
		$$A = \begin{pmatrix}
		-1 & 0 \\ 
		3 & 2 
		\end{pmatrix}$$
	\end{exo}

	\section*{Colle 3}
	\setcounter{exo}{0}
	Oliver Killan (12): a la bonne idée de calculer les 1ères puissances et conjecturer le résultat. un peu lent.\\
	Antonin VERJUS (12): a la bonne idée de calculer les 1ères puissances et conjecturer le résultat. un peu lent.\\
		
	\begin{exo}
		Calculer $A^n$, où:
		$$A = \begin{pmatrix}
		1 & 1 & 1 \\ 
		0 & 1 & 1 \\ 
		0 & 0 & 1
		\end{pmatrix}$$
	\end{exo}

\end{document}
\documentclass[10pt,a4paper]{article}
\usepackage[utf8]{inputenc}
\usepackage{amsmath}
\usepackage{amsfonts}
\usepackage{amssymb}
\usepackage{graphicx}
\usepackage{lmodern,textcomp}
\title{Colle PCSI 7: Sommes et produits}

\newcounter{question}
\newcommand{\initQ}{\setcounter{question}{0}}
\newenvironment{question}{\addtocounter{question}{1}
	\noindent {\it {Question} \thequestion.\ }}
{\par}
\newcounter{exo}
\newcommand{\Z}{\mathbb{Z}}

\newcommand{\initE}{\setcounter{exo}{0}}
\newenvironment{exo}{\vspace{0.5cm}\setcounter{question}{0}\addtocounter{exo}{1} \noindent \textbf{Exercice \theexo}. \normalsize }{\par}

\begin{document}
	\maketitle

	\section*{Colle 1}
	DEMET Louis (8): se souvient de la formule de cours mais pas de la démo. Ecrit $\sum a_k b_k = (\sum a_k) (\sum b_k)$. A déjà oublié les racines de l'unité... \\
	DERRAR Youri (8): ne se souvient pas de la formule ni de la preuve. Ne sait pas appliquer la formule de la somme d'une suite géométrique.
	
	\begin{exo}
		Formule de Bernouilli.
	\end{exo}
	
	\begin{exo}
		Somme et produit des racines n ième de l'unité. Somme des distances des racines nièmes à 1?
	\end{exo}

	\begin{exo}
		Calculer $\sum_{k=1}^{n} cos(k x)$, $\sum \binom{n}{k} \cos(kx)$, $\sum_{k=1}^{n} k cos(k x)$.
	\end{exo}	
	
	\section*{Colle 2}
	\setcounter{exo}{0}
	Réjane GRADELET (10): se souvient de la formule de cours mais pas de la démo. Ne se souvient pas de la formule de la somme des termes d'une suite géométrique.\\
	
	
	\begin{exo}
		Somme des termes d'une suite arithmétique?
	\end{exo}
	
	\begin{exo}
		Calculer: $\sum x^{k}$ puis $\sum k 2^{k-1}$. 
	\end{exo}

	\begin{exo}
		Calculer: 
		$$\sum k k!$$ 
	\end{exo}	
	\section*{Colle 3}
	\setcounter{exo}{0}
	DIEULOT Agathe (18): parfait: se souvient bien de la preuve du binôme, de la méthode pour $\sum k^2$...\\
	DERET Simon (8): Ne connaît pas la formule du binôme et ne sait pas la démontrer. Ecrit $\sum k^2 = (\sum k)(\sum k)$.
	
	\begin{exo}
		Formule du binome de Newton?
	\end{exo}
	\begin{exo}
		Calculer ($\sum (k+1)^3 - k^3$ puis) $\sum k^2$.
	\end{exo}
	
	\begin{exo}
		Calculer $\sum_{k\text{ pair}} \binom{n}{k}$.
	\end{exo}
\end{document}
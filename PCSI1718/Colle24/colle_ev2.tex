\documentclass[10pt,a4paper]{article}
\usepackage{base}

\title{Colle PCSI 24: EV 2}

\newcounter{question}
\newcommand{\initQ}{\setcounter{question}{0}}
\newenvironment{question}{\addtocounter{question}{1}
	\noindent {\it {Question} \thequestion.\ }}
{\par}
\newcounter{exo}
\newcommand{\initE}{\setcounter{exo}{0}}
\newenvironment{exo}{\vspace{0.5cm}\setcounter{question}{0}\addtocounter{exo}{1} \noindent \textbf{Exercice \theexo}. \normalsize }{\par}

\begin{document}
	\maketitle
	
	\section*{Colle 1}
	\setcounter{exo}{0}
	GODEAU Victor (10): ne sait pas définir $F + G$\\
	Adrien (11): rajoute l'hypothèse $F$ et $G$ supplémentaire dans la formule de Grassmann. Dit que $\dim(\s{0}) = 1$. Ne connaît pas la définition de $F + G$\\
	
	\begin{exo}
		Formule de Grassman.
	\end{exo}
		
	\begin{exo}
		Mq l'ens des $x \longmapsto (ax^2 + bx + c)\cos(x)$ forme un EV et déterminer sa dimension
	\end{exo}	
	
	\section*{Colle 2}
	\setcounter{exo}{0}
	MONTEIL Alicia (12): ne fait pas attention avant de multiplier une inégalité par une constante\\
	Hugo MOTTET (16): Bien\\
	
	\begin{exo}
		Thm de la base incomplète
	\end{exo}

	\begin{exo}
		Montrer que ... est une base de $R^3$	
	\end{exo}

	\begin{exo}
		L'ens. des suites croissantes est-il un EV? 
	\end{exo}	

	\section*{Colle 3}
	\setcounter{exo}{0}
	GOUX Alexandre (11): ne sait pas définir le fait d'être supplémentaire. Dit que l'ens des suites est un SEV de $R^2$. Ne pense pas à utiliser la définition de SEV au lieu de EV.\\
	Bastien (13): Ne connaît pas la définition d'une somme directe.\\
	
	\begin{exo}
		Propriétés équivalentes pour être supplémentaire.
	\end{exo}

	\begin{exo}
		Soit $E$ l'ensemble des suites réelles $p$-périodiques.
		Montrer que E est un espace vectoriel de dimension finie et déterminer celle-ci.
	\end{exo}	

\end{document}
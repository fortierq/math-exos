\documentclass[10pt,a4paper]{article}
\usepackage{base}

\title{Colle PCSI 27: applications linéaires}

\newcounter{question}
\newcommand{\initQ}{\setcounter{question}{0}}
\newenvironment{question}{\addtocounter{question}{1}
	\noindent {\it {Question} \thequestion.\ }}
{\par}
\newcounter{exo}
\newcommand{\initE}{\setcounter{exo}{0}}
\newenvironment{exo}{\vspace{0.5cm}\setcounter{question}{0}\addtocounter{exo}{1} \noindent \textbf{Exercice \theexo}. \normalsize }{\par}

\begin{document}
	\maketitle
	
	\section*{Colle 1}

	\setcounter{exo}{0}
	EL Abbassi (16): très bien\\
	Mathilde (13): manque de pratique dans le développement par ligne (oublie les signes, prend la mauvaise sous-matrice...)
	
	\begin{exo}
		Définition du déterminant d'une matrice
	\end{exo}

	\begin{exo}
		Calculer un déterminant (0 sur la diagonale, 1 au dessus, -1 en dessous), par récurrence.
	\end{exo}	

	\section*{Colle 2}

	\setcounter{exo}{0}
	FERNANDEZ (13): dit que $det(A+B) = det(A) + det(B)$. Bien sinon.\\
	Cléo (15): bien.\\
	
	\begin{exo}
		Toutes les propriétés que tu connais sur le déterminant.
	\end{exo}
		
	\begin{exo}
		Calculer un déterminant (0 sur la diagonale, 1 ailleurs), par récurrence.
	\end{exo}	
	
	\section*{Colle 3}
	\setcounter{exo}{0}
	Youri (11): beaucoup d'erreurs, manque de pratique.\\
	Alex (12): oubli pas mal de choses.\\
	
	\begin{exo}
		Définition du déterminant d'un endomorphisme
	\end{exo}

	\begin{exo}
		Calculer un déterminant (a+b sur la diagonale, a juste en bas, b juste en haut), par récurrence.
	\end{exo}
\end{document}
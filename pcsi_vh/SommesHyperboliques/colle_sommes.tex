\documentclass[10pt,a4paper]{article}
\usepackage[utf8]{inputenc}
\usepackage{amsmath}
\usepackage{amsfonts}
\usepackage{amssymb}
\usepackage{graphicx}
\title{Colle PCSI 8: Sommes, fonctions trigonométriques et hyperboliques.}

\newcounter{question}
\newcommand{\initQ}{\setcounter{question}{0}}
\newenvironment{question}{\addtocounter{question}{1}
	\noindent {\it {Question} \thequestion.\ }}
{\par}
\newcounter{exo}
\newcommand{\Z}{\mathbb{Z}}

\newcommand{\initE}{\setcounter{exo}{0}}
\newenvironment{exo}{\vspace{0.5cm}\setcounter{question}{0}\addtocounter{exo}{1} \noindent \textbf{Exercice \theexo}. \normalsize }{\par}

\begin{document}
	\maketitle
	
	\section*{Colle 1}
	\setcounter{exo}{0}
	SEJOURNET Baptiste (cours: 5, exo: 4, note: 9): racine nièmes oubliées. Très lent.\\
	DJEBRA Ines (cours: 5, exo: 5, note: 10): racine nièmes oubliées. ne connait pas les règles de l'exponentielle
	
	\begin{exo}
		Formule de Bernouilli. Propriété des sommes?
	\end{exo}
	
	\begin{exo}
		Somme et produit des racines n ième de l'unité. 
	\end{exo}

	\begin{exo}
		Calculer $\sum k^2$ puis $\sum k^3$.
	\end{exo}	
	
	\section*{Colle 2}
	\setcounter{exo}{0}
	SPADETO Clément (cours: 8, exo: 7, note: 15): bien\\
	DHIFAOUI Mohamed (cours: 6, exo: 7, note: 13): se trompe dans l'interversion de sommes doubles. Sinon bien.
	
	\begin{exo}
		Étude complète des fonctions ch et sh. Qu'est ce que le principe de récurrence?
	\end{exo}
	
	\begin{exo}
		Calculer:
		$$\sum k k!$$ $$\sum k 2^{k-1}$$ $$\prod_{0}^{n} \sin(\frac{x}{2^k})$$
	\end{exo}
	
	\section*{Colle 3}
	\setcounter{exo}{0}
	Agathe:
	
	\begin{exo}
		Formule de Pascal. Autres propriétés du coeff binomial?
	\end{exo}
	\begin{exo}
%		\begin{enumerate}
%			\item Soit $S$ un ensemble de taille $n$. Quel est le nombre de sous ensembles de $S$ de taille $k$? Retrouver la formule de Pascal.
%			\item Montrer qu'il y a autant de sous ensembles de taille pair que de sous ensembles de taille impair, dans $S$.
%			\item Combien y a t-il de sous ensembles de $S$ de taille un multiple de 3?
%		\end{enumerate}
		Calculer $\sum k \binom{n}{k}$, $\sum (-1)^k\binom{n}{k}$, $\sum k(k-1) \binom{n}{k}$...
	\end{exo}
	
\end{document}
\documentclass[10pt,a4paper]{article}
\usepackage[utf8]{inputenc}
\usepackage{amsmath}
\usepackage{amsfonts}
\usepackage{amssymb}
\usepackage{graphicx}
\title{Colle PCSI 21: limite, continuité, analyse asymptotique.}

\newcounter{question}
\newcommand{\initQ}{\setcounter{question}{0}}
\newenvironment{question}{\addtocounter{question}{1}
	\noindent {\it {Question} \thequestion.\ }}
{\par}
\newcounter{exo}
\newcommand{\Z}{\mathbb{Z}}

\newcommand{\initE}{\setcounter{exo}{0}}
\newenvironment{exo}{\vspace{0.5cm}\setcounter{question}{0}\addtocounter{exo}{1} \noindent \textbf{Exercice \theexo}. \normalsize }{\par}

\begin{document}
	\maketitle
	
	
	\section*{Colle 1}
	\setcounter{exo}{0}
	DHIFAOUI Mohamed (cours: 6, exo: 6, note: 12): \\
	LAABI Amine (cours: 5, exo: 6, note: 11): ne connaît pas bien du tout les extremums.\\
	
	\begin{exo}
		Théorème de l'inégalité des accroissements finis
	\end{exo}

	\begin{exo}
		Tout ce que tu connais sur les extremum locaux?
	\end{exo}
	
	\begin{exo}
		Soit $f: \mathbb{R} \rightarrow \mathbb{R}$ continue décroissante. En utilisant $g(x) = f(x) - x$, montrer que $f$ a un point fixe.
	\end{exo}	

	\section*{Colle 2}
	\setcounter{exo}{0}
	SPADETTO Clément (cours: 5, exo: 9, note: 14): erreur dans la preuve du thm: dit $m =\frac{f(x) - f(a)}{x-a}$. Très bien pour les exos.\\
	Kylian LINIGER (cours: 7, exo: 5, note: 12): écrit $(f \circ g)' = g' + f' \circ g$. Ne se souvient pas bien du TVI.\\
	
	\begin{exo}
		Théorème des accroissements finis
	\end{exo}		

	\begin{exo}
		Toutes les opérations sur les dérivées.
	\end{exo}

	\begin{exo}
		Soit $f: [0, 1] \rightarrow [0, 1]$ continue. En utilisant $g(x) = f(x) - x$, montrer que $f$ a un point fixe.
	\end{exo}

	\section*{Colle 3}
	\setcounter{exo}{0}
	STEFFANN (cours: 6, exo: 6, note: 12): parle de $\lim_{n \longrightarrow 0} u_n$. Ne se souvenait pas du thm de convergence monotone.\\
	NACHIN Olivier (cours: 9, exo: 9, note: 18): très précis, très rigoureux.\\
	
	\begin{exo}
		Théorème de la limite de la dérivée.
	\end{exo}		
	
	\begin{exo}
		Tout ce que tu connais sur le TVI?
	\end{exo}
	
	\begin{exo}
		Mq $x + e^x = n$ a une unique solution $\forall n \in \mathbb{N}$. Limite? Equivalent?
	\end{exo}	
\end{document}
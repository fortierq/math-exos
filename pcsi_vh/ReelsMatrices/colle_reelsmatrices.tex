\documentclass[10pt,a4paper]{article}
\usepackage[utf8]{inputenc}
\usepackage{amsmath}
\usepackage{amsfonts}
\usepackage{amssymb}
\usepackage{graphicx}
\title{Colle PCSI 15: suites.}

\newcounter{question}
\newcommand{\initQ}{\setcounter{question}{0}}
\newenvironment{question}{\addtocounter{question}{1}
	\noindent {\it {Question} \thequestion.\ }}
{\par}
\newcounter{exo}
\newcommand{\Z}{\mathbb{Z}}

\newcommand{\initE}{\setcounter{exo}{0}}
\newenvironment{exo}{\vspace{0.5cm}\setcounter{question}{0}\addtocounter{exo}{1} \noindent \textbf{Exercice \theexo}. \normalsize }{\par}

\begin{document}
	\maketitle
	
	
	\section*{Colle 1}
	\setcounter{exo}{0}
	Léa Lurinhg (cours: 6, exo: 5, note: 11): erreurs de calculs dans la méhode de Gauss.\\
	
	\begin{exo}
		Inversion des matrices par la méthode de Gauss-Jordan.
	\end{exo}
	
	\begin{exo}
		Règles de De Morgan?
	\end{exo}
	
	\begin{exo}
		Borne inf et sup de $E = \lbrace \frac{1}{2^n} + \frac{(-1)^n}{n} \rbrace$?
	\end{exo}

	\begin{exo}
	\end{exo}
	
	\section*{Colle 2}
	\setcounter{exo}{0}
	Antonin BONNOT (cours: 7, exo: 8, note: 15): Bonne colle.\\
	FRECHARD Dorian (cours: 5, exo: 7, note: 12): trop approximatif.\\
	
	\begin{exo}
		Toute matrice inversible à gauche (resp. à droite) est inversible.
	\end{exo}
	
	\begin{exo}
		Opérations élémentaires de la méthode de Gauss?
	\end{exo}		

	\begin{exo}
		Inversion de $\begin{pmatrix}
			2 & 3 & 1 \\ 
			1 & 1 & 2 \\ 
			1 & 1 & 1
		\end{pmatrix} $
	\end{exo}		
	\section*{Colle 3}
	\setcounter{exo}{0}
	BELLONCLE Martin (cours: 7, exo: 7, note: 14): bien, un peu lent sur l'exercice.\\
	Nicolas (cours: 6, exo: 8, note: 14): n'a pas bien compris les bornes inf et sup. Très rapide pour la résolution de système linéaire.\\
		
	\begin{exo}
		Propriétés caractéristiques de la partie entière.
	\end{exo}
	
	\begin{exo}
		Enoncé de la décomposition ER.
	\end{exo}	

	\begin{exo}
		Borne sup et inf de $\lbrace (-1)^n + \frac{1}{n+1} \rbrace$?
	\end{exo}	
	
\end{document}
\documentclass[10pt,a4paper]{article}
\usepackage[utf8]{inputenc}
\usepackage{amsmath}
\usepackage{amsfonts}
\usepackage{amssymb}
\usepackage{graphicx}
\title{Colle PCSI 10: Fonctions et intégrales	.}

\newcounter{question}
\newcommand{\initQ}{\setcounter{question}{0}}
\newenvironment{question}{\addtocounter{question}{1}
	\noindent {\it {Question} \thequestion.\ }}
{\par}
\newcounter{exo}
\newcommand{\Z}{\mathbb{Z}}

\newcommand{\initE}{\setcounter{exo}{0}}
\newenvironment{exo}{\vspace{0.5cm}\setcounter{question}{0}\addtocounter{exo}{1} \noindent \textbf{Exercice \theexo}. \normalsize }{\par}

\begin{document}
	\maketitle
	
	
	\section*{Colle 1}
	\setcounter{exo}{0}
	GUES Flora (cours: 6/10, exo: 6/10, note: 12/20): erreur dans les dérivées et primitives de $\frac{1}{x}$.\\
	HENRY(cours: 6/10, exo: 6/10, note: 12/20): ne sait pas mettre deux fractions au même dénominateur... peu rigoureux.
	
	\begin{exo}
		Dérivabilité de la fonction $\arcsin$. 
	\end{exo}
	
	\begin{exo}
		Méthode de calcul d'une primitive $\frac{1}{P(x)}$, $P$ de degré 2.
	\end{exo}

	\begin{exo}
		Mq $\forall x \geq 0$, $ \arctan(x) \geq \frac{x}{x^2 + 1}$.
	\end{exo}

	\begin{exo}
		Soit $I_{p, q} = \int_{0}^{1} t^p (1 - t)^q dt$. 
		\begin{itemize}
			\item Mq $$I_{p, q} = \frac{q}{p+1} I_{p+1, q-1}$$
			\item Mq $$I_{p, q} = \frac{p! q!}{(p+q+1)!}$$
			\item Calculer $$\sum_{k=0}^{q} \binom{q}{k} \frac{(-1)^k}{p+k+1}$$
		\end{itemize}
	\end{exo}
	
	\section*{Colle 2}
	\setcounter{exo}{0}
	MARGUERITE Léa (cours: 6/10, exo: 7/10, note: 13/20): dessine arcsin avec deux images pour le même argument.\\
	GUILLAUME-SAGE (cours: 5/10, exo: 5/10, note: 10/20): ne sait pas changer de variable, écrit que $\sin(1) = 0$.\\
	Commentaire mythique: ah mais la formule de linéarisation s'applique aussi si on a $2x$ au lieu de $x$!\\
	
	\begin{exo}
		Théorème d'intégration par parties
	\end{exo}

	\begin{exo}
		Dessin de $\sin$ et $\arcsin$.
	\end{exo}
	
	\begin{exo}
		Calculer $\int_{-1}^{1} t^2 \sqrt{1 - t^2} dt$ (= $\frac{\pi}{8}$)
	\end{exo}
			
	\begin{exo}
		Résoudre $\arcsin(\tan(x)) = x$.
	\end{exo}
			
	\section*{Colle 3}
	\setcounter{exo}{0}
	MOUILLEFARINE Paul (cours: 6/10, exo: 8/10, note: 14/20): erreurs ds la formule de chgt de variable.\\
	FRICK (cours: 7/10, exo: 7/10, note: 14/20): erreur de primitive de $\exp(3t)$.\\
	
	\begin{exo}
		Unicité des primitives, à addition d'une constante près.
	\end{exo}

	\begin{exo}
		Formule de chgt de variable?
	\end{exo}

	\begin{exo}
		Calculer $\arcsin \cos \frac{7 \pi}{4}$ (= $\frac{\pi}{4}$).
	\end{exo}
	
	\begin{exo}
		Primitive de $x \longmapsto \frac{1}{x^2 - x - 1}$?
	\end{exo}

	\begin{exo}
		$\int_{0}^{\pi} \exp(t) \sin(3t) dt$?
	\end{exo}	
\end{document}
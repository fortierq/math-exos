\documentclass[10pt,a4paper]{article}
\usepackage[utf8]{inputenc}
\usepackage{amsmath}
\usepackage{amsfonts}
\usepackage{amssymb}
\usepackage{graphicx}
\title{Colle PCSI 29: Séries et intégration.}

\newcounter{question}
\newcommand{\initQ}{\setcounter{question}{0}}
\newenvironment{question}{\addtocounter{question}{1}
	\noindent {\it {Question} \thequestion.\ }}
{\par}
\newcounter{exo}
\newcommand{\Z}{\mathbb{Z}}

\newcommand{\initE}{\setcounter{exo}{0}}
\newenvironment{exo}{\vspace{0.5cm}\setcounter{question}{0}\addtocounter{exo}{1} \noindent \textbf{Exercice \theexo}. \normalsize }{\par}

\begin{document}
	\maketitle
% TODO: exo 95 MPSIDDL Applications linéaires
	
	\section*{Colle 1}
	\setcounter{exo}{0}
	MARION Caroline (cours: 3, exo: 6, note: 9): ne connaît pas la méthode de comparaison série-intégrale, et mal les théorèmes de comparaison.\\
	Clément SPADETTO (cours: 8, exo: 7, note: 15): bien\\
	
	\begin{exo}
		 Croissance de l'intégrale
	\end{exo}

	\begin{exo}
		Cours: Comparaison de deux séries?
	\end{exo}

	\begin{exo}
		Sommes de Bertrand: CV de $\sum \frac{1}{n \log(n)}$?
	\end{exo}	
	
	\begin{exo}
		CV et limite de $\sum{\frac{k}{2^k}}$ (avec indications)
	\end{exo}
	
	\section*{Colle 2}
	\setcounter{exo}{0}
	TURCK Bertrand (exo: 8, cours: 7, note: 15): Bien\\
	DHIFAOUI Mohamed (exo: 7, cours: 8, note: 15): Bien\\
	
	\begin{exo}
			Formule Taylor reste intégral
	\end{exo}		
	
	\begin{exo}
		Somme Riemann?
	\end{exo}		
	
	\begin{exo}
		Limite de $\sum_{k=n}^{2n} \sin(\frac{\pi}{k})$?
	\end{exo}

	\section*{Colle 3}
	\setcounter{exo}{0}
	Thomas MIGOT (cours: 6, exo: 8, note: 14): ne se souvient plus du thm $\int f = 0$, $f \geq 0$ $\implies$ $f = 0$\\
	STEFFANN Axelle (cours: 3, exo: 5, note: 8): ne connaît pas la méthode de comparaison série/intégrale. dit que $u_n \longrightarrow 0$ $\implies$ $\sum u_n$ CV car "on a déjà utilisé ce résultat en TD". Bonne blague! Et dit aussi que $ln(n) \longrightarrow 0$ en $\infty$... \\
	
	\begin{exo}
		$\sum u_n$ CV $\implies$ $u_n \longrightarrow 0$
	\end{exo}		
	
	\begin{exo}
		Formule Taylor?
	\end{exo}

	\begin{exo}
		Mq $\sum \frac{n}{2^n}$ CV et déterminer sa somme.
	\end{exo}
	
	\begin{exo}
	Soit $f : [0,1] \longrightarrow [0, 1]$ continue tq $\int f^2 = \int f$. Mq $f = 1$.
	\end{exo}
	
\end{document}	
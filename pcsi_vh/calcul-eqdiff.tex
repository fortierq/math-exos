\documentclass[10pt,a4paper]{article}
\usepackage[T1]{fontenc}
\usepackage[utf8]{inputenc}
\usepackage{amsmath}
\usepackage{amsfonts}
\usepackage{amssymb}
\usepackage{graphicx}
\title{Colle PCSI: Techniques de calcul, Équations différentielles}

\newcounter{question}
\newcommand{\initQ}{\setcounter{question}{0}}
\newenvironment{question}{\addtocounter{question}{1}
	\noindent {\it {Question} \thequestion.\ }}
{\par}
\newcounter{exo}

\newcommand{\initE}{\setcounter{exo}{0}}
\newenvironment{exo}{\vspace{0.5cm}\setcounter{question}{0}\addtocounter{exo}{1} \noindent \textbf{Exercice \theexo}. \normalsize }{\par}

\begin{document}
	\maketitle
	
	\section*{Colle 1}
	Karim Malmaud 11\\
	Léa Schamberger 11\\
	\begin{exo}
			Donner le théorème de décomposition en facteurs premiers, et expliquer comment en déduire le PGCD et le PPCM de deux nombres. Calculer le PGCD et PPCM de 966 et 980.
	\end{exo}

	\begin{exo}
		Résoudre le système suivant:
		\[ \begin{cases} 
		x_1 + x_2 + x_3 + x_4 = 0 \\
		2 x_1 + x_2 + x_3 = 1 \\
		x_1 - x_4 = 1 \\
		3 x_1 + x_2 + x_3 - x_4 = 2 
		\end{cases}
		\]	
	\end{exo}
	
	\begin{exo}
		Soient $A$ et $B$ deux sous groupes d'un groupe $G$. Soit $AB = \lbrace ab \in G : a \in A, b \in B \rbrace$. \\
		\begin{question}
			Est-ce que $AB$ est forcément un sous groupe de $G$?
		\end{question}
		\begin{question}
			Démontrer que $AB$ est un sous groupe de $G$ ssi $AB = BA$.
		\end{question}
		\begin{question}
			Démontrer que, si $AB = BA$, $AB$ est le sous groupe de $G$ engendré par $A \cup B$.
		\end{question}
	\end{exo}

	\section*{Colle 2}
	\setcounter{exo}{0}
	Christian Morello 14\\
	Fournier Julie 15\\
	\begin{exo}
			Quels sont les sous groupes de (Z, +)? Preuve?
	\end{exo}
	\begin{exo}
			Démontrer que, pour tout $x \in ]0, \frac{\pi}{2}$, $\frac{2}{\pi}] \leq \frac{\sin(x)}{x} \leq 1$.
	\end{exo}
	

	 
\end{document}
\documentclass[10pt,a4paper]{article}
\usepackage[utf8]{inputenc}
\usepackage{amsmath}
\usepackage{amsfonts}
\usepackage{amssymb}
\usepackage{graphicx}
\title{Colle PCSI 24: polynômes et EV.}

\newcounter{question}
\newcommand{\initQ}{\setcounter{question}{0}}
\newenvironment{question}{\addtocounter{question}{1}
	\noindent {\it {Question} \thequestion.\ }}
{\par}
\newcounter{exo}
\newcommand{\Z}{\mathbb{Z}}

\newcommand{\initE}{\setcounter{exo}{0}}
\newenvironment{exo}{\vspace{0.5cm}\setcounter{question}{0}\addtocounter{exo}{1} \noindent \textbf{Exercice \theexo}. \normalsize }{\par}

\begin{document}
	\maketitle
	
	
	\section*{Colle 1}
	\setcounter{exo}{0}
	MAMEDOV Djémali (cours: 4, exo: 7, note: 11): dit que $K[X]$ est un ensemble de matrices. Ne se souvient pas des relations coeff-racines.\\
	MARTI Sébastien (cours: 6, exo: 7, note: 13): confond degré et multiplicité. 
	
	\begin{exo}
		 L'intersection de deux sous-espaces vectoriels est un sous-espace vectoriel. 
	\end{exo}

	\begin{exo}
		Donner des exemples d'EV.
	\end{exo}
	
	\begin{exo}
		Calculer la somme des racines $n$ième de l'unité
	\end{exo}	
	
	
	\section*{Colle 2}
	\setcounter{exo}{0}
	BROUILLARD Alizée (cours: 8, exo: 7, note: 15): Bien\\
	MARGUERITE Léa (cours: 7, exo: 5, note: 12): écrit $(0, X, X^2, ...)$ comme base de $\mathbb{R}[X]$.\\
			
	\begin{exo}
		 Propriété fondamentale des familles libres : possibilité d'identifier les coefficients dans les combinaisons linéaires. 
	\end{exo}		

	\begin{exo}
		Tout ce que tu connais sur les bases?
	\end{exo}

	\begin{exo}
		\begin{enumerate}
			\item Si $x_0$ est racine de $P$ avec multiplicité $m$, mq $x_0$ est racine de $P'$ avec multiplicité $m - 1$.
			\item Mq si $P$ est scindé et deg($P$) $\geq 2$ alors $P'$ est scindé.
		\end{enumerate}
	\end{exo}

	\section*{Colle 3}
	\setcounter{exo}{0}
	BOUAZA Yakoub (cours: 5, exo: 6, note: 11): ne se souvient pas des relations coefficients-racines\\
	MOUILLEFARINE Paul (cours: 7, exo: 8, note: 15): oublie l'hypothèse scindé pour les relations coefficients-racines.\\
	
	\begin{exo}
		$K_n$[X] est un sous-espace vectoriel de K[X]
	\end{exo}		
	
	\begin{exo}
		Relations entre coefficients et racines.
	\end{exo}
		
	\begin{exo}
		Trouver les $P \in C[X]$ vérifiant
		$P (X^2 ) = P (X)P (X + 1)$.\\
		
		Indice: quelles sont les racines possibles?
	\end{exo}	
\end{document}
\documentclass[10pt,a4paper]{article}
\usepackage[utf8]{inputenc}
\usepackage{amsmath}
\usepackage{amsfonts}
\usepackage{amssymb}
\usepackage{graphicx}
\title{Colle PCSI 20: limite, continuité, analyse asymptotique.}

\newcounter{question}
\newcommand{\initQ}{\setcounter{question}{0}}
\newenvironment{question}{\addtocounter{question}{1}
	\noindent {\it {Question} \thequestion.\ }}
{\par}
\newcounter{exo}
\newcommand{\Z}{\mathbb{Z}}

\newcommand{\initE}{\setcounter{exo}{0}}
\newenvironment{exo}{\vspace{0.5cm}\setcounter{question}{0}\addtocounter{exo}{1} \noindent \textbf{Exercice \theexo}. \normalsize }{\par}

\begin{document}
	\maketitle
	
	
	\section*{Colle 1}
	\setcounter{exo}{0}
	GUILLAUME-SAGE (cours: 4, exo: 7, note: 11): erreur dans l'énoncé de cours\\
	
	\begin{exo}
		Formules de calcul avec les fonctions négligeables : transitivité, insensibilité aux constantes multiplicatives, sommes de ô, changement de variable
	\end{exo}

	\begin{exo}
		Donner le domaine de définition et de continuité de $f(x) = ...$
	\end{exo}
	
	\begin{exo}
		Soit $f: [0, 1] \rightarrow [0, 1]$ continue. En utilisant $g(x) = f(x) - x$, montrer que $f$ a un point fixe.
	\end{exo}

	\section*{Colle 2}
	\setcounter{exo}{0}
	HENRY (cours: 6, exo: 6, note: 12): oublie de vérifier que l'expression sous la racine est positif...\\
	
	\begin{exo}
		Théorème des valeurs intermédiaires
	\end{exo}		

	\begin{exo}
		Donner le domaine de définition et de continuité de $f(x) = ...$
	\end{exo}

	\section*{Colle 3}
	\setcounter{exo}{0}
	
\end{document}
\documentclass[10pt,a4paper]{article}
\usepackage[utf8]{inputenc}
\usepackage{amsmath}
\usepackage{amsfonts}
\usepackage{amssymb}
\usepackage{graphicx}
\title{Colle PCSI 19: ensembles, applications, dénombrement.}

\newcounter{question}
\newcommand{\initQ}{\setcounter{question}{0}}
\newenvironment{question}{\addtocounter{question}{1}
	\noindent {\it {Question} \thequestion.\ }}
{\par}
\newcounter{exo}
\newcommand{\Z}{\mathbb{Z}}

\newcommand{\initE}{\setcounter{exo}{0}}
\newenvironment{exo}{\vspace{0.5cm}\setcounter{question}{0}\addtocounter{exo}{1} \noindent \textbf{Exercice \theexo}. \normalsize }{\par}

\begin{document}
	\maketitle
	
	
	\section*{Colle 1}
	\setcounter{exo}{0}
	Couriol Clément (cours: 5, exo: 6, note: 11/20): ne se souvient pas bien des arrangements. Moyen pour l'exo.\\
	MAMEDOV (cours: 6, exo: 5, note: 11/20): ne se souvient pas bien des arrangements. dit qu'une fonction injective de E dans E est bijective\\
	
	\begin{exo}
		La composée de deux applications injective (resp. surjective, bijective) est injective (resp. surjective, bijective).
	\end{exo}
	
	\begin{exo}
		Arrangement: définition, nombre de $p$-arrangements, nombre d'injection de $E$ dans $F$ est le nombre de $\vert E \vert$ - arrangements sur $F$, 2-arrangements de $\lbrace 1, 2, 3 \rbrace$.
	\end{exo}

	\begin{exo}
		Mq si $f : E \longrightarrow E$ vérifie $f \circ f = f$ et $f$ injective ou surjective alors $f = id$. 
	\end{exo}

	\begin{exo}
		Combien y a t-il de sous-ensembles de taille pair d'un ensemble à $n$ éléments? Utiliser 2 méthodes.
	\end{exo}
	
	\section*{Colle 2}
	\setcounter{exo}{0}
	Mathieu Collilieux (cours: 6, exo: 6, note: 12): parle de réciproque d'une fonction sans se soucier de savoir si elle est bijective. Peu rigoureux.\\
	BROUILLARD Alizée (cours: 7, exo: 4, note: 11): se noie dans le brouillard avec ses notations.\\
	
	\begin{exo}
		Caractérisation des applications injectives
	\end{exo}
	
	\begin{exo}
		Formule de Poincaré.
	\end{exo}		

	\begin{exo}
		Si $f \circ f \circ f = f$ alors $f$ surjective $\Longleftrightarrow$ $f$ injective.
	\end{exo}	

	\section*{Colle 3}
	\setcounter{exo}{0}
	GUES Flora (cours: 8, exo: 7, note: 15/20): bien, mais manque d'initiative pour l'exo. \\
	Bouaza Yakoub (cours: 6, exo:6, note: 12): confonds cardinal et ensemble\\
	
	\begin{exo}
		Cardinal d'une réunion d'ensembles deux à deux disjoints
	\end{exo}
	
	\begin{exo}
		Permutations et combinaisons.
	\end{exo}	

	\begin{exo}
		Si $h \circ g \circ f$ et $g \circ f \circ h$ surjectives et $f \circ h \circ g$ est injective alors $f, g, h$ sont bijectives.
	\end{exo}		

	
\end{document}
\documentclass[10pt,a4paper]{article}
\usepackage[utf8]{inputenc}
\usepackage[T1]{fontenc}
\usepackage{amsmath}
\usepackage{amsfonts}
\usepackage{amssymb}
\usepackage{graphicx}
\title{Colle PCSI 30: Séries, intégration et représentation matricielle.}

\newcounter{question}
\newcommand{\initQ}{\setcounter{question}{0}}
\newenvironment{question}{\addtocounter{question}{1}
	\noindent {\it {Question} \thequestion.\ }}
{\par}
\newcounter{exo}
\newcommand{\Z}{\mathbb{Z}}

\newcommand{\initE}{\setcounter{exo}{0}}
\newenvironment{exo}{\vspace{0.5cm}\setcounter{question}{0}\addtocounter{exo}{1} \noindent \textbf{Exercice \theexo}. \normalsize }{\par}

\begin{document}
	\maketitle
	
	\section*{Colle 1}
	\setcounter{exo}{0}
	MARTI Sébastien (cours: 7, exo: 8, note: 15):\\
	TONDU Camille (cours: 7, exo: 7, note: 14): comparaison série intégrale bien connue
	
	\begin{exo}
		 Étude des séries géométriques
	\end{exo}

	\begin{exo}
		Méthode comparaison série intégrale?
	\end{exo}

	\begin{exo}
		Sommes de Bertrand: CV de $\sum \frac{1}{n \log(n)}$?
	\end{exo}	
	
	\begin{exo}
		CV et limite de $\sum{\frac{k}{2^k}}$ (avec indications)
	\end{exo}
	
	\section*{Colle 2}
	\setcounter{exo}{0}
	MARGUERITE Léa (exo: 4, cours: 5, note: 9): ne sait pas comment montrer le thm sur les séries de Riemann. ne connaît pas le principe de comparaison série-intégrale.\\
	SAULDUBOIS Robin (cours: 6, exo: 5, note: 11): dessine $\cos$ quand je lui demande $\sin$. Ne sait pas démontrer $\sin(x) \leq x$\\
	
	\begin{exo}
		Formule de changement de base pour les endomorphismes
	\end{exo}
	
	\begin{exo}
		Cours: Comment calculer le rang d’une matrice? Complexité?
	\end{exo}		
	
	\begin{exo}
		Théorèmes de comparaisons des séries.
	\end{exo}		
	
	\begin{exo}
		Limite de $\sum_{k=n}^{2n} \sin(\frac{\pi}{k})$?
	\end{exo}

	\section*{Colle 3}
	\setcounter{exo}{0}
	MOUILLEFARINE Paul (cours: 8, exo:  8, note: 16): bien, prends des initiatives.\\
	STUDER Ulysse (cours: 5, exo: 5, note: 10): démo de cours légèrement approximative. ne sait pas mettre une somme simple sous forme de somme de Riemann.\\
	
	\begin{exo}
		Théorème du calcul intégral
	\end{exo}		
	
	\begin{exo}
		Formule Taylor reste intégral.
	\end{exo}

	\begin{exo}
		$\prod_{k=1}^{n} (1+\frac{k}{n})^{\frac{1}{n}}$?
	\end{exo}
	
	\begin{exo}
	Soit $f : [0,1] \longrightarrow [0, 1]$ continue tq $\int f^2 = \int f$. Mq $f = 1$.
	\end{exo}
	
\end{document}	
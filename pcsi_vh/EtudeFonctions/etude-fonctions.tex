\documentclass[10pt,a4paper]{article}
\usepackage[utf8]{inputenc}
\usepackage{amsmath}
\usepackage{amsfonts}
\usepackage{amssymb}
\usepackage{graphicx}
\title{Colle PCSI 1: ED, équations}

\newcounter{question}
\newcommand{\initQ}{\setcounter{question}{0}}
\newenvironment{question}{\addtocounter{question}{1}
	\noindent {\it {Question} \thequestion.\ }}
{\par}
\newcounter{exo}
\newcommand{\Z}{\mathbb{Z}}

\newcommand{\initE}{\setcounter{exo}{0}}
\newenvironment{exo}{\vspace{0.5cm}\setcounter{question}{0}\addtocounter{exo}{1} \noindent \textbf{Exercice \theexo}. \normalsize }{\par}

\begin{document}
	\maketitle
	
	\section*{Colle 1}
	PERRET Emeline (note: 14): assez bien mais ne pense pas à diviser par 2 pour résoudre 2y' + y = 0\\
	DERET Simon (note: 9): ne connaît pas du tout le thm des solutions des ED d'ordre 2. Dérive par rapport à la mauvaise variable... dit que (C u)' = 0 si C est constant... ne sait pas résoudre une éq du 2nd degré.
	
	\begin{exo}
		Ensemble des solutions d'une ED ordre 2?
	\end{exo}
	\begin{exo}
		Résoudre $x_1 + x_2 =4$, $x_1 x_2 = -1$.
	\end{exo}
	
	\section*{Colle 2}
	\setcounter{exo}{0}
	OLIVER Killian (note: 8): n'a aucune idée de ce qu'est une dérivée. Ne sait pas faire de tableau de variation, dit plusieurs fois que sin(0) =0, que $\sin' = - \cos$, ne sait pas dessiner $\sin$... Dit $f' \geq 0 \implies f \geq 0$\\
	DIVOUX Gaëlle (note: 13): assez bien sur la def de dérivée. Dit $f' \geq 0 \implies f \geq 0$ mais se rattrape
	
	\begin{exo}
		Définition dérivée? Géométriquement?
	\end{exo}

	\begin{exo}
		Etudier $f(x) = x - sin(x)$ puis montrer que $x \leq sin(x)$ $\forall x \geq 0$
	\end{exo}	
	
	\section*{Colle 3}
	\setcounter{exo}{0}
	RIONDET Baptiste (note: 15): bien mais ne pense pas à diviser par 2 pour résoudre 2y' + y = 0\\
	DERRAR Youri (note: 12): ne pense pas à diviser par 2 pour résoudre 2y' + y = 0
	
	\begin{exo}
		Toutes les opérations sur les dérivées?
	\end{exo}

	\begin{exo}
		ED du 1er ordre.
	\end{exo}	
	
	\begin{exo}
		Solution de $yy' + y^2 = 1$. Indice: poser $z = y'$ puis trouver $z$.
	\end{exo}	
	 
\end{document}
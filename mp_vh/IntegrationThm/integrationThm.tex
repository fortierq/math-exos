\documentclass[10pt,a4paper]{article}
\usepackage{base}
\title{Colle MP 18: théorèmes d'intégration}

\newcounter{question}
\newcommand{\initQ}{\setcounter{question}{0}}
\newenvironment{question}{\addtocounter{question}{1}
	\noindent {\it {Question} \thequestion.\ }}
{\par}
\newcounter{exo}

\newcommand{\initE}{\setcounter{exo}{0}}
\newenvironment{exo}{\vspace{0.5cm}\setcounter{question}{0}\addtocounter{exo}{1} \noindent \textbf{Exercice \theexo}. \normalsize }{\par}

\begin{document}
	\maketitle
	
	\section*{Colle 1}
	Stepan (15): petite confusion sur les variables ($x$ et $t$)\\
	Valentin (16): petite confusion sur les variables ($x$ et $t$)\\
	
	\begin{exo}
		TCD
	\end{exo}

	\begin{exo}
		Mq $\lim_{n \rightarrow \infty} \int_0^n (1 - \frac{x^2}{n^2})^{n^2}$ = $\int_0^\infty e^{-x^2}dx$.
	\end{exo}

	\begin{exo} (131)
		\begin{enumerate}
			\item Mq $\int_0^\infty \frac{sin(t)}{t} dt$ est bien définie.
			\item Soit $F(x) = \int_{0}^{\infty} \frac{e^{-xt} \sin t}{t}$. Quelle est $\lim_{\infty} F(x)?$
			\item Calculer $F'$ sur $]0, \infty[$.
			\item En supposant $F$ continue en 0, calculer $I$.
		\end{enumerate}
	\end{exo}
		
	\section*{Colle 2}
	\setcounter{exo}{0}
	Tom (16): bien\\
	Achille (14): ne pense pas à utiliser les questions précédentes de l'exercice\\
	
	\begin{exo}
		intégration terme à terme d’une série de fonctions	\end{exo}

	\begin{exo}
		Soit $f(x) = \int_{0}^{x} e^{-t^2}$ et $g(x) = \int_{0}^{1} \frac{e^{-x^2 (1+t^2)}}{1+t^2}$.\\
		Mq $g$ est dérivable et $g'(x) = -2f'(x) f(x)$.\\
		Mq $g(x) + f(x)^2 = \frac{\pi}{4}$
		En déduire $\lim_\infty f(x)$.
	\end{exo}
	
	\begin{exo}
		Calcul de $\int_0^{\infty} \frac{\sin(t)}{t} e^{-tx} dt$, $\forall x > 0$.
	\end{exo}
		
	\section*{Colle 3}
	\setcounter{exo}{0}
	Lily (16): bien\\
	Julien Rauch (13): majore par une fonction constante sur $\mathbb{R}$ (en disant qu'elle est intégrable). Manque parfois de rigueur (écrit par exemple $\lim f(x) \longrightarrow ...$). \\
	Arnaud (13): bien mais lent\\
	
	\begin{exo}
		dérivabilité sous le signe intégrale
	\end{exo}

	\begin{exo} (29)
		Mq $\int_0^\infty \frac{t}{e^t - 1} dt$ = $\sum_1^\infty \frac{1}{n^2}$.
	\end{exo}
	
	\begin{exo}
		Limite puis équivalent de $\int_{1}^{\infty} e^{-x^n} dx$? (poser $t = x^n$)
	\end{exo}

	\begin{exo}
		Calcul de $\int_0^{\infty} e^{-t^2} \cos(xt) dt$?
	\end{exo}

\end{document}
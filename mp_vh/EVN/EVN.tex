\documentclass[10pt,a4paper]{article}
\usepackage[utf8]{inputenc}
\usepackage{amsmath}
\usepackage{amsfonts}
\usepackage{amssymb}
\usepackage{graphicx}
\usepackage[french]{babel}
\title{Colle MP: EVN}

\newcounter{question}
\newcommand{\initQ}{\setcounter{question}{0}}
\newenvironment{question}{\addtocounter{question}{1}
	\noindent {\it {Question} \thequestion.\ }}
{\par}
\newcounter{exo}
\newcommand{\Z}{\mathbb{Z}}

\newcommand{\initE}{\setcounter{exo}{0}}
\newenvironment{exo}{\vspace{0.5cm}\setcounter{question}{0}\addtocounter{exo}{1} \noindent \textbf{Exercice \theexo}. \normalsize }{\par}

\begin{document}
	\maketitle
	
	\section*{Colle 1}
	Mathilde (13): petites erreurs de quantificateurs\\
	Adrien (14): petites erreurs de quantificateurs\\
		
	\begin{exo}
	Cours: tout compact est fermé borné
	\end{exo}
	
	\begin{exo}
		Mq $GL_n(\mathbb{R})$ n'est pas connexe par arc. Mq $GL_n(\mathbb{C})$ est connexe par arc. Est-il fermé? Convexe?
	\end{exo}
		
	\begin{exo}
		Soit $E$ un ensemble fini de $\mathbb{R}^2$. Mq $\mathbb{R}^2 - E$ est connexe par arc.
	\end{exo}
	
	\section*{Colle 2}
	\setcounter{exo}{0}
	GUALDI Baptiste (13): oubli du thm des bornes.\\
	Bérenger (14): bien\\
	
	\begin{exo}
		Cours: l'image d'un connexe par arc par une fonction continue est connexe par arc
	\end{exo}

%	\begin{exo}
%		Soit $A$ un ensemble convexe d'un EVN. Mq son intérieur est convexe.
%	\end{exo}
	
	\begin{exo}
		Soit $f : K \longmapsto K$, $K$ compacte d'un EVN tq, $\forall x \neq y$:
		$$\Arrowvert f(x) - f(y) \Arrowvert < \Arrowvert x - y \Arrowvert$$ 
	\begin{enumerate}
		\item (En considérant $x \longmapsto \Arrowvert f(x) - x \Arrowvert$), mq $f$ a un unique point fixe $c$.\\
		\item Soit $x_n$ tq $x_{n+1} = f(x_n)$ et $x_0 \in K$. Montrer que $x_n \longrightarrow c$.
	\end{enumerate}
	\end{exo}

	\begin{exo}
		Mq l'intérieur d'un convexe est convexe. Est-ce vrai en remplaçant convexe par connexe par arc?
	\end{exo}
	
	\section*{Colle 3}
	\setcounter{exo}{0}
	Nathan (13): petites erreurs pour écrire la négation d'une proposition.\\
	Théo (15): bien\\
	
	\begin{exo}
		Thm de Heine.
	\end{exo}
	
	\begin{exo}
		Montrer que l'union de deux connexes par arcs non disjoints est connexe par arcs. Et pour l'intersection?
	\end{exo}

	\begin{exo}
		Mq $GL_n(\mathbb{C})$ est un ouvert dense dans $M_n(\mathbb{C})$.\\
		
		Mq $\chi(AB) = \chi(BA)$ pour $A\in GL_n(\mathbb{C})$ puis $A, B$ quelconque. 
	\end{exo}
		 
\end{document}
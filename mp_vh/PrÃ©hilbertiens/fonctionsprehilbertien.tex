\documentclass[10pt,a4paper]{article}
\usepackage[utf8]{inputenc}
\usepackage{lmodern}
\usepackage{amsmath}
\usepackage{amsfonts}
\usepackage{amssymb}
\usepackage{graphicx}
\usepackage[french]{babel}
\title{Colle MP 10: séries de fonctions, espaces préhilbertiens}

\newcounter{question}
\newcommand{\initQ}{\setcounter{question}{0}}
\newenvironment{question}{\addtocounter{question}{1}
	\noindent {\it {Question} \thequestion.\ }}
{\par}
\newcounter{exo}
\newcommand{\Z}{\mathbb{Z}}

\newcommand{\initE}{\setcounter{exo}{0}}
\newenvironment{exo}{\vspace{0.5cm}\setcounter{question}{0}\addtocounter{exo}{1} \noindent \textbf{Exercice \theexo}. \normalsize }{\par}

\begin{document}
	\maketitle
	
	\section*{Colle 1}
	Slava (14): bien\\
	Reda (10): démo de cours mal sue
		
	\begin{exo}
		Implications des différentes CV.
	\end{exo}

	\begin{exo}
		Trouver le minimum de $\sum x_i^2$ tels que $\sum x_i = K$.
	\end{exo}
	
	\begin{exo}
		Soit $f : ]0, \infty[ \longrightarrow \mathbb{R}$, $f(x) = \sum_{k=1}^{\infty}	\frac{1}{sh(kx)}$.\\
		Donner un équivalent de $f$ en $\infty$.
	\end{exo}
	
	\section*{Colle 2}
	\setcounter{exo}{0}
	Arthur S (13): assez bien\\
	Nino (11): quelques bêtises
	
	\begin{exo}
		cours
	\end{exo}

	\begin{exo}
		CU de $x^n \ln(x)$?
	\end{exo}
		
	\begin{exo}
		définition, continuité de $\sum \exp(-x\sqrt{n})$?
	\end{exo}

	\section*{Colle 3}
	\setcounter{exo}{0}
	Elsa (13): écrit $e^{x^{\sqrt{n}}} = e^{x\frac{n}{2}}$.\\
	PROST Vincent (13): confond nombre et vecteur. bien pour démo de cours.
	
	\begin{exo}
		Cauchy Schwartz
	\end{exo}

	\begin{exo}
		CU de $\sqrt{x^2 + \frac{1}{n}}$?
%		Calculer $\lim_\infty \int_{-1}^{1} f_n(x) dx$.	
	\end{exo}
	
	\begin{exo}
		
	\end{exo}
	
\end{document}
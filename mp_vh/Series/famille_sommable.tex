\documentclass[10pt,a4paper]{article}
\usepackage[utf8]{inputenc}
\usepackage{amsmath}
\usepackage{amsfonts}
\usepackage{amssymb}
\usepackage{graphicx}
\usepackage[french]{babel}
\title{Colle MP 8: Séries}

\newcounter{question}
\newcommand{\initQ}{\setcounter{question}{0}}
\newenvironment{question}{\addtocounter{question}{1}
	\noindent {\it {Question} \thequestion.\ }}
{\par}
\newcounter{exo}
\newcommand{\Z}{\mathbb{Z}}

\newcommand{\initE}{\setcounter{exo}{0}}
\newenvironment{exo}{\vspace{0.5cm}\setcounter{question}{0}\addtocounter{exo}{1} \noindent \textbf{Exercice \theexo}. \normalsize }{\par}

\begin{document}
	\maketitle
	
	\section*{Colle 1}
	
	Réda (14): petite erreur dans l'application du produit de Cauchy.\\
	Marouane (12): erreurs dans le produit de Cauchy. Assez bien mais confus.\\
	
	\begin{exo}
		Fubini.
	\end{exo}
	
	\begin{exo}
		Convergence et calcul de $\sum_{n=0}^{\infty} \sum_{k\geq n}^{\infty} \frac{1}{k!}$ (=2e)? 
	\end{exo}

\begin{exo}
	Convergence puis calcul de $\sum_{k=1}^{\infty} \frac{k}{2^k}$? (en utilisant produit de Cauchy et dérivée)
\end{exo}
%	\begin{exo}
%		Convergence de $\sum \frac{1}{{p_n}^{p_n}}$, où $p_n$ = nb de chiffres dans l'écriture décimale de $n$?
%	\end{exo}
	
	\section*{Colle 2}
	\setcounter{exo}{0}
	Etienne (16): bien\\
	Nathan (14): bien mais lent\\
	
	\begin{exo}
		Q et D dénombrable.
	\end{exo}

	\begin{exo}
		En utilisant le thm de sommation par paquet, déterminer pour quels $\alpha \in \mathbb{R}$ $(\frac{1}{(m+n)^\alpha})_{m, n}$ est sommable.
	\end{exo}
	
%	\begin{exo}
%		Adhérence et intérieur des matrices diagonalisables de $\mathcal{M}_n(\mathcal{C})$.
%	\end{exo}
	
	\section*{Colle 3}
	\setcounter{exo}{0}
	Julien (12): ne connaît pas bien la preuve. Inverse mal deux sommes. Dit que $\ln$ est décroissante (!!).\\
	Kévin (14): petit oubli dans la preuve. Oubli du thm de Fubini. Bien sinon.\\
	
	\begin{exo}
		Théorème de Cauchy.
	\end{exo}
	
	\begin{exo}
		\begin{enumerate}
			\item Si $\alpha >$ 1, trouver un équivalent de $\sum_{k=n}^\infty \frac{1}{k^\alpha}$.
			\item Pour quelles valeurs $\sum_{n=0}^\infty \sum_{k=n+1}^\infty \frac{1}{k^\alpha}$ converge?
			\item Montrer que $\sum_{n=0}^\infty \sum_{k=n}^\infty \frac{1}{k^\alpha} = 	\sum_{p\geq1} \frac{1}{p^{\alpha - 1}}$. 
		\end{enumerate}	
	\end{exo}

	\begin{exo}
		Equivalent de $\ln(n!)$ quand $n \longrightarrow \infty$?
	\end{exo}

\end{document}
\documentclass[10pt,a4paper]{article}
\usepackage[utf8]{inputenc}
\usepackage{amsmath}
\usepackage{amsfonts}
\usepackage{amssymb}
\usepackage{graphicx}
\usepackage[french]{babel}
\title{Colle MP: topologie}

\newcounter{question}
\newcommand{\initQ}{\setcounter{question}{0}}
\newenvironment{question}{\addtocounter{question}{1}
	\noindent {\it {Question} \thequestion.\ }}
{\par}
\newcounter{exo}
\newcommand{\Z}{\mathbb{Z}}

\newcommand{\initE}{\setcounter{exo}{0}}
\newenvironment{exo}{\vspace{0.5cm}\setcounter{question}{0}\addtocounter{exo}{1} \noindent \textbf{Exercice \theexo}. \normalsize }{\par}

\begin{document}
	\maketitle
	
	\section*{Colle 1}
	Stepan (11): ne connaît pas bien les définitions. veut toujours se ramener à la frontière pour une raison que j'ignore.\\
	GUIlLEMAUD Tom (13): ne se souvient pas bien de la définition d'un intérieur. Assez bien sinon.\\
	KORDYLAS Layla (11): ne connait pas bien la preuve (et la définition d'ouvert/fermé)\\
	Sarion Manon (12): ne connait pas bien sa preuve\\
	
	\begin{exo}
		Question cours
	\end{exo}

	\begin{exo}
		On suppose que A est une partie convexe d'un espace vectoriel normé E .
		(a) Montrer que $\bar{A}$ est convexe.
		(b) L'intérieur de A est-elle convexe ?
	\end{exo}
	
	\begin{exo}
		Mq l'intérieur d'un SEV est soit vide, soit l'espace entier.
	\end{exo}
	
	\section*{Colle 2}
	\setcounter{exo}{0}
	NADAL Julien (11): ne connaît pas bien la définition de norme equivalente \\
	BONNETAIN Baptiste (14):\\
	Robin DUCHENE (12): ne connait pas bien la preuve. mélange les définitions \\
	COMBES Perrine (12): a du mal avec la déf de normes équivalentes\\
	
	\begin{exo}
		Question cours
	\end{exo}

	\begin{exo}
		Est-ce que la norme 1, 2 et la norme sup sur $R^2$ sont équivalentes (et mq ce sont bien des normes)\\
		Est-ce que la norme 1, 2 et la norme sup sur les fct continues de 0, 1 dans R sont équivalentes (et mq ce sont bien des normes)
	\end{exo}
	
	\section*{Colle 3}
	\setcounter{exo}{0}
	Abdel (12): ne connaît pas bien la définition d'adherence\\
	AMRANE Paul (14): Bien\\
	DAUDEY Clément (14): Sérieux\\
	SCHWARTZBROD Luc (10): ne connait pas les definitions\\
	MISCHLER Anicia (15):\\
	
	\begin{exo}
		Question cours
	\end{exo}
	
	\begin{exo}
		La somme de 2 normes est-elle une norme?
	\end{exo}
	
	\begin{exo}
	Soient U et V deux ouverts denses d'un espace vectoriel normé E .
	(a) Établir que U $\cap$ V est encore un ouvert dense de E .\\
	(b) Est-ce toujours vrai si on enlève << ouvert >>?\\
	(c) En déduire que la réunion de deux fermés d'intérieurs vides est aussi
	d'intérieur vide.
	\end{exo}
	 
\end{document}
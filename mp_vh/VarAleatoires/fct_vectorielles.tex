\documentclass[10pt,a4paper]{article}
\usepackage{base}
\title{Colle MP 16: Fonctions vectorielles}

\newcounter{question}
\newcommand{\initQ}{\setcounter{question}{0}}
\newenvironment{question}{\addtocounter{question}{1}
	\noindent {\it {Question} \thequestion.\ }}
{\par}
\newcounter{exo}

\newcommand{\initE}{\setcounter{exo}{0}}
\newenvironment{exo}{\vspace{0.5cm}\setcounter{question}{0}\addtocounter{exo}{1} \noindent \textbf{Exercice \theexo}. \normalsize }{\par}

\begin{document}
	\maketitle
	
	\section*{Colle 1}
	Solène (16): très bien.\\
	PIERRE Alexandre (12): preuve incomplète\\
	
	\begin{exo}
		Propriétés fonctions génératrices.
	\end{exo}

	\begin{exo} (erreur dans la méthode des rectangles)
		Soit $f$ $C^1$ sur $[a, b]$. Soit $a_k = a + k\frac{b-a}{n}$.\\ Mq il existe une constante $M$ tq:
		$$\vert \int_{a}^{b} f - \frac{b-a}{n} \sum_{k=0}^{n-1} f(a_k) \vert \leq M \frac{(b-a)^2}{2n}$$
	\end{exo}
	
	\section*{Colle 2}
	\setcounter{exo}{0}
	MARTIN Manon (12): problèmes d'indices dans les sommes. dérive au lieu d'intégrer pour trouver une primitive.\\
	GAUBERT Baptiste (11): ne connaît pas du tout le théorème sur les sommes de Riemann.\\
	
	\begin{exo}
		Fonctions génératrices usuelles.
	\end{exo}

	\begin{exo}
		En faisant apparaître une somme de Riemann, donner un équivalent de $\sum_{k=0}^{n} \sqrt{k}$.
	\end{exo}
	
	\begin{exo}
		Donne un développement asymptotique à 2 termes de la suite des solutions de $\exp(x) + x = n$.
	\end{exo}

	\begin{exo}
		Soit $f$ une fonction $\mathcal{C}^2$ et $c$ tel que $f(c) = 0$, $f'(c) \neq 0$.\\
		Alors, si $u_0$ est assez proche de $c$, la suite $u_n$ obtenue par la méthode de Newton converge vers $c$.\\
		De plus il existe une constante $K$ telle que:
		$$\vert u_{n} - c \vert \leq K \vert u_{n-1} - c \vert^2$$
	\end{exo}
	
	\section*{Colle 3}
	\setcounter{exo}{0}
	Juliette (14): petite confusion entre $f$ et $f(x)$.\\
	Colin (14): inverse hypothèse et conclusion dans la démo de cours.\\
	
	\begin{exo}
		Thm limite de la dérivé.
	\end{exo}

	\begin{exo}
		Trouver la limite de:
		$$\sum_{k=1}^{n} \sin(\frac{k}{n}) \sin(\frac{k}{n^2}) ~~(= \int t \sin(t))$$
		Aide: $\sin(\frac{k}{n^2}) \approx \frac{k}{n^2}$.
	\end{exo}
	\begin{exo}
		Trouver un équivalent de la suite définie par:
		$$u_0 \in ]0, \pi[$$
		$$u_{n+1} = \sin(u_n)$$
		($u_n \sim \sqrt{\frac{6}{n}}$)
	\end{exo}

\end{document}
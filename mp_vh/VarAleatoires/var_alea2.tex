\documentclass[10pt,a4paper]{article}
\usepackage[utf8]{inputenc}
\usepackage{amsmath}
\usepackage{amsfonts}
\usepackage{amssymb}
\usepackage{graphicx}
\usepackage[french]{babel}

\title{Colle MP: Variables aléatoires et fonctions vectorielles}

\newcounter{question}
\newcommand{\initQ}{\setcounter{question}{0}}
\newenvironment{question}{\addtocounter{question}{1}
	\noindent {\it {Question} \thequestion.\ }}
{\par}
\newcounter{exo}
\newcommand{\Z}{\mathbb{Z}}

\newcommand{\initE}{\setcounter{exo}{0}}
\newenvironment{exo}{\vspace{0.5cm}\setcounter{question}{0}\addtocounter{exo}{1} \noindent \textbf{Exercice \theexo}. \normalsize }{\par}

\begin{document}
	\maketitle
	
	\section*{Colle 1}
	GUY Matthias (14): petite erreur sur les espaces de départs et d'arrivées (intervalle de R et non pas EV) des fonctions lors de la composition. Un peu lent sur l'exo.\\
	Audrey (16): très bien\\

	\begin{exo}
		Composée de fonctions dérivables.
	\end{exo}

	\begin{exo}
		En faisant apparaître une somme de Riemann, donner un équivalent de $\sum_{k=0}^{n} \sqrt{k}$.
	\end{exo}

	\begin{exo} (exo 15)
		\begin{enumerate}
			\item Mq $E(X) = \sum_{k=1}^\infty P(X \geq k)$
			\item Si $X$, $Y$ sont uniformes sur $\lbrace 1,..., n \rbrace$, quelle est l'espérance de $min(X, Y)$ et $max(X, Y)$?
		\end{enumerate}
	\end{exo}	
	
%	\begin{exo} (erreur dans la méthode des rectangles)
%		Soit $f$ $C^1$ sur $[a, b]$. Soit $a_k = a + k\frac{b-a}{n}$.\\ Mq il existe une constante $M$ tq:
%		$$\vert \int_{a}^{b} f - \frac{b-a}{n} \sum_{k=0}^{n-1} f(a_k) \vert \leq M \frac{(b-a)^2}{2n}$$
%	\end{exo}

	\section*{Colle 2}
	\setcounter{exo}{0}
	GUILLET Lucas (16): très bien\\
	Emma (14): ne reconnaît pas le développement de $-\ln(1-x)$. 
	\begin{exo}
		Inégalité de Bienaymé-Tchebychev
	\end{exo}
	
%	\begin{exo} (22)
%		Soit X et Y deux variables aléatoires indépendantes suivant des lois géométriques de paramètres $p, q \in ]0 ; 1[$.
%		Calculer $P(X < Y)$.
%	\end{exo}

	\begin{exo}
		Soit X une variable aléatoire suivant une loi géométrique de paramètre p. Calculer $E(\frac{1}{X})$.
	\end{exo}
	
	\begin{exo}
		Trouver la limite de:
		$$\sum_{k=1}^{n} \sin(\frac{k}{n}) \sin(\frac{k}{n^2}) ~~(= \int t \sin(t))$$
		Aide: $\sin(\frac{k}{n^2}) \approx \frac{k}{n^2}$.
	\end{exo}

	\section*{Colle 3}
	\setcounter{exo}{0}
	Pierre (13): confusion dans la loi de Poisson\\
	Lisa (13): ne pense pas à utiliser la formule de dérivation des fonctions composées\\

	\begin{exo}
		Inégalité de Cauchy-Schwarz
	\end{exo}

	\begin{exo}
		Soit $X$ loi géométrique de paramètre $1/n$. Mq $P(X \geq n^2) \leq \frac{1}{n}$ puis $P(\vert X - n \vert \geq n) \leq 1 - \frac{1}{n}$.
	\end{exo}

	\begin{exo}
		Mq la somme de 2 variables de Poisson indépendantes est une variable de Poisson.
	\end{exo}

	\begin{exo}
	Soit I un intervalle, E un espace vectoriel euclidien et $f:I \longmapsto E$ dérivable. On suppose de plus que $f$ ne s'annule pas et on pose, pour tout $t\in I$, $g(t)= \parallel f(t) \parallel$. Démontrer que g est dérivable et donner g'.
	\end{exo}

%	\begin{exo}
%		Une urne contient 4 boules rapportant 0, 1, 1, 2 points. On y effectue n tirages avec remise et l'on note S le score total obtenu.\\
%		Déterminer la fonction génératrice de S et en déduire la loi de S .
%	\end{exo}
%	
\end{document}
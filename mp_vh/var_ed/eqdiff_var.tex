\documentclass[10pt,a4paper]{article}
\usepackage{base}
\title{Colle MP: Équations différentielles et intégration}

\newcounter{question}
\newcommand{\initQ}{\setcounter{question}{0}}
\newenvironment{question}{\addtocounter{question}{1}
	\noindent {\it {Question} \thequestion.\ }}
{\par}
\newcounter{exo}

\newcommand{\initE}{\setcounter{exo}{0}}
\newenvironment{exo}{\vspace{0.5cm}\setcounter{question}{0}\addtocounter{exo}{1} \noindent \textbf{Exercice \theexo}. \normalsize }{\par}

\begin{document}
	\maketitle
	
	\section*{Colle 1}	
	Lilian (14): erreurs de calcul. Pense qu'il faut raccorder les solutions particulières (au lieu des solutions complètes)\\
	Célia (13): démonstration de cours en partie oubliée\\

	\begin{exo}
		Variance de la somme de VA
	\end{exo}

	\begin{exo} (22)
		Soit X et Y deux variables aléatoires indépendantes suivant des lois géométriques de paramètres $p, q \in ]0 ; 1[$.
		Calculer $P(X \leq Y)$.
	\end{exo}
	
	\begin{exo}
		$ty' + 2y = \frac{t}{t^2+1}$ (avec raccords).
	\end{exo}
			
	\section*{Colle 2}
	\setcounter{exo}{0}
	Bastien (14): a du mal à retrouver le développement en série entière de $\frac{1}{(1-t)^2}$\\
	Charlotte (13): a du mal à retrouver le développement en série entière de $\frac{1}{(1-t)^2}$\\

	\begin{exo}
		Question de cours
	\end{exo}

	\begin{exo}
		Soient $X$ et $Y$ deux lois géométriques indépendantes de paramètres $p$.\\
		Donner la loi de $X + Y$ (2 méthodes: avec ou sans série génératrice).
	\end{exo}
	
	\begin{exo}
%		$(t^2 + 1)y'' + 2ty + 1 = 0$
		Résoudre $(1 - t)y' - y = t$ (avec raccords).
	\end{exo}
	\newpage

	\section*{Colle 3}
	\setcounter{exo}{0}
	Kévin (17): oubli de solution dans l'ED. Très bien sinon.\\
	Nolwenn (14): oubli de justifier le fait qu'une fonction génératrice caractérise la loi \\

	\begin{exo}
	Cours: Si X et Y sont indépendantes, alors $f ( X )$ et $g ( Y )$ sont indépendantes.
	\end{exo}

	\begin{exo}
%		(83) Montrer que $\int_{x}^{\infty} \frac{e^{-t}}{t} \sim \frac{e^{-x}}{x}$.
		Calculer $G_{X + Y}$ où $X$ et $Y$ sont deux VA de Poisson indépendantes. En déduire la loi de $X + Y$.
	\end{exo}

	\begin{exo}
		$yy' + y^2 = \frac{exp(-2x)}{2}$ (poser $z = y^2$)
	\end{exo}
	
	\begin{exo}
		Problème du coupon collector.
	\end{exo}

\end{document}
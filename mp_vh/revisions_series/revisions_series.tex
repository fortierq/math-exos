\documentclass[10pt,a4paper]{article}
\usepackage{base}
\title{Colle MP 14: révisions}

\newcounter{question}
\newcommand{\initQ}{\setcounter{question}{0}}
\newenvironment{question}{\addtocounter{question}{1}
	\noindent {\it {Question} \thequestion.\ }}
{\par}
\newcounter{exo}

\newcommand{\initE}{\setcounter{exo}{0}}
\newenvironment{exo}{\vspace{0.5cm}\setcounter{question}{0}\addtocounter{exo}{1} \noindent \textbf{Exercice \theexo}. \normalsize }{\par}

\begin{document}
	\maketitle
	
	\section*{Colle 1}
	Théo CALLEY (11): des difficultés\\
	BOUHELIER Julien (14): quelques confusions sur le rayon de CV et inégalités entre complexes \\
	
	\begin{exo}
		Quels sont les $z$ pour lesquels $\sum \frac{z^n}{n}$ converge?
	\end{exo}
	
	\begin{exo}
		Pour $a, b \in \mathbb{R}$, valeur minimale de l'intégrale:
		$$\int_{0}^{\pi} (t - a \cos(t) - b \sin(t))^2 dt$$ 
	\end{exo}
	
	\section*{Colle 2}
	\setcounter{exo}{0}
	CHARRIERE Baptiste (8): ne se souvient pas de Cauchy-Schwarz. Très faible.\\
	MARGUIER Agathe (13): correct.\\
	
	\begin{exo}
		Soit $a_0 = 1$ et $a_{n+1} = \sum_{k=0}^{n} a_k a_{n-k}$.
		\begin{itemize}
			\item Calculer $S(x) = \sum_{k=0}^{\infty} a_k x^k$.
			\item En déduire $a_n$, $\forall n$.
		\end{itemize}
	\end{exo}

	\begin{exo}
		Soit $f : ]0, \infty[ \longrightarrow \mathbb{R}$, $f(x) = \sum_{k=1}^{\infty}	\frac{1}{sh(kx)}$.\\
		Donner un équivalent de $f$ en $\infty$.
	\end{exo}

	\section*{Colle 3}
	\setcounter{exo}{0}
	DESHAYES Pierre (15): Bien\\
	
	\begin{exo}
		DL de la série harmonique, en posant $v_n = H_n - \ln(n)$.	
	\end{exo}

	\begin{exo}
		Soit $A \in O_n(\mathbb{R})$. Mq:
		$$\vert \sum a_{i,j} \vert \leq n$$ 
		Indice: utiliser vecteur avec que des 1.
	\end{exo}
	
	\begin{exo}
		Soit $M \in O_n(\mathbb{R})$. Mq:
		$$\sum \vert m_{i,j} \vert \leq n \sqrt{n}$$ 
		Indice: utiliser $(A, B) \longmapsto tr({}^t A B)$.
	\end{exo}
	
\end{document}
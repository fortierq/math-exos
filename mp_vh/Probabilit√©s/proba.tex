\documentclass[10pt,a4paper]{article}
\usepackage{base}
\title{Colle MP: Probabilités}

\newcounter{question}
\newcommand{\initQ}{\setcounter{question}{0}}
\newenvironment{question}{\addtocounter{question}{1}
	\noindent {\it {Question} \thequestion.\ }}
{\par}
\newcounter{exo}

\newcommand{\initE}{\setcounter{exo}{0}}
\newenvironment{exo}{\vspace{0.5cm}\setcounter{question}{0}\addtocounter{exo}{1} \noindent \textbf{Exercice \theexo}. \normalsize }{\par}

\begin{document}
	\maketitle
	
	\section*{Colle 1}
	Théophane DUCRET (14): bien mais ne se souvient pas de la formule de Pascal\\
	Layla (15): bien\\
	
	_collison 240 math crypto
	\begin{exo}
		Demo cours
	\end{exo}

	\begin{exo}
		Mq (par récurrence) il existe $\binom{n+p-1}{n}$ $p$-uplets de $\mathbb{N}^p$ dont la somme est égal est à $n$.
	\end{exo}
	
	\begin{exo}
		Combien y a-t-il d'applications strictement croissantes de $\s{1, ..., n}$ vers $\s{1, ..., p}$ ?
	\end{exo}
	
	\section*{Colle 2}
	\setcounter{exo}{0}
	Guichon Joannes (12): confusion sur l'union/intersection\\
	Laura (11): ne connaît pas bien la définition d'une tribu. Écrit $P(A) \cup P(B)$.\\
	
	\begin{exo}
	   Demo cours
	\end{exo}

	\begin{exo}
		%T = $\lbrace A \subset  E A ou \bar{A}$ est dénombrable $\rbrace$
		Montrer que T = $\s{ A \subset \Omega \tq A ~ou~ \bar{A}~ est~ denombrable}$ est une tribu sur $\Omega$.
	\end{exo}
		
	\section*{Colle 3}
	\setcounter{exo}{0}
	Pauline (11): confusion entre $\cup$ et +. Considère deux ensembles $A$, $B$ pour montrer la stabilité par complémentaire (et regarde $\bar{A \cup B}$).\\
	Jeanne (9): ne se souvient pas du tout de la définition d'une tribu\\
	Ysaline (14): bien\\
	
	\begin{exo}
		Demo cours
	\end{exo}

	\begin{exo}
		Mq l'intersection de deux tribus est une tribu.
	\end{exo}
	
	\begin{exo}
		\begin{enumerate}
			\item On tire uniformément au hasard un sous-ensemble de $\lbrace 1, ..., n \rbrace$. Quelle est la probabilité qu'il soit de taille pair? 
			\item On tire uniformément au hasard deux sous-ensembles $A$ et $B$ de $\lbrace 1, ..., n \rbrace$. Quelle est la probabilité que $A \subseteq B$? Que $A \cap B = \emptyset$?
		\end{enumerate}
	\end{exo}
	
	\begin{exo}
		(Coupon collector)\\
		Un enfant collectionne des cartes. Il y a $n$ cartes et l'enfant les veut tous. A chaque achat d'une carte, il obtient une carte uniformément au hasard parmi les $n$ possibles. Combien d'achats devra t-il faire pour avoir toutes les cartes? (réponse: $\sim n \ln(n)$).\\
		Indice: Si l'enfant a $k$ cartes, quelle est la proba qu'il doive faire $p$ achats pour obtenir une nouvelle carte? (réponse: $\frac{k}{n}^{p-1} (1 - \frac{k}{n}$))\\
		Application numérique: estimer le nombre d'achats nécessaires pour avoir les 150 cartes de Pokémon.
	\end{exo}

\end{document}
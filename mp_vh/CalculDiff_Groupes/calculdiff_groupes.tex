\documentclass[10pt,a4paper]{article}
\usepackage{base}

\title{Colle MP 21: Calcul différentielle}

\newcounter{question}
\newcommand{\initQ}{\setcounter{question}{0}}
\newenvironment{question}{\addtocounter{question}{1}
	\noindent {\it {Question} \thequestion.\ }}
{\par}
\newcounter{exo}

\newcommand{\initE}{\setcounter{exo}{0}}
\newenvironment{exo}{\vspace{0.5cm}\setcounter{question}{0}\addtocounter{exo}{1} \noindent \textbf{Exercice \theexo}. \normalsize }{\par}

\begin{document}
	\maketitle
	
	\section*{Colle 1}	
	Paul MAUBLANC (13): ne connaît pas l'algorithme pour trouver les coeff de Bézout\\
	Eloise (16): très bien, très clair\\
	
	\begin{exo}
		Théorème de représentation des formes linéaires.
	\end{exo}

	\begin{exo}
		Est-ce que 18 est inversible dans $(\mathbb{Z}/ 49\mathbb{Z}, \times)$? Donner son inverse.
	\end{exo}
	
	\begin{exo}
		Extrema de $f(x, y) = x^2 + xy + y^2 + 2x - 2y$
	\end{exo}	
			
	\section*{Colle 2}
	\setcounter{exo}{0}
	Grégoire (14): bien pour la démo. ne connaît pas bien l'algorithme pour trouver les coeff de Bézout\\
	Julie (14): bien mais un peu hésitante\\
	
	\begin{exo}
		Cours: que peut-on dire d'un extremum local d'une fonction différentiable? (démo)
	\end{exo}
	\begin{exo}
		Quel est le chiffre des unités de $3^{2019}$?
	\end{exo}
	
%	\begin{exo}
%		Trouver tous les $x \in \mathbb{N}$ vérifiant:
%		$$x \equiv 2 [12]$$
%		$$x \equiv 7 [25]$$
%	\end{exo}
	
	\begin{exo}
		Recherche d'extremum.
	\end{exo}	
	
	\section*{Colle 3}
	\setcounter{exo}{0}
	Rémi (12): ne connaît pas très bien le cours.\\
	Lise GARDAVAUD (15): bien\\
	
	\begin{exo}
		Cours
	\end{exo}

	\begin{exo}
		Trouver les solutions de $7x = 2$ mod $37$.
	\end{exo}	

	\begin{exo}
		Extremum locaux de $x^3 + y^3$?
	\end{exo}	
	
	\begin{exo}
		Résoudre:
		$$3x+4y \equiv 5 [13]$$
		$$2x+5y \equiv 7 [13]$$
		($y=9 [13]$, $x=7 [13]$)
	\end{exo}	

\end{document}
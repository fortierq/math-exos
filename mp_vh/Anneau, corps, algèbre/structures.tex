\documentclass[10pt,a4paper]{article}
\usepackage[utf8]{inputenc}
\usepackage{amsmath}
\usepackage{amsfonts}
\usepackage{amssymb}
\usepackage{graphicx}
\title{Colle MP 2: Structures et algèbre linéaire}

\newcounter{question}
\newcommand{\initQ}{\setcounter{question}{0}}
\newenvironment{question}{\addtocounter{question}{1}
	\noindent {\it {Question} \thequestion.\ }}
{\par}
\newcounter{exo}
\newcommand{\Z}{\mathbb{Z}}

\newcommand{\initE}{\setcounter{exo}{0}}
\newenvironment{exo}{\vspace{0.5cm}\setcounter{question}{0}\addtocounter{exo}{1} \noindent \textbf{Exercice \theexo}. \normalsize }{\par}

\begin{document}
	\maketitle
	
	\section*{Colle 1}
	CALLEY Théo (13): assez bien\\
	Julien (12): ne se souvient plus des propriétés des congruences. \\
	
	\begin{exo}
		Intersection et somme d’idéaux
	\end{exo}

	\begin{exo}
		Chiffre des unités de $7^{222}$?
	\end{exo}

	\begin{exo}
		Nombre de carrés dans $Z/pZ$? Indice: utiliser $\phi : x \longmapsto x^2$.
	\end{exo}
		
	\section*{Colle 2}
	\setcounter{exo}{0}
	SAGET Geoffray (14): bien sauf quand il dit que tr(AB)=tr(A)tr(B)\\
	MARGUIER Agathe (15): bien\\
	
	\begin{exo}
		Deux matrices semblables ont même déterminant et même trace. La réciproque est-elle vraie?
	\end{exo}
	
	\begin{exo}
		Factoriser dans C[X] puis R[X] le polynome $X^{2n} - 2\cos(m)X^n + 1$
	\end{exo}

	\begin{exo}
		Soit $p \geq 4$ premier. Montrer que $24 \vert p^2 - 1$.
	\end{exo}

	\section*{Colle 3}
	\setcounter{exo}{0}
	CHARRIERE Baptiste (12): correct\\
	DESHAYES Pierre (11): ne connaît pas bien sa preuve et démonstration peu clair dans ses propos\\
	
	\begin{exo}
		Idéaux de K[X]?
	\end{exo}
	
	\begin{exo}
		Mq si $p$ premier alors $p \vert \binom{p}{k}$, puis $n^p = n$ [$p$]
	\end{exo}

	\begin{exo}
		Résoudre $x \equiv 2 [5]$, $x \equiv 3 [7]$.
	\end{exo}
	
	\begin{exo}
		Expression de $\phi(n)$ avec décomposition facteurs premiers?
	\end{exo}
	 
	\section*{Colle 4}
	LE TETU Elise (15): bien\\
	
	\begin{exo}
		Noyau d'un morphisme d'anneaux?
	\end{exo}
	
	\begin{exo}
		Formule pour l'ordre de $k$ dans Z \ n Z?
	\end{exo}
	
	\begin{exo}
		Soit $p$ un entier. Montrer que $p$ est premier ssi:
		$$(p-1)! \equiv -1 [p]$$
	\end{exo}

\end{document}
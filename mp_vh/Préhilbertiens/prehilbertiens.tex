\documentclass[10pt,a4paper]{article}
\usepackage{base}
\title{Colle MP 11: Espaces préhilbertiens}

\newcounter{question}
\newcommand{\initQ}{\setcounter{question}{0}}
\newenvironment{question}{\addtocounter{question}{1}
	\noindent {\it {Question} \thequestion.\ }}
{\par}
\newcounter{exo}

\newcommand{\initE}{\setcounter{exo}{0}}
\newenvironment{exo}{\vspace{0.5cm}\setcounter{question}{0}\addtocounter{exo}{1} \noindent \textbf{Exercice \theexo}. \normalsize }{\par}

\begin{document}
	\maketitle
	
	\section*{Colle 1}
	Célia (15): bonne connaissance du cours et des méthodes \\
	Kévin (16): Très bien.\\	

	
	\begin{exo}
		Thm spectral
	\end{exo}

	\begin{exo}
		Soit u un endomorphisme symétrique d'un espace euclidien E vérifiant, pour tout $x \in E$, $⟨u(x),x⟩=0$. Mq $u = 0$.
	\end{exo}

	\begin{exo}
		Soit $u:E \longmapsto$ E tel que, pour tous $x,y \in E$, on a $⟨u(x),y⟩=⟨x,u(y)⟩$. Démontrer que $u$ est linéaire.

	\end{exo}
%	\begin{exo}
%		Pour $a, b \in \mathbb{R}$, valeur minimale de l'intégrale:
%		$$\int_{0}^{\pi} (t - a \cos(t) - b \sin(t))^2 dt$$ 
%	\end{exo}
		
	\section*{Colle 2}
	\setcounter{exo}{0}
	Nolwenn (13): oubli du théorème spectral.\\
	Lilian (13): assez bien

	\begin{exo}
		Caract endo symétrique par sa matrice
	\end{exo}

%	\begin{exo}
%		Donner une base orthonormale de $\mathbb{R_2[X]}$ pour $(P \vert Q) = \int_{-1}^1 P(t)Q(t) dt$.
%	\end{exo}
%	
%	\begin{exo}
%		(57) Mq $S_n$ et $A_n$ sont supplémentaire orthogonaux pour le prod canonique. Distance d'une matrice ... à $S_3$?
%	\end{exo}

	\begin{exo}
		Soit $A \in M_n(\mathbb{R})$ symétrique. On suppose qu'il existe $p \in \mathbb{N}$ tel que $A^p=0$. Mq $A = 0$.
	\end{exo}

	\begin{exo}
		Soient u,v deux endomorphismes symétrique d'un espace euclidien qui commutent.
 Soit $\lambda$ une valeur propre de u. On pose $F=ker(u−\lambda id)$. Démontrer que $F$ et $F^\bot$ sont stables par v.\\
 Puis démontrer qu'il existe une base orthonormale de E diagonalisant simultanément u et v.
	\end{exo}

	\begin{exo}
		(39) Soit $f$ endo d'un espace euclidien tq $(f(x) \vert x) = 0$, $\forall x$. Mq $Ker f = Im f ^\perp$.
	\end{exo}
	
	\section*{Colle 3}
	\setcounter{exo}{0}
	Charlotte (11): grosses erreurs dans le polynôme caractéristique. Erreurs de calculs. \\
	Bastien (14): Bien.

	\begin{exo}
		Pythagore
	\end{exo}

	\begin{exo}
		Soit E un espace vectoriel euclidien. Pour $f \in L(E)$, on note $\rho(f) = max \lbrace \vert \lambda \vert; \lambda \textnormal{ valeur propre de f }\rbrace$. \\
		On pose également $\parallel f \parallel=\sup \lbrace \parallel f(x) \parallel; \parallel x \parallel \leq 1 \rbrace$. Démontrer que si f est symétrique, alors $\parallel f \parallel=\rho(f)$.
	\end{exo}

	\begin{exo}
		Soit $E = C ([−1, 1] , R)$ muni du produit scalaire $\int_{-1}^{1} fg$.\\
		Quel est l'orthogonal des fonctions nulles sur $[-1, 0]$? Sont-ils supplémentaires?
	\end{exo}

%	\begin{exo}
%		Mq $O_n(\mathbb{R})$ est un compact non connexe par arc. Quelles sont ses composantes connexes par arcs?
%	\end{exo}
%	
%	\begin{exo}
%		Soit $A \in O_n(\mathbb{R})$. Mq:
%		$$\vert \sum a_{i,j} \vert \leq n$$ 
%		Indice: utiliser vecteur avec que des 1.
%	\end{exo}
%	
%	\begin{exo}
%		Soit $M \in O_n(\mathbb{R})$. Mq:
%		$$\sum \vert m_{i,j} \vert \leq n \sqrt{n}$$ 
%		Indice: utiliser $(A, B) \longmapsto tr({}^t A B)$.
%	\end{exo}
%	
%	\begin{exo}
%		Mq si $(x \vert y) = 0$ $\implies$ $(f(x) \vert f(y)) = 0$ alors $\exists \lambda$, $\parallel f(x) \parallel = \lambda \parallel x \parallel$.
%	\end{exo}

\end{document}
\documentclass[10pt,a4paper]{article}
\usepackage{base}
\title{Colle MP 17: Intégration}

\newcounter{question}
\newcommand{\initQ}{\setcounter{question}{0}}
\newenvironment{question}{\addtocounter{question}{1}
	\noindent {\it {Question} \thequestion.\ }}
{\par}
\newcounter{exo}

\newcommand{\initE}{\setcounter{exo}{0}}
\newenvironment{exo}{\vspace{0.5cm}\setcounter{question}{0}\addtocounter{exo}{1} \noindent \textbf{Exercice \theexo}. \normalsize }{\par}

\begin{document}
	\maketitle
	
	\section*{Colle 1}
	Ysaline (15): petit oubli de la comparaison série-intégrale.\\
	Théophane DUCRET (13): oubli des méthodes de DL. écrit $\ln(ab) = \ln(a)\ln(b)$.\\
	
	\begin{exo}
		Cours
	\end{exo}	

	\begin{exo}
		$$\int_{0}^{\frac{\pi}{2}} \ln(\sin(x))dx?$$
	\end{exo}
	%\begin{exo}
	%	Trouver un équivalent en $\infty$ de $F(x) = \int_{0}^{x} e^{\frac{t^2}{2}}dt$.\\
	%	Méthode de "correction de la dérivée".
	%\end{exo}
	
	\section*{Colle 2}
	\setcounter{exo}{0}
	Laura (12): erreur dans les DL (de $\sin$, de $\ln$)\\
	Joannès (16): dit que $f$ intégrable $\implies$ $f \longrightarrow 0$, quel que soit $f$. Bien sinon.\\
	
	\begin{exo}
		Cours
	\end{exo}

	\begin{exo}
		Calcul d'intégrales.
	\end{exo}
	\begin{exo}
		(14) Soit $f$ continue intégrable sur $\R$. Est-ce que $f \longrightarrow 0$? Et si $f$ décroissante? Et $xf(x)$?
	\end{exo}

	\section*{Colle 3}
	\setcounter{exo}{0}
	Jeanne (11): oubli dans la méthode de décomposition en éléments simples. \\
	Pauline (11): écrit $\frac{1}{a+b} = \frac{1}{a} + \frac{1}{b}$...oubli dans la méthode de décomposition en éléments simples.\\
	
	\begin{exo}
		Cours
	\end{exo}

	\begin{exo}
		Calcul d'intégrales.
	\end{exo}
		
	\begin{exo}
		Etudier l'intégrabilité de $\int_{0}^{\infty} {\sin(x)}$, $\int_{0}^{\infty} \frac{\c{\sin(x)}}{x}$, $\int_{0}^{\infty} \frac{\sin(x)}{x}$.
	\end{exo}

\end{document}
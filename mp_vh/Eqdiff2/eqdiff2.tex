\documentclass[10pt,a4paper]{article}
\usepackage[utf8]{inputenc}
\usepackage{base}
\usepackage[french]{babel}
\title{Colle MP 20: Équations différentielles et calcul différentiel}

\newcounter{question}
\newcommand{\initQ}{\setcounter{question}{0}}
\newenvironment{question}{\addtocounter{question}{1}
	\noindent {\it {Question} \thequestion.\ }}
{\par}
\newcounter{exo}

\newcommand{\initE}{\setcounter{exo}{0}}
\newenvironment{exo}{\vspace{0.5cm}\setcounter{question}{0}\addtocounter{exo}{1} \noindent \textbf{Exercice \theexo}. \normalsize }{\par}

\begin{document}
	\maketitle
	
	\section*{Colle 1}	
	Pierre (11: énoncé du thm de cours non su. ne pense pas à utiliser la superposition des solutions.\\
	Calley (13): assez bien.
	
	\begin{exo}
		cours
	\end{exo}

	\begin{exo}
		équa diff ordre 2
	\end{exo}

	\begin{exo}
		$y'(t) + y(-t) = e^t$ (indice: se ramener à une ED d'ordre 2 classique)
	\end{exo}			
	\section*{Colle 2}
	\setcounter{exo}{0}
	MARGUIER Agathe (14): question de cours bien. bien pour les calculs, mais un peu lente.\\
	Baptiste (13): assez bien mais un peu lent.\\
	
	\begin{exo}
		cours
	\end{exo}

	\begin{exo}
		équa diff ordre 2
	\end{exo}
	
	\begin{exo}
		Résolution d'équa diff avec série entière
	\end{exo}
			
	\section*{Colle 3}
	\setcounter{exo}{0}
	BOUHELIER Julien (15): Oublie que les matrices doivent commuter pour utiliser le binôme de Newton). Sinon bien.\\
	Léo (14): question de cours bien. Essaie des choses trop compliquées pour trouver la différentielle de $M \longmapsto M^2$.\\
	
	\begin{exo}
		cours
	\end{exo}

	\begin{exo}
		équa diff ordre 2
	\end{exo}
	\begin{exo}
		Différentielle de $M \longmapsto M^2$?
	\end{exo}

	\section*{Colle 4 (rattrapage)}
	\setcounter{exo}{0}
	Stepan (14): bien\\
	
	\begin{exo}
		Soit X une variable aléatoire suivant une loi géométrique de paramètre p. Calculer $E(\frac{1}{X})$.
	\end{exo}
	
	\begin{exo}
		Mq la somme de 2 variables de Poisson indépendantes est une variable de Poisson.
	\end{exo}

\end{document}
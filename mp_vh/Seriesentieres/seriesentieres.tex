\documentclass[10pt,a4paper]{article}
\usepackage[utf8]{inputenc}
\usepackage{amsmath}
\usepackage{amsfonts}
\usepackage{amssymb}
\usepackage{graphicx}
\usepackage[french]{babel}
\title{Colle MP 12: Séries entières}

\newcounter{question}
\newcommand{\initQ}{\setcounter{question}{0}}
\newenvironment{question}{\addtocounter{question}{1}
	\noindent {\it {Question} \thequestion.\ }}
{\par}
\newcounter{exo}
\newcommand{\Z}{\mathbb{Z}}

\newcommand{\initE}{\setcounter{exo}{0}}
\newenvironment{exo}{\vspace{0.5cm}\setcounter{question}{0}\addtocounter{exo}{1} \noindent \textbf{Exercice \theexo}. \normalsize }{\par}

\begin{document}
\maketitle
\verb|https://www.youtube.com/watch?v=bOXCLR3Wric&t=1239s|
	
	\section*{Colle 1}
	Jawed (14): Assez bien.\\
	Laura (15): Bien. Se souvient de la démo de $\sin(n) \not\longrightarrow 0$.\\
		
	\begin{exo}
		Lemme d'Abel.
	\end{exo}

%	\begin{exo}
%		Rayon et calcul de $\sum_{n=1}^{\infty} (1 + \frac{1}{2} + ... + \frac{1}{n}) t^n$ ($= \frac{\ln(1 - t)}{t - 1})$.
%	\end{exo}

	\begin{exo}
		Rayon de CV de $\sum \sin(n)z^n$?
	\end{exo}
	\begin{exo}
		%DSE de $\frac{1}{(1-z)^{k+1}}$?
		Soit $\sum a_n z^n$ série entière de rayon R. Quel est le rayon de CV de $\sum a_n^2 z^n$?
	\end{exo}
			
	\begin{exo}
		Mq si une série entière est nulle sur un voisinage de 0, elle est identiquement nulle.
	\end{exo}
	\section*{Colle 2}
	\setcounter{exo}{0}
	Achille (12): confusions et erreurs multiples dans la formule du produit de Cauchy\\
	
	\begin{exo}
		Règle de D'Alembert.
	\end{exo}
	
	\begin{exo}
		DSE et rayon de CV de $\ln(1 + x - 2x^2)$?
	\end{exo}
	\begin{exo}
		Rayon et calcul de $\sum \frac{x^n}{n(n+1)(n+2)}$.
	\end{exo}

	\begin{exo}
		Soit $a_0 = 1$ et $a_{n+1} = \sum_{k=0}^{n} a_k a_{n-k}$.
		\begin{itemize}
			\item Calculer $S(x) = \sum_{k=0}^{\infty} a_k x^k$, en mq $xS(x)^2 = S(x) - 1$.
			\item En déduire $a_n$, $\forall n$.
		\end{itemize}
	\end{exo}

	\begin{exo}
		Montrer que si $\alpha \in \mathbb{R}$, $f(t) =\cos(\alpha \arcsin(t))$ est développable en série entière sur un voisinage de 0 et trouver son rayon de convergence.\\
		Aide: montrer que $f$ vérifie l'ED $(1 - t^2) y'' - t y' + \alpha^2 y = 0$.
	\end{exo}
	
	\section*{Colle 3}
	\setcounter{exo}{0}
	Julien (15): Petite erreur de calcul. Bien sinon.\\
	
	\begin{exo}
		Règle de D'Alembert.
	\end{exo}
	
	\begin{exo}
		DSE et rayon de CV de $\ln(1 + x - 2x^2)$?
	\end{exo}
	\begin{exo}
		Rayon et calcul de $\sum \frac{x^n}{n(n+1)(n+2)}$.
	\end{exo}

	\begin{exo}
		Soit $a_0 = 1$ et $a_{n+1} = \sum_{k=0}^{n} a_k a_{n-k}$.
		\begin{itemize}
			\item Calculer $S(x) = \sum_{k=0}^{\infty} a_k x^k$, en mq $xS(x)^2 = S(x) - 1$.
			\item En déduire $a_n$, $\forall n$.
		\end{itemize}
	\end{exo}

	\begin{exo}
		Montrer que si $\alpha \in \mathbb{R}$, $f(t) =\cos(\alpha \arcsin(t))$ est développable en série entière sur un voisinage de 0 et trouver son rayon de convergence.\\
		Aide: montrer que $f$ vérifie l'ED $(1 - t^2) y'' - t y' + \alpha^2 y = 0$.
	\end{exo}
		
	\section*{Colle 4}
	\setcounter{exo}{0}
	Alex (10): ne connaît pas bien la démo. Manque de pratique.\\
	Pauline (14): ne pense pas à utiliser la propriété fondamental de $\ln$. Un peu lent dans les calculs mais bien sinon.\\

	\begin{exo}
		Règle de d'Alembert.
	\end{exo}
	
	\begin{exo}
		DSE de $\ln(1 + x -2x^2)$?.
	\end{exo}

	\begin{exo}
		Quels sont les $z$ pour lesquels $\sum \frac{z^n}{n}$ converge?
	\end{exo}

	\begin{exo}
		Soit $B_n$ le nombre de partitions de $\lbrace 1, 2, ..., n \rbrace$.
		\begin{enumerate}
			\item Calculer $B_1$, $B_2$, $B_3$.
			\item Mq $$B_{n+1} = \sum_{k=0}^{n} \binom{n}{k} B_k$$
			\item Soit $S_n = \frac{1}{e} \sum_{k=0}^{\infty} \frac{k^n}{k!}$. Mq $B_n = S_n$.
			\item Montrer que $\sum \frac{B_n}{n!} x^n$ a un rayon de CV infini, et calculer sa somme.
		\end{enumerate}
	\end{exo}

\end{document}
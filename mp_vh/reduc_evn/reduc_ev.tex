\documentclass[10pt,a4paper]{article}
\usepackage[utf8]{inputenc}
\usepackage{amsmath}
\usepackage{amsfonts}
\usepackage{amssymb}
\usepackage{graphicx}
\title{Colle MP: réduction et EVN}

\newcounter{question}
\newcommand{\initQ}{\setcounter{question}{0}}
\newenvironment{question}{\addtocounter{question}{1}
	\noindent {\it {Question} \thequestion.\ }}
{\par}
\newcounter{exo}
\newcommand{\Z}{\mathbb{Z}}

\newcommand{\initE}{\setcounter{exo}{0}}
\newenvironment{exo}{\vspace{0.5cm}\setcounter{question}{0}\addtocounter{exo}{1} \noindent \textbf{Exercice \theexo}. \normalsize }{\par}

\begin{document}
	\maketitle
	
	\section*{Colle 1}
	\setcounter{exo}{0}
	Eloise (11): Un peu trop hésitante. Quelques confusions sur les valeurs/vecteur propres, élément/ensemble...\\
	Mael (16): Très bien.\\
	
	
	\begin{exo}
		Question de cours. Une intersection infinie d'ouverts est-elle forcément ouverte?
	\end{exo}
	
	\begin{exo} %(278)
		Mq $u$ est nilpotent ssi $Sp(u) = \lbrace 0 \rbrace$.\\
		Mq $u$ est nilpotent ssi $Tr(u^k) = 0$ , $\forall 1 \leq k \leq n$.
	\end{exo}
	
	\section*{Colle 2}
	Bérangère (15): Bon travail.\\
	Michael (13): ne se souvient pas de la formule d'inversion d'une matrice 2x2. Bien sinon.\\
	
	\begin{exo}
		Question de cours.
	\end{exo}
	
	\begin{exo}
		Trouver les suites vérifiant:
		$$u_{n+1} = u_n - v_n$$
		$$v_{n+1} = 2u_n + 4v_n$$
	\end{exo}

	\section*{Colle 3}
	\setcounter{exo}{0}
	Maeva (15): Confusion sur ce qu'est un polynôme scindé à racine simple.\\
	Meriton (14): Confusion sur ce qu'est un polynôme scindé à racine simple. Un peu lent.\\
	
	\begin{exo}
		Question de cours.
	\end{exo}

	\begin{exo} %(268)
		Soit $A \in GL_n(\mathbb{C})$ et $p \in N^*$. \\
		Mq $A$ diagonalisable $\Longleftrightarrow$ $A^p$ diagonalisable.\\
		Est-ce que ce résultat reste vrai si $A \notin GL_n(\mathbb{C})$?
	\end{exo}
	 
\end{document}
\documentclass[10pt,a4paper]{article}
\usepackage{base}

\title{Colle MP 21: Calcul différentielle}

\newcounter{question}
\newcommand{\initQ}{\setcounter{question}{0}}
\newenvironment{question}{\addtocounter{question}{1}
	\noindent {\it {Question} \thequestion.\ }}
{\par}
\newcounter{exo}

\newcommand{\initE}{\setcounter{exo}{0}}
\newenvironment{exo}{\vspace{0.5cm}\setcounter{question}{0}\addtocounter{exo}{1} \noindent \textbf{Exercice \theexo}. \normalsize }{\par}

\begin{document}
	\maketitle
	
	\section*{Colle 1}	
	Mathilde (15): bien \\
	Adrien (14): bien sauf des hésitations sur la recherche d'extremums sur compact/ouvert\\
	
	\begin{exo}
		Théorème de représentation des formes linéaires.
	\end{exo}

	\begin{exo}
		Extrema de $f(x, y) = x^2 + xy + y^2 + 2x - 2y$
	\end{exo}	
			
	\section*{Colle 2}
	\setcounter{exo}{0}
	GUALDI (12): démo non correctement sue. erreurs élémentaires dans les équivalents\\
	Bérenger (12): démo non correctement sue. confusion sur la recherche des points critiques\\
	
	\begin{exo}
		Cours: que peut-on dire d'un extremum local d'une fonction différentiable? (démo)
	\end{exo}

	\begin{exo}
		Déterminer si une fonction est continue.
	\end{exo}	
	
	\begin{exo}
		Recherche d'extremum.
	\end{exo}	
	
	\section*{Colle 3}
	\setcounter{exo}{0}
	Nathan BILLERY (13): s'emmêle dans les justifications (compacité, ouvert, extremum...)\\
	Théo (17): très bien\\
	
	\begin{exo}
		Cours
	\end{exo}

	\begin{exo}
		Recherche d'extremum.
	\end{exo}	

	\begin{exo}
		Déterminer si une fonction est continue.
	\end{exo}	
	
	\begin{exo}
		Différentielle de $M \longmapsto M^2$?
	\end{exo}	
	\begin{exo}
		Différentielle de $M \longmapsto M^{-1}$? (avec indications)
	\end{exo}
\end{document}
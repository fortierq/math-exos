\documentclass[10pt,a4paper]{article}
\usepackage[utf8]{inputenc}
\usepackage{amsmath}
\usepackage{amsfonts}
\usepackage{amssymb}
\usepackage{graphicx}
\title{Colle PCSI 3: Étude de fonctions et trigonométrie}

\newcounter{question}
\newcommand{\initQ}{\setcounter{question}{0}}
\newenvironment{question}{\addtocounter{question}{1}
	\noindent {\it {Question} \thequestion.\ }}
{\par}
\newcounter{exo}
\newcommand{\Z}{\mathbb{Z}}

\newcommand{\initE}{\setcounter{exo}{0}}
\newenvironment{exo}{\vspace{0.5cm}\setcounter{question}{0}\addtocounter{exo}{1} \noindent \textbf{Exercice \theexo}. \normalsize }{\par}

\begin{document}
	\maketitle

	\section*{Colle 1}
	Robin Sauldubois (cours: 5, exo: 6, note: 11): écrit $cos(a + b) = cos(a) + cos(b)$, puis une deuxième fois alors que je lui ai dit d'utiliser une formule d'addition du cours (!) et $(u^n)' = n (u')^{n-1}$\\
	FOLCO Thomas (cours: 8, exo: 9, note: 17): excellent, il a résolu 4 exos (dont deux difficiles) avec assez peu d'indications 
	
	\begin{exo}
		$ln(ab) = ln(a)+ln(b)$?
	\end{exo}
	
	\begin{exo}
		Calculer $\cos(\frac{\pi}{8})$.
	\end{exo}
	
	\begin{exo}
		Résoudre l'équation:
		$$\cos(x) - \sqrt{3} \sin(x) = 1$$
	\end{exo}

	\section*{Colle 2}
	\setcounter{exo}{0}
	Ulysse Studer (cours: 6, exo: 6, note: 12):  arrive 5-10min en retard... écrit $ln(a^n) = ln(n a)$, des équivalences entre des trucs sans rapport, écrit $h(x)$ est dérivable. Peu clair dans ses explications. \\
	FINET Céline (cours: 6, exo: 5, note 11): écrit ln(a+b) = ln(a)+ln(b). Reste bloquée devant l'exo sans rien faire. 
	\begin{exo}
		Formule (et démonstration) d'addition pour $\tan$.
	\end{exo}

	\begin{exo}
		Montrer que ($\ln(x) \leq x - 1$ $\forall x > 0$, puis que) $e \geq (1 + \frac{1}{n})^n$ pour tout $n \geq 2$.
	\end{exo}

	\section*{Colle 3}
	\setcounter{exo}{0}
	Tondu Camille (cours: 4, exo: 4, note: 8): écrit énormément de bêtises: $(a+b)^2 = a^2 + b^2$, besoin de 30min de réflexion et de l'aide pour résoudre $\sin^2(x) = 0$, erreurs de signes, écrit $\cos^2(x) - \sin^2(x) = 1$, ne sait pas montrer $\sin(x) \leq x$ pour $x \geq 0$... \\
	FRECHARD Dorian (cours: 4, exo: 6, note: 10): n'a pas été capable de m'énoncer le théorème de cours demandé (inégalité triangulaire). Erreur dans la dérivé de $u^n$. 
	
	\begin{exo}
		Inégalité triangulaire et démonstration.
	\end{exo}
	
	\begin{exo}
		Résoudre l'équation:
		$$\cos(2x) = \cos^2(x)$$
	\end{exo}
	
	\begin{exo}
		Montrer que, pour tout $x \in [0, \pi]$:
		$$x \geq \sin(x) \geq x - \frac{x^2}{\pi}$$
	\end{exo}
\end{document}
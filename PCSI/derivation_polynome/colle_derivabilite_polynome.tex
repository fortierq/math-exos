\documentclass[10pt,a4paper]{article}
\usepackage[utf8]{inputenc}
\usepackage{amsmath}
\usepackage{amsfonts}
\usepackage{amssymb}
\usepackage{graphicx}
\title{Colle PCSI 23: dérivabilité, polynômes.}

\newcounter{question}
\newcommand{\initQ}{\setcounter{question}{0}}
\newenvironment{question}{\addtocounter{question}{1}
	\noindent {\it {Question} \thequestion.\ }}
{\par}
\newcounter{exo}
\newcommand{\Z}{\mathbb{Z}}

\newcommand{\initE}{\setcounter{exo}{0}}
\newenvironment{exo}{\vspace{0.5cm}\setcounter{question}{0}\addtocounter{exo}{1} \noindent \textbf{Exercice \theexo}. \normalsize }{\par}

\begin{document}
	\maketitle
	
	\section*{Colle 1}
	\setcounter{exo}{0}
	Fréchard Dorian (cours: 4, exo: 5, note: 9): Ne se souvient pas bien de la preuve de cours. Oublie de vérifier que le polynôme est bien dans $\mathbb{C}[X]$. Ne sait plus comment dériver $\frac{1}{x^n}$...\\
	MARION Caroline (cours: 7, exo: 8, note: 15): Bien.\\
	
	\begin{exo}
		Tout polynôme de $\mathbb{R}[X]$ de degré au moins 3 est factorisable dans $\mathbb{R}[X]$.
	\end{exo}

	\begin{exo}
		Tout ce que tu connais sur les extremum locaux?
	\end{exo}
	
	\begin{exo}
		Mq $x \in \mathbb{R} \longmapsto x^2 \sin(\frac{1}{x})$ (0 en 0) est dérivable mais pas $C^1$. 
	\end{exo}	
	
	\section*{Colle 2}
	\setcounter{exo}{0}
	LAGNEAUX (cours: 5, exo: 8, note: 13): ne se souvient pas du thm des bornes. Bien sinon.\\	
	TURCK Bertrand (cours: 5, exo: 8, note: 13): ne connaît pas le thm des bornes, ni le théorème de condition nécessaire d'extremum local.\\
		
	\begin{exo}
		Reltionas entre coefficients et racines d'un polynôme scindé (note pour les colleurs : seuls le produit et la somme des racines est au programme de PCSI). 
	\end{exo}		

	\begin{exo}
		Division euclidienne des polynômes?
	\end{exo}

	\begin{exo}
		Dérivée $n$ième de $x \longmapsto x^2 \ln(x)$?
	\end{exo}
	
	\begin{exo}
		(Darboux) Soit $f : I \longrightarrow \mathbb{R}$ dérivable, $a, b \in I$.\\
		
		\begin{enumerate}
			\item On suppose $f'(a) < 0 < f'(b)$ . Mq $\exists c$, $f'(c) = 0$.
			\item Soit $f'(a) <  \lambda < f'(b)$. Soit $g(x) = f(x) - \lambda x$.\\
			Considérer un minimum de $g$ pour montrer que $\exists c$, $f'(c) = \lambda$.  
		\end{enumerate}	
	\end{exo}

	\section*{Colle 3}
	\setcounter{exo}{0}
	PANIER Estielle (cours: 7, exo: 5, note: 12): bien pour la preuve, peu précise sur l'exo.\\
	MIGOT Thomas (cours: 8, exo: 8, 16/20): Très bien (exo déjà fait?).\\
	
	\begin{exo}
		Lien entre racines et factorisation, avec son lemme : division euclidienne d'un polynôme par X-a
	\end{exo}		
	
	\begin{exo}
		Thm de Rolle?
	\end{exo}
	
	\begin{exo}
		$f : [-1, 1] \longrightarrow \mathbb{R}$ $C^1$ sur $[-1, 1]$, 2 fois dérivable sur $]-1, 1[$ tq $f(-1) = -1$, $f(0) = 0$, $f(1) = 1$.\\
		Mq $\exists c \in ]-1, 1[$, $f''(c) = 0$.
	\end{exo}	
\end{document}
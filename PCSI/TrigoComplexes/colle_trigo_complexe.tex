\documentclass[10pt,a4paper]{article}
\usepackage[utf8]{inputenc}
\usepackage{amsmath}
\usepackage{amsfonts}
\usepackage{amssymb}
\usepackage{graphicx}
\title{Colle PCSI 3: Étude de fonctions et trigonométrie}

\newcounter{question}
\newcommand{\initQ}{\setcounter{question}{0}}
\newenvironment{question}{\addtocounter{question}{1}
	\noindent {\it {Question} \thequestion.\ }}
{\par}
\newcounter{exo}
\newcommand{\Z}{\mathbb{Z}}

\newcommand{\initE}{\setcounter{exo}{0}}
\newenvironment{exo}{\vspace{0.5cm}\setcounter{question}{0}\addtocounter{exo}{1} \noindent \textbf{Exercice \theexo}. \normalsize }{\par}

\begin{document}
	\maketitle

	\section*{Colle 1}
	Clément GUILLAUME-SAGE (cours:7, exo:7, note: 14): bien, a besoin d'aide pour les exos.\\
	MARTI Sébastien (cours: 5, exo:6, note: 11): écrit $\forall z, ~\vert z + z' \vert = \vert z \vert + \vert z' \vert$, n'arrive pas à finir la preuve du thm de cours.\\
	\begin{exo} 
		\begin{itemize}
			\item Compatibilité du module avec les opérations (§ II.2)
			\item Enoncer l'inégalité triangulaire et cas d'égalité. Dessin. 
		\end{itemize}
	\end{exo}
	
	\begin{exo}
		Calculer $\cos(\frac{\pi}{8})$.
	\end{exo}
	
	\begin{exo}
		Résoudre l'équation:
		$$\cos(x) - \sqrt{3} \sin(x) = 1$$
	\end{exo}

	\section*{Colle 2}
	\setcounter{exo}{0}
	FRICK Jonas (cours: 6, exo: 5, note: 11): n'a aucune idée de comment prouver $\ln(x) \leq x - 1$. Ne sais pas faire un tableau de variations...\\
	MARION Caroline (cours: 5, exo: 5, note: 10): écrit $lim_{+\infty} (x-ln(x)) = - \infty$, puis $0$, ne sais pas faire un tableau de variations. écrit $(ln(x))^n = n ln(x)$.
	
	\begin{exo}
		\begin{itemize}
			\item Compatibilité de la conjugaison avec les opérations (§ I.4)
			\item Tout ce que tu sais sur l'argument d'un nombre complexe? Définitions, propriétés...
		\end{itemize}
	\end{exo}

	\begin{exo}
		Montrer que $\ln(x) \leq x - 1$ $\forall x > 0$, puis que $e \geq (1 + \frac{1}{n})^n$ pour tout $n \geq 2$.
	\end{exo}

	\section*{Colle 3}
	\setcounter{exo}{0}
	Henry Quentin (cours: 6, exo: 7, note: 13): écrit $arg(z^n) = arg(z)^n$, sinon correct.\\
	VICOMTE Romaric (cours: 3, exo: 5, note: 8): ne connaît pas $arg(z z') = arg(z)arg(z')$, ni sa preuve. écrit $cos(a+ib)$, des égalités sans aucun sens.
	
	\begin{exo}
		\begin{itemize}
			\item Argument d'un produit et corollaire (inverse et quotient seulement) (§ II.3)
			\item Tout ce que tu sais sur l'exponentielle d'un nombre complexe? Définitions, propriétés...
		\end{itemize}
	\end{exo}
	
	\begin{exo}
		Résoudre l'équation:
		$$\cos(2x) = \cos^2(x)$$
	\end{exo}
	
	\begin{exo}
		Montrer que, pour tout $x \in [0, \frac{\pi}{2}]$:
		$$x \geq \sin(x) \geq x - \frac{x^2}{\pi} ~ (*)$$
		Ensuite, montrer que $(*)$ reste vraie pour tout $x \in [0, {\pi}]$, sans nouvelle étude de fonction.
	\end{exo}
\end{document}
\documentclass[10pt,a4paper]{article}
\usepackage[utf8]{inputenc}
\usepackage{amsmath}
\usepackage{amsfonts}
\usepackage{amssymb}
\usepackage{graphicx}
\title{Colle PCSI 6: Nombres complexes et sommes}

\newcounter{question}
\newcommand{\initQ}{\setcounter{question}{0}}
\newenvironment{question}{\addtocounter{question}{1}
	\noindent {\it {Question} \thequestion.\ }}
{\par}
\newcounter{exo}
\newcommand{\Z}{\mathbb{Z}}

\newcommand{\initE}{\setcounter{exo}{0}}
\newenvironment{exo}{\vspace{0.5cm}\setcounter{question}{0}\addtocounter{exo}{1} \noindent \textbf{Exercice \theexo}. \normalsize }{\par}

\begin{document}
	\maketitle

	\section*{Colle 1}
	LUHRING (cours: 7, exo: 6, note: 13/20): écrit ln(i) puis ln(nombre négatif)... oublie que l'égalité des arguments est seulement modulo 2pi. Sinon bien.
	
	\begin{exo}
		Formule pour $\sum_p^q \lambda^k$.\\
		Puis: que connais-tu comme transformation complexe?
	\end{exo}

	\begin{exo}
		Résoudre $e^z + 2 e^{-z} = i$ (je les aide en disant de multiplier par $e^z$ et de poser $Z = e^z$).
	\end{exo}
	
	\section*{Colle 2}
	\setcounter{exo}{0}
	MAMEDOV Djémali (cours: 4, exo: 6, note: 10/20): ne se souvient pas des formule d'euler, ni de la méthode pour somme cos(kx). Brouillon.\\
	
	Alizée Brouillard (cours: 9, exo: 7: note: 16/20): cours très bien connu, se souvient parfaitement de la démonstration de $\sum_{k=1}^{n} cos(k x)$. 
	
	\begin{exo}
		Formule pour $\sum_p^q k$.\\
		Puis: Racines n -ièmes complexes?
	\end{exo}
	
	\begin{exo}
		Calculer $\sum_{k=1}^{n} cos(k x)$ puis $\sum_{k=1}^{n} k cos(k x)$.
	\end{exo}	
	
	\begin{exo}
		Montrer que trois points $A, B, C$ d'affixes $a, b, c$ forment un triangle équilatéral ssi $a + jb + c j^2 = 0$ ou $a + c j + b j^2  = 0$
	\end{exo}
	
	\section*{Colle 3}
	\setcounter{exo}{0}
	MIGOT (cours: 6, exo: 6, note: 12/20): écrit $exp(ab) =exp(a) exp(b)$.\\
	Bouaza Yakoub (cours: 4, exo: 6, note: 10/20): approximatif sur les racines n ièmes. Peu familier avec l'exponentielle et ses formules.
	
	\begin{exo}
		Forme complexe d'une rotation\\
		Formule de Moivre et d'Euler?
	\end{exo}
	
	\begin{exo}
		Somme et produit des racines n ième de l'unité.
	\end{exo}
	
	\begin{exo}
		Nature de la transformation $z \longmapsto \frac{\sqrt{3}}{2} z + \frac{i}{2} z$?
	\end{exo}
	
\end{document}
\documentclass[10pt,a4paper]{article}
\usepackage[utf8]{inputenc}
\usepackage{amsmath}
\usepackage{amsfonts}
\usepackage{amssymb}
\usepackage{graphicx}
\title{Colle PCSI 8: Sommes, fonctions trigonométriques et hyperboliques.}

\newcounter{question}
\newcommand{\initQ}{\setcounter{question}{0}}
\newenvironment{question}{\addtocounter{question}{1}
	\noindent {\it {Question} \thequestion.\ }}
{\par}
\newcounter{exo}
\newcommand{\Z}{\mathbb{Z}}

\newcommand{\initE}{\setcounter{exo}{0}}
\newenvironment{exo}{\vspace{0.5cm}\setcounter{question}{0}\addtocounter{exo}{1} \noindent \textbf{Exercice \theexo}. \normalsize }{\par}

\begin{document}
	\maketitle
	
	
	\section*{Colle 1}
	\setcounter{exo}{0}
	TRAVAILLOT (cours: 5, exo: 6, note: 11/20): peu rigoureux (parle de $f(t)$ au lieu de $f$...), ne dit pas qu'il y a un probleme de dérivabilité en -1, 1. \\
	FOLCO (cours: 7, exo: 8, note: 15/20): Bien
	
	\begin{exo}
		Dérivabilité de la fonction $\arcsin$. 
	\end{exo}
	
	\begin{exo}
		Théorème de l'image directe: énoncé, contre-exemple si $f$ n'est pas continue? 
	\end{exo}

	\begin{exo}
		Montrer que $f: x \longmapsto \frac{x}{1-x^2}$ est bijective de $]-1, 1[$ dans $\mathbb{R}$ et exprimer sa bijection réciproque.
	\end{exo}	

	\begin{exo}
		Mq $\forall x \geq 0$, $ \arctan(x) \geq \frac{x}{x^2 + 1}$.
	\end{exo}

	\section*{Colle 2}
	\setcounter{exo}{0}
	TURCK (cours: 9, exo: 8, note: 17): Très bien\\
	FINET (cours: 6, exo:  5, note: 11): perdue sur l'exercice $\arcsin(\tan(x)) = x$ et les implications/équivalences. Du mal avec les dessins.
	
	\begin{exo}
	    $\forall t \in [-1, 1]$, $\sin(\arccos(t)) = \sqrt{1 - t^2}$.
	\end{exo}

	\begin{exo}
		Dessin des fonctions circulaires et inverses.
	\end{exo}
		
	\begin{exo}
		Résoudre $\arcsin(\tan(x)) = x$.
	\end{exo}
			
	\section*{Colle 3}
	\setcounter{exo}{0}
	VERMOT (cours: 6/10, exo: 5/10, note: 11): peu rigoureux, ne fait pas attention à une division par 0. Me dit que continue $\implies$ dérivable. Problème d'équivalences.\\
	FRECHARD (cours: 4/10, exo:4/10, note: 8): mélange un peu tout, dit beaucoup de choses qui n'ont pas de sens. ne connaît pas bien la démo de cours.
	
	\begin{exo}
		$\forall t > 0, \arctan(t)+ \arctan(\frac{1}{t}) = \frac{\pi}{2}$, par étude de fonctions.
	\end{exo}

	\begin{exo}
		Définition et propriétés de la fonction réciproque d'une fonction bijective.
	\end{exo}

	\begin{exo}
		Calculer $\arcsin \cos \frac{7 \pi}{4}$ (= $\frac{\pi}{4}$).
	\end{exo}
	
\end{document}
\documentclass[10pt,a4paper]{article}
\usepackage[utf8]{inputenc}
\usepackage{amsmath}
\usepackage{amsfonts}
\usepackage{amssymb}
\usepackage{graphicx}
\title{Colle PCSI 25: EV.}

\newcounter{question}
\newcommand{\initQ}{\setcounter{question}{0}}
\newenvironment{question}{\addtocounter{question}{1}
	\noindent {\it {Question} \thequestion.\ }}
{\par}
\newcounter{exo}
\newcommand{\Z}{\mathbb{Z}}

\newcommand{\initE}{\setcounter{exo}{0}}
\newenvironment{exo}{\vspace{0.5cm}\setcounter{question}{0}\addtocounter{exo}{1} \noindent \textbf{Exercice \theexo}. \normalsize }{\par}

\begin{document}
	\maketitle
	
	
	\section*{Colle 1}
	\setcounter{exo}{0}
	COURIOL Clément (cours: 5, exo: 7, note: 12): ne se souvient plus du théorème sur l'image d'une base par un isomorphisme. \\
	
	\begin{exo}
		 Formules de probabilités composées (versions pour 2 et pour n
	\end{exo}

	\begin{exo}
		Tout ce que tu connais sur les opérations sur les applications linéaires?
	\end{exo}

	\begin{exo}
		Noyau et image de la dérivation sur les polynômes?
	\end{exo}	

	\begin{exo}
		Est-ce que $X^2 + 1$, $X^2+ X - 1$, $X^2 + X$ est une base de $\mathbb{R}_2[X]$?
	\end{exo}	
	
	\section*{Colle 2}
	\setcounter{exo}{0}
	GUES Flora (cours: 6, exo: 6, note: 12): écrit, si $f$ linéaire, $f(x) = 0$ $\implies$ $x = 0$.
	
	\begin{exo}
		 Formules de Bayes
	\end{exo}		

	\begin{exo}
		Tout ce que tu connais sur les bases.
	\end{exo}

	\begin{exo}
		Est-ce que $ker(f \circ f) \subset ker(f)$? l'inverse? im?
	\end{exo}	

	\section*{Colle 3}
	\setcounter{exo}{0}
	Collilieux Mathieu (cours: 7, exo: 8, note: 15): bien\\
	
	
	\begin{exo}
		 Formules des probabilités totales
	\end{exo}		
	
	\begin{exo}
		Définition et prop de VA indép?
	\end{exo}

	\begin{exo}
		Mq $(1)$, $(n^2)$, $(n^3)$ sont indépendantes.
	\end{exo}
\end{document}	
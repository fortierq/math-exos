\documentclass[10pt,a4paper]{article}
\usepackage[utf8]{inputenc}
\usepackage{amsmath}
\usepackage{amsfonts}
\usepackage{amssymb}
\usepackage{graphicx}
\title{Colle PCSI 20: dénombrement et continuité.}

\newcounter{question}
\newcommand{\initQ}{\setcounter{question}{0}}
\newenvironment{question}{\addtocounter{question}{1}
	\noindent {\it {Question} \thequestion.\ }}
{\par}
\newcounter{exo}
\newcommand{\Z}{\mathbb{Z}}

\newcommand{\initE}{\setcounter{exo}{0}}
\newenvironment{exo}{\vspace{0.5cm}\setcounter{question}{0}\addtocounter{exo}{1} \noindent \textbf{Exercice \theexo}. \normalsize }{\par}

\begin{document}
	\maketitle
	
	
	\section*{Colle 1}
	\setcounter{exo}{0}
	TRAVAILLOT (cours: 7, exo: 7, note: 14): oublie le nb de partie de taille $k$. Mais présente bien ses idées.\\
	FINET Céline (cours: 7, exo: 7, note: 14): gros soucis de compréhension avec $\emptyset$. Sinon bien. \\
	\begin{exo}
		Cardinal des fonctions (injectives) de $E$ dans $F$
	\end{exo}
	
	\begin{exo}
		Arrangement: définition, nombre de $p$-arrangements, nombre d'injection de $E$ dans $F$ est le nombre de $\vert E \vert$ - arrangements sur $F$, 2-arrangements de $\lbrace 1, 2, 3 \rbrace$.
	\end{exo}

	\begin{exo}
		Soit E un ensemble à n éléments. Combien y a-t-il de parties X et Y de E telles que
		$X \subset Y$ ?
	\end{exo}
	
	\begin{exo}
		Formule de Vandermonde?
	\end{exo}

	\section*{Colle 2}
	\setcounter{exo}{0}
	VERMOT (Cours: 3, exo: 4, note: 7): démo de cours non sue. Ne connaît pas le nombre $2^n$ de ss ens, ni le nombre de sous-ens de taille $k$...\\
	Ines DJEBRA (cours: 5, exo: 6, note: 11): ne connait plus la définition du coeff binomial. Se trompe dans des choses simples.
	
	\begin{exo}
		Formule de Poincaré.
	\end{exo}		

	\begin{exo}
		Combien y a t-il de sous-ensembles de taille pair d'un ensemble à $n$ éléments? Utiliser 2 méthodes.
	\end{exo}

	\begin{exo}
		Nombre de façon d'obtenir 1 as exactement au poker?
	\end{exo}	

	\section*{Colle 3}
	\setcounter{exo}{0}
	VICOMTE romaric (cours: 5, exo: 6, note: 11): ne se souvient pas du nb de sous ensemble d'un ens, ni de la formule du binôme de Newton.\\
	
	\begin{exo}
		Thm composition limites
	\end{exo}
	
	\begin{exo}
		Permutations et combinaisons.
	\end{exo}	

	\begin{exo}
		Soit $E = \lbrace 1, ..., n \rbrace$. Calculer $\sum_{X \subset E} \vert X \vert$, $\sum_{X, Y \subset E} \vert X \cap Y \vert$
	\end{exo}		
	
\end{document}
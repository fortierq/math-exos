\documentclass[10pt,a4paper]{article}
\usepackage[utf8]{inputenc}
\usepackage{amsmath}
\usepackage{amsfonts}
\usepackage{amssymb}
\usepackage{graphicx}
\title{Colle PCSI 5: Trigonométrie et nombres complexes}

\newcounter{question}
\newcommand{\initQ}{\setcounter{question}{0}}
\newenvironment{question}{\addtocounter{question}{1}
	\noindent {\it {Question} \thequestion.\ }}
{\par}
\newcounter{exo}
\newcommand{\Z}{\mathbb{Z}}

\newcommand{\initE}{\setcounter{exo}{0}}
\newenvironment{exo}{\vspace{0.5cm}\setcounter{question}{0}\addtocounter{exo}{1} \noindent \textbf{Exercice \theexo}. \normalsize }{\par}

\begin{document}
	\maketitle

	\section*{Colle 1}
	BELLONCLE (cours: 7, exo: 6, note: 13): assez bien\\
	LINIGER Kylian (cours: 7, exo: 4, note: 11): très lent sur les exos. 
	
	\begin{exo} 
		\begin{itemize}
				\item Argument d'un produit et corollaire: $arg(\frac{z}{z'})$
			\item Enoncer l'inégalité triangulaire et cas d'égalité. Dessin. 
		\end{itemize}
	\end{exo}
	
	\begin{exo}
		Calculer $\cos(\frac{\pi}{8})$.
	\end{exo}

	\begin{exo}
		Résoudre $e^z + 2 e^{-z} = i$ (je les aide en disant de multiplier par $e^z$ et de poser $Z = e^z$).
	\end{exo}
	
	\section*{Colle 2}
	\setcounter{exo}{0}
	BONNOT (cours: 7, exo: 6, note: 12): Clair dans sa présentation. Ecrit $sin(ix)$. Me dit que $\forall x \in \mathbb{R}$, $x^2 \geq x$. \\
	LAGNEAUX Nicolas (cours: 7, exo: 8, note: 15): Très bien. 
	
	\begin{exo}
		\begin{itemize}
			\item Formulaire des équations du deuxième degré à coefficient complexes
			\item Tout ce que tu sais sur l'argument d'un nombre complexe? Définitions, propriétés...
		\end{itemize}
	\end{exo}
	
	\begin{exo}
		Calculer $\int_{-\pi}^{\pi} \sin^5(t)dt$.
	\end{exo}
	\begin{exo}
		\begin{enumerate}
			\item Montrer $\forall n \in \mathbb{N}$, $x \notin \pi \mathbb{Z}$ :
			$$\sum_{k=0}^{n} \cos(2kx) = \frac{\sin((n+1)x) \cos(nx)}{\sin(x)}$$
			\item Montrer:
			$$\vert \sin(x) \vert \geq \frac{1 - cos(2x)}{2}$$
			\item Déduire:
			$$\sum_{k=1}^{n} \vert \sin(k) \vert \geq \frac{n}{2}  - \frac{1}{2 \sin(1)}$$
		\end{enumerate}
	\end{exo}
	
	\section*{Colle 3}
	\setcounter{exo}{0}
	PONCOT Thomas (cours: 3, exo: 7, note: 10): ne connait pas les racines nièmes, ni la démo. Ecrit vraiment à l'arrache. Bien sur les exos.\\
	JAGHRI Théo (cours: 6, exo: 5, note: 11): mauvais compréhension des congruences. 
	
	
	\begin{exo}
		\begin{itemize}
			\item Liste des racines n-ièmes de l'unité
			\item Tout ce que tu sais sur l'exponentielle d'un nombre complexe? Définitions, propriétés...
		\end{itemize}
	\end{exo}

	\begin{exo}
		Trouver les $n \in \mathbb{N}$ tq $(1 + i)^n \in \mathbb{R}$.
	\end{exo}

	\begin{exo}
		Résoudre l'équation:
		$$\cos(x) - \sqrt{3} \sin(x) = 1$$
	\end{exo}
%	
%	\begin{exo}
%		Résoudre l'équation:
%		$$\cos(2x) = \cos^2(x)$$
%	\end{exo}
\end{document}
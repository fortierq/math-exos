\documentclass[10pt,a4paper]{article}
\usepackage[utf8]{inputenc}
\usepackage{amsmath}
\usepackage{amsfonts}
\usepackage{amssymb}
\usepackage{graphicx}
\title{Colle PCSI 27: Probas et dimension finie.}

\newcounter{question}
\newcommand{\initQ}{\setcounter{question}{0}}
\newenvironment{question}{\addtocounter{question}{1}
	\noindent {\it {Question} \thequestion.\ }}
{\par}
\newcounter{exo}
\newcommand{\Z}{\mathbb{Z}}

\newcommand{\initE}{\setcounter{exo}{0}}
\newenvironment{exo}{\vspace{0.5cm}\setcounter{question}{0}\addtocounter{exo}{1} \noindent \textbf{Exercice \theexo}. \normalsize }{\par}

\begin{document}
	\maketitle
	
	
	\section*{Colle 1}
	\setcounter{exo}{0}
	LAABI Amine (cours: 3, exo: 7, note: 10): ne connaît pas la définition d'ensembles indépendants, et pas très bien le théorème de caractérisation des bases avec la dimension. \\
	
	\begin{exo}
		  Formules des probabilités totales (versions sans et avec conditionnement).
	\end{exo}

	\begin{exo}
		Que connais tu sur les bases extraites et incomplètes?
	\end{exo}

	\begin{exo}
		Est-ce qu'un événement peut être indépendant de lui-même?
	\end{exo}	

	\begin{exo}
		Est-ce que $X^2 + 1$, $X^2+ X - 1$, $X^2 + X$ est une base de $\mathbb{R}_2[X]$?
	\end{exo}	
	
	\section*{Colle 2}
	\setcounter{exo}{0}
	LINIGER (cours: 7, exo: 7, note: 14): Assez bien\\
	TRAVAILLOT (cours: 6, exo: 6, note: 12) : assez lent.\\
	
	\begin{exo}
		  Caractérisation des sommes directes par l'unicité de la décomposition.
	\end{exo}		

	\begin{exo}
		Tout ce que tu connais sur la dimension d'un EV.
	\end{exo}

	\begin{exo}
		Est-ce que l'ensemble des fonctions $\lbrace$ ayant une limite en $+ \infty$, bornées, monotones, impaires $\rbrace$ sont des EV? 
	\end{exo}	

	\begin{exo}
		Est-ce que $ker(f \circ f) \subset ker(f)$? l'inverse? im?
	\end{exo}	

	\section*{Colle 3}
	\setcounter{exo}{0}
	NACHIN (cours: 8, exo: 7, note: 15): Bien\\
	VERMOT (cours: 6, exo: 6, note: 12: assez lent.\\
	
	\begin{exo}
		  Caractérisation des bases utilisant la dimension,
	\end{exo}		
	
	\begin{exo}
		Supplémentaires
	\end{exo}

	\begin{exo}
		Quelle est la proba d'obtenir une paire au poker? (32 cartes)
	\end{exo}
	
	\begin{exo}
		Mq $(1)$, $(n^2)$, $(n^3)$ sont indépendantes.
	\end{exo}
\end{document}	
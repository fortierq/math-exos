\documentclass[10pt,a4paper]{article}
\usepackage[utf8]{inputenc}
\usepackage{amsmath}
\usepackage{amsfonts}
\usepackage{amssymb}
\usepackage{graphicx}
\title{Colle PCSI 15: suites.}

\newcounter{question}
\newcommand{\initQ}{\setcounter{question}{0}}
\newenvironment{question}{\addtocounter{question}{1}
	\noindent {\it {Question} \thequestion.\ }}
{\par}
\newcounter{exo}
\newcommand{\Z}{\mathbb{Z}}

\newcommand{\initE}{\setcounter{exo}{0}}
\newenvironment{exo}{\vspace{0.5cm}\setcounter{question}{0}\addtocounter{exo}{1} \noindent \textbf{Exercice \theexo}. \normalsize }{\par}

\begin{document}
	\maketitle
	
	\section*{Colle 1}
	\setcounter{exo}{0}
	MARION Caroline (10/20): Trop d'hésitations dans le cours. Ecrit 2*$3^n$ = $6^n$.\\
	SPADETTO (13/20) : Assez bien.\\
	
	\begin{exo}
		Théorème de comparaison (première version)
	\end{exo}
	
	\begin{exo}
		Théorème composition limites.
	\end{exo}
	
	\begin{exo}
		Terme général de $u_{n+1} = 3 u_n + 2$, $u_0 = 1$?
	\end{exo}

	\begin{exo}
		On définit $u_0 \in [0, \frac{\pi}{2}]$, $u_{n+1} = \sin(u_n)$. Est-ce que $u_n$ a une limite? laquelle?
	\end{exo}
	
	\section*{Colle 2}
	\setcounter{exo}{0}
	TURCK Bertrand (15/20): Bien.\\
	DHIFAOUI (13/20): erreurs de calculs.\\
	
	\begin{exo}
		Théorème des suites adjacentes (en admettant dans un premier temps que $\forall (p, q) \in \mathbb{N}^2, \, u_p \leq v_q$ ; le colleur pourra en demander la preuve
	\end{exo}

	\begin{exo}
		Théorème des gendarmes?
	\end{exo}

	\begin{exo}
		Terme général de $2u_{n+2} = 6 u_{n+1} + 8 u_n$, $u_1 = 5$, $u_0 = -1$?
	\end{exo}
		
	
	\section*{Colle 3}
	\setcounter{exo}{0}
	MIGOT Thomas (14/20): Bien.\\
	STEFFANN Axelle (12/20): Ecrit 2*$3^n$ = $6^n$.\\
	
	\begin{exo}
		Unicité de la limite d'une suite convergente
	\end{exo}
	
	\begin{exo}
		Thm limite monotone
	\end{exo}	
	
	\begin{exo}
		Limite de $u_0 \in \mathbb{R}^+$, $u_{n+1} = \frac{1}{2} (u_n +\frac{a}{u_n})$?
	\end{exo}
	
\end{document}
\documentclass[10pt,a4paper]{article}
\usepackage[utf8]{inputenc}
\usepackage{amsmath}
\usepackage{amsfonts}
\usepackage{amssymb}
\usepackage{graphicx}
\title{Colle PCSI 7: Sommes et produits}

\newcounter{question}
\newcommand{\initQ}{\setcounter{question}{0}}
\newenvironment{question}{\addtocounter{question}{1}
	\noindent {\it {Question} \thequestion.\ }}
{\par}
\newcounter{exo}
\newcommand{\Z}{\mathbb{Z}}

\newcommand{\initE}{\setcounter{exo}{0}}
\newenvironment{exo}{\vspace{0.5cm}\setcounter{question}{0}\addtocounter{exo}{1} \noindent \textbf{Exercice \theexo}. \normalsize }{\par}

\begin{document}
	\maketitle

	\section*{Colle 1}
	
	\begin{exo}
		Formule de Bernouilli. Propriétés sur les sommes?
	\end{exo}
	
	\begin{exo}
		Somme et produit des racines n ième de l'unité. Somme des distances des racines nièmes à 1?
	\end{exo}

	\begin{exo}
		Calculer $\sum \binom{n}{k} \cos(kx)$, $\sum_{k=1}^{n} k cos(k x)$.
	\end{exo}	
	
	\section*{Colle 2}
	\setcounter{exo}{0}
	NACHIN (cours: 8, exo: 9, note: 17): très bien.
	
	\begin{exo}
		Formule pour $\sum_p^q \lambda^k$. Somme des $k$? somme des $k^2$?
	\end{exo}
	
	\begin{exo}
		Calculer:
		$$\sum k k!$$ $$\sum k 2^{k-1}$$
	\end{exo}
	\begin{exo}
		\begin{enumerate}
			\item Montrer $\forall n \in \mathbb{N}$, $x \notin \pi \mathbb{Z}$ :
			$$\sum_{k=0}^{n} \cos(2kx) = \frac{\sin((n+1)x) \cos(nx)}{\sin(x)}$$
			\item Montrer:
			$$\vert \sin(x) \vert \geq \frac{1 - cos(2x)}{2}$$
			\item Déduire:
			$$\sum_{k=1}^{n} \vert \sin(k) \vert \geq \frac{n}{2}  - \frac{1}{2 \sin(1)}$$
		\end{enumerate}
	\end{exo}
	
	\section*{Colle 3}
	\setcounter{exo}{0}
	Clément Couriol (cours: 4, exo: 5, note: 9): ne connait pas la formule du binôme de Newton. Ecrit $\sin(i)$ \\
	
	PANIER Estelle (cours: 6, exo: 5, note: 11): pas mal de petites bêtises pour les exos, mais connaît bien son cours.
	
	\begin{exo}
		Formule de Pascal. Autres propriétés du coeff binomial?
	\end{exo}
	\begin{exo}
		\begin{enumerate}
			\item Soit $S$ un ensemble de taille $n$. Quel est le nombre de sous ensembles de $S$ de taille $k$? Retrouver la formule de Pascal.
			\item Montrer qu'il y a autant de sous ensembles de taille pair que de sous ensembles de taille impair, dans $S$.
			\item Combien y a t-il de sous ensembles de $S$ de taille un multiple de 3?
		\end{enumerate}
	\end{exo}
	
\end{document}
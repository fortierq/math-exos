\documentclass[10pt,a4paper]{article}
\usepackage[utf8]{inputenc}
\usepackage{amsmath}
\usepackage{amsfonts}
\usepackage{amssymb}
\usepackage{graphicx}
\title{Colle PCSI 11: intégrale.}

\newcounter{question}
\newcommand{\initQ}{\setcounter{question}{0}}
\newenvironment{question}{\addtocounter{question}{1}
	\noindent {\it {Question} \thequestion.\ }}
{\par}
\newcounter{exo}
\newcommand{\Z}{\mathbb{Z}}

\newcommand{\initE}{\setcounter{exo}{0}}
\newenvironment{exo}{\vspace{0.5cm}\setcounter{question}{0}\addtocounter{exo}{1} \noindent \textbf{Exercice \theexo}. \normalsize }{\par}

\begin{document}
	\maketitle
	
	
	\section*{Colle 1}
	\setcounter{exo}{0}
	TONDU (cours: 6/10, exo: 6/10, note: 12/20): très lent mais sinon ok.\\
	LINIGER (Cours: 8/10, Exo: 5/10, Note: 13/20): bien sur le cours mais un peu perdu sur l'exo (et pas mal d'erreurs de calculs).\\
	
	\begin{exo}
		IPP
	\end{exo}
	
	\begin{exo}
		Méthode variation de la constante?
	\end{exo}
	
	\begin{exo}
		$\int_{0}^{\pi} \exp(t) \sin(3t) dt$?
	\end{exo}	
	
	\begin{exo}
		Soit $I_{p, q} = \int_{0}^{1} t^p (1 - t)^q dt$. 
		\begin{itemize}
			\item Mq $$I_{p, q} = \frac{q}{p+1} I_{p+1, q-1}$$
			\item Mq $$I_{p, q} = \frac{p! q!}{(p+q+1)!}$$
			\item Calculer $$\sum_{k=0}^{q} \binom{q}{k} \frac{(-1)^k}{p+k+1}$$
		\end{itemize}
	\end{exo}
	
	\section*{Colle 2}
	\setcounter{exo}{0}
	SAULDUBOIS (cours: 5/10, exo: 7/10, note: 12/20): erreur primitive de $1/x^2$. Erreur formule chgt variable et dérivée de $\cos$.\\
	LAGNEAUX (cours: 3/10, exo: 5/10, note: 8/20): ne connaît pas la démonstration de cours. Beaucoup de bêtises.\\
	
	\begin{exo}
		Eq diff homogène y' - a y = 0
	\end{exo}

	\begin{exo}
		Méthode de calcul d'une primitive $\frac{1}{P(x)}$, $P$ de degré 2.
	\end{exo}
	
	\begin{exo}
		Calculer $\int_{-1}^{1} t^2 \sqrt{1 - t^2} dt$ (= $\frac{\pi}{8}$)
	\end{exo}
			
	\begin{exo}
		Wallis:
		$$I_n = \int_{0}^{\pi} \sin^n t dt$$
		\begin{itemize}
			\item Mq $I_n = \int \cos^n t dt$.
			\item Mq $I_{n+2} = \frac{n+1}{n+2} I_n$.
			\item $I_n = ?$
			\item Mq $n I_n I_{n-1} = \frac{\pi}{2}$.
		\end{itemize}
	\end{exo}
			
	\section*{Colle 3}
	\setcounter{exo}{0}
	STUDER (cours: 5/10, exo: 6/10, Note: 11/20): ne connaît pas la définition de la dérivée. Dit continue $\implies$ dérivable. Très brouillon.\\
	
	\begin{exo}
		Unicité des primitives, à addition d'une constante près.
	\end{exo}

	\begin{exo}
		Formule de chgt de variable?
	\end{exo}
	
	\begin{exo}
		Calculer $\lim_0 \frac{1}{x} \int_{0}^{x} f(t)dt$? Interprétation? 
	\end{exo}
	
	\begin{exo}
		Primitive de $x \longmapsto \frac{1}{x^2 - x - 1}$?
	\end{exo}

\end{document}
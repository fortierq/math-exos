\documentclass[10pt,a4paper]{article}
\usepackage[utf8]{inputenc}
\usepackage{amsmath}
\usepackage{amsfonts}
\usepackage{amssymb}
\usepackage{graphicx}
\title{Colle PCSI 28: Probas et dimension finie 2.}

\newcounter{question}
\newcommand{\initQ}{\setcounter{question}{0}}
\newenvironment{question}{\addtocounter{question}{1}
	\noindent {\it {Question} \thequestion.\ }}
{\par}
\newcounter{exo}
\newcommand{\Z}{\mathbb{Z}}

\newcommand{\initE}{\setcounter{exo}{0}}
\newenvironment{exo}{\vspace{0.5cm}\setcounter{question}{0}\addtocounter{exo}{1} \noindent \textbf{Exercice \theexo}. \normalsize }{\par}

\begin{document}
	\maketitle
% TODO: exo 95 MPSIDDL Applications linéaires
	
	\section*{Colle 1}
	\setcounter{exo}{0}
	DE THOMASIS Léna (cours: 7, exo: 6, note: 13): ne connaît pas la définition de $E + F$\\
	GUILLAUME-SAGE Clément (cours: 7, exo: 8, note: 15): parle de $dim(f)$, $f$ endomorphisme.\\
	
	\begin{exo}
		  Propriétés des projecteurs
	\end{exo}

	\begin{exo}
		Enoncer le thm du rang.
	\end{exo}

	\begin{exo}
		Soient E un K -espace vectoriel et p, q deux projecteurs de E qui commutent.
		Montrer que p $\circ$ q est un projecteur de E . En déterminer noyau et image. (Ker (p $\circ$ q) = Ker(p) + Ker(q) et Im(p $\circ$ q) = Im(p) $\cap$ Im(q)).
	\end{exo}	
	
	\section*{Colle 2}
	\setcounter{exo}{0}
	SEJOURNET Baptiste (cours: 5, exo: 7, note: 12): met beaucoup de temps à se souvenir de la démo de cours.\\
	HENRY(cours: 7, exo: 8, note: 15): parle de $f^{-1}$ sans vérifier que $f$ est bijective.\\
	
	\begin{exo}
		   Rang de composées d'applications linéaires. 
	\end{exo}		

	\begin{exo}
		Formule de Bayes?
	\end{exo}

	\begin{exo}
		Quelle est la proba d'obtenir une paire au poker? (32 cartes)
	\end{exo}
	
	\begin{exo}
		Soient f, g $\in$ L(E).\\
		Mq $\vert rg(f ) - rg(g) \vert \leq rg(f + g) \leq rg(f ) + rg(g)$
	\end{exo}	

	\section*{Colle 3}
	\setcounter{exo}{0}
	FOLCO Thomas (cours: 5, exo: 8, note: 13): ne connaît pas le thm des probabilités totales.\\
	
	\begin{exo}
		  Caractérisation des bases utilisant la dimension,
	\end{exo}		
	
	\begin{exo}
		Formule des probas totales?
	\end{exo}

	\begin{exo}
		Soit $f$ endo de $E$ fini tq $rg(f^2) = rg(f)$. Mq $Im(f^2) = Im(f)$, puis $ker(f^2) = ker(f)$ et enfin $ker(f) + im(f) = dim(E)$.
	\end{exo}
	
\end{document}	